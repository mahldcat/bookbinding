\poemchapter{The Low-Down White}

\begin{poemblock}
This is the pay-day up at the mines, when the bearded brutes come down;\\
There's money to burn in the streets to-night, so I've sent my klooch to town,\\
With a haggard face and a ribband of red entwined in her hair of brown.

And I know at the dawn she'll come reeling home with the bottles, one, two, three;\\
One for herself to drown her shame, and two big bottles for me,\\
To make me forget the thing I am and the man I used to be.

To make me forget the brand of the dog, as I crouch in this hideous place;\\
To make me forget once I kindled the light of love in a lady's face,\\
Where even the squalid Siwash now holds me a black disgrace.

Oh, I have guarded my secret well! And who would dream as I speak\\
In a tribal tongue like a rogue unhung, 'mid the ranch-house filth and reek,\\
I could roll to bed with a Latin phrase, and rise with a verse of Greek?

Yet I was a senior prizeman once, and the pride of a college eight;\\
Called to the bar—my friends were true! but they could not keep me straight;\\
Then came the divorce, and I went abroad and "died" on the River Plate.

But I'm not dead yet; though with half a lung there isn't time to spare,\\
And I hope that the year will see me out, and, thank God, no one will care—\\
Save maybe the little slim Siwash girl with the rose of shame in her hair.

She will come with the dawn, and the dawn is near; I can see its evil glow,\\
Like a corpse-light seen through a frosty pane in a night of want and woe;\\
And yonder she comes, by the bleak bull-pines, swift staggering through the snow.

\end{poemblock}
