\poemchapter{The Man from Eldorado}

\begin{poemblock}
 He's the man from Eldorado, and he's just arrived in town,
  In moccasins and oily buckskin shirt.
 He's gaunt as any Indian, and pretty nigh as brown;
  He's greasy, and he smells of sweat and dirt.
 He sports a crop of whiskers that would shame a healthy hog;
  Hard work has racked his joints and stooped his back;
 He slops along the sidewalk followed by his yellow dog,
  But he's got a bunch of gold-dust in his sack.

 He seems a little wistful as he blinks at all the lights,
  And maybe he is thinking of his claim
 And the dark and dwarfish cabin where he lay and dreamed at nights,
  (Thank God, he'll never see the place again!)
 Where he lived on tinned tomatoes, beef embalmed and sourdough bread,
  On rusty beans and bacon furred with mould;
 His stomach's out of kilter and his system full of lead,
  But it's over, and his poke is full of gold.

 He has panted at the windlass, he has loaded in the drift,
  He has pounded at the face of oozy clay;
 He has taxed himself to sickness, dark and damp and double shift,
  He has labored like a demon night and day.
 And now, praise God, it's over, and he seems to breathe again
  Of new-mown hay, the warm, wet, friendly loam;
 He sees a snowy orchard in a green and dimpling plain,
  And a little vine-clad cottage, and it's--Home.


 II.

 He's the man from Eldorado, and he's had a bite and sup,
  And he's met in with a drouthy friend or two;
 He's cached away his gold-dust, but he's sort of bucking up,
  So he's kept enough to-night to see him through.
 His eye is bright and genial, his tongue no longer lags;
  His heart is brimming o'er with joy and mirth;
 He may be far from savory, he may be clad in rags,
  But to-night he feels as if he owns the earth.

 Says he:  "Boys, here is where the shaggy North and I will shake;
  I thought I'd never manage to get free.
 I kept on making misses; but at last I've got my stake;
  There's no more thawing frozen muck for me.
 I am going to God's Country, where I'll live the simple life;
  I'll buy a bit of land and make a start;
 I'll carve a little homestead, and I'll win a little wife,
  And raise ten little kids to cheer my heart."

 They signified their sympathy by crowding to the bar;
  They bellied up three deep and drank his health.
 He shed a radiant smile around and smoked a rank cigar;
  They wished him honor, happiness and wealth.
 They drank unto his wife to be--that unsuspecting maid;
  They drank unto his children half a score;
 And when they got through drinking very tenderly they laid
  The man from Eldorado on the floor.


 III.

 He's the man from Eldorado, and he's only starting in
  To cultivate a thousand-dollar jag.
 His poke is full of gold-dust and his heart is full of sin,
  And he's dancing with a girl called Muckluck Mag.
 She's as light as any fairy; she's as pretty as a peach;
  She's mistress of the witchcraft to beguile;
 There's sunshine in her manner, there is music in her speech,
  And there's concentrated honey in her smile.

 Oh, the fever of the dance-hall and the glitter and the shine,
  The beauty, and the jewels, and the whirl,
 The madness of the music, the rapture of the wine,
  The languorous allurement of a girl!
 She is like a lost madonna; he is gaunt, unkempt and grim;
  But she fondles him and gazes in his eyes;
 Her kisses seek his heavy lips, and soon it seems to him
  He has staked a little claim in Paradise.

 "Who's for a juicy two-step?" cries the master of the floor;
  The music throbs with soft, seductive beat.
 There's glitter, gilt and gladness; there are pretty girls galore;
  There's a woolly man with moccasins on feet.
 They know they've got him going; he is buying wine for all;
  They crowd around as buzzards at a feast,
 Then when his poke is empty they boost him from the hall,
  And spurn him in the gutter like a beast.

 He's the man from Eldorado, and he's painting red the town;
  Behind he leaves a trail of yellow dust;
 In a whirl of senseless riot he is ramping up and down;
  There's nothing checks his madness and his lust.
 And soon the word is passed around--it travels like a flame;
  They fight to clutch his hand and call him friend,
 The chevaliers of lost repute, the dames of sorry fame;
  Then comes the grim awakening--the end.


 IV.

 He's the man from Eldorado, and he gives a grand affair;
  There's feasting, dancing, wine without restraint.
 The smooth Beau Brummels of the bar, the faro men, are there;
  The tinhorns and purveyors of red paint;
 The sleek and painted women, their predacious eyes aglow--
  Sure Klondike City never saw the like;
 Then Muckluck Mag proposed the toast, "The giver of the show,
  The livest sport that ever hit the pike."

 The "live one" rises to his feet; he stammers to reply--
  And then there comes before his muddled brain
 A vision of green vastitudes beneath an April sky,
  And clover pastures drenched with silver rain.
 He knows that it can never be, that he is down and out;
  Life leers at him with foul and fetid breath;
 And then amid the revelry, the song and cheer and shout,
  He suddenly grows grim and cold as death.

 He grips the table tensely, and he says:  "Dear friends of mine,
  I've let you dip your fingers in my purse;
 I've crammed you at my table, and I've drowned you in my wine,
  And I've little left to give you but--my curse.
 I've failed supremely in my plans; it's rather late to whine;
  My poke is mighty weasened up and small.
 I thank you each for coming here; the happiness is mine--
  And now, you thieves and harlots, take it all."

 He twists the thong from off his poke; he swings it o'er his head;
  The nuggets fall around their feet like grain.
 They rattle over roof and wall; they scatter, roll and spread;
  The dust is like a shower of golden rain.
 The guests a moment stand aghast, then grovel on the floor;
  They fight, and snarl, and claw, like beasts of prey;
 And then, as everybody grabbed and everybody swore,
  The man from Eldorado slipped away.


 V.

 He's the man from Eldorado, and they found him stiff and dead,
  Half covered by the freezing ooze and dirt.
 A clotted Colt was in his hand, a hole was in his head,
  And he wore an old and oily buckskin shirt.
 His eyes were fixed and horrible, as one who hails the end;
  The frost had set him rigid as a log;
 And there, half lying on his breast, his last and only friend,
  There crouched and whined a mangy yellow dog.
\end{poemblock}
