\poemchapter{The Ballad of the Brand}

\begin{poemblock}
'Twas up in a land long famed for gold, where women were far and rare,\\
Tellus, the smith, had taken to wife a maiden amazingly fair;\\
Tellus, the brawny worker in iron, hairy and heavy of hand,\\
Saw her and loved her and bore her away from the tribe of a Southern land;\\
Deeming her worthy to queen his home and mother him little ones,\\
That the name of Tellus, the master smith, might live in his stalwart sons.

Now there was little of law in the land, and evil doings were rife,\\
And every man who joyed in his home guarded the fame of his wife;\\
For there were those of the silver tongue and the honeyed art to beguile,\\
Who would cozen the heart from a woman's breast\\
\idt  and damn her soul with a smile.\\
And there were women too quick to heed a look or a whispered word,\\
And once in a while a man was slain, and the ire of the King was stirred;\\
So far and wide he proclaimed his wrath, and this was the law he willed:\\
"That whosoever killeth a man, even shall he be killed."

Now Tellus, the smith, he trusted his wife; his heart was empty of fear.\\
High on the hill was the gleam of their hearth, a beacon of love and cheer.\\
High on the hill they builded their bower,\\
\idt  where the broom and the bracken meet;\\
Under a grave of oaks it was, hushed and drowsily sweet.\\
Here he enshrined her, his dearest saint, his idol, the light of his eye;\\
Her kisses rested upon his lips as brushes a butterfly.\\
The weight of her arms around his neck was light as the thistle down;\\
And sweetly she studied to win his smile, and gently she mocked his frown.\\
And when at the close of the dusty day his clangorous toil was done,\\
She hastened to meet him down the way all lit by the amber sun.

Their dove-cot gleamed in the golden light, a temple of stainless love;\\
Like the hanging cup of a big blue flower was the topaz sky above.\\
The roses and lilies yearned to her,\\
\idt  as swift through their throng she pressed;\\
A little white, fragile, fluttering thing\\
\idt  that lay like a child on his breast.\\
Then the heart of Tellus, the smith, was proud,\\
\idt  and sang for the joy of life,\\
And there in the bronzing summertide he thanked the gods for his wife.

Now there was one called Philo, a scribe, a man of exquisite grace,\\
Carved like the god Apollo in limb, fair as Adonis in face;\\
Eager and winning in manner, full of such radiant charm,\\
Womenkind fought for his favor and loved to their uttermost harm.\\
Such was his craft and his knowledge, such was his skill at the game,\\
Never was woman could flout him, so be he plotted her shame.\\
And so he drank deep of pleasure, and then it fell on a day\\
He gazed on the wife of Tellus and marked her out for his prey.

Tellus, the smith, was merry, and the time of the year it was June,\\
So he said to his stalwart helpers:  "Shut down the forge at noon.\\
Go ye and joy in the sunshine, rest in the coolth of the grove,\\
Drift on the dreamy river, every man with his love."\\
Then to himself:  "Oh, Beloved, sweet will be your surprise;\\
To-day will we sport like children, laugh in each other's eyes;\\
Weave gay garlands of poppies, crown each other with flowers,\\
Pull plump carp from the lilies, rifle the ferny bowers.\\
To-day with feasting and gladness the wine of Cyprus will flow;\\
To-day is the day we were wedded only a twelvemonth ago."

The larks trilled high in the heavens; his heart was lyric with joy;\\
He plucked a posy of lilies; he sped like a love-sick boy.\\
He stole up the velvety pathway--his cottage was sunsteeped and still;\\
Vines honeysuckled the window; softly he peeped o'er the sill.\\
The lilies dropped from his fingers; devils were choking his breath;\\
Rigid with horror, he stiffened; ghastly his face was as death.\\
Like a nun whose faith in the Virgin is met with a prurient jibe,\\
He shrank--'twas the wife of his bosom in the arms of Philo, the scribe.

Tellus went back to his smithy; he reeled like a drunken man;\\
His heart was riven with anguish; his brain was brooding a plan.\\
Straight to his anvil he hurried; started his furnace aglow;\\
Heated his iron and shaped it with savage and masterful blow.\\
Sparks showered over and round him; swiftly under his hand\\
There at last it was finished--a hideous and infamous Brand.

That night the wife of his bosom, the light of joy in her eyes,\\
Kissed him with words of rapture; but he knew that her words were lies.\\
Never was she so beguiling, never so merry of speech\\
(For passion ripens a woman as the sunshine ripens a peach).\\
He clenched his teeth into silence; he yielded up to her lure,\\
Though he knew that her breasts were heaving from the fire of her paramour.\\
"To-morrow," he said, "to-morrow"--he wove her hair in a strand,\\
Twisted it round his fingers and smiled as he thought of the Brand.

The morrow was come, and Tellus swiftly stole up the hill.\\
Butterflies drowsed in the noon-heat; coverts were sunsteeped and still.\\
Softly he padded the pathway unto the porch, and within\\
Heard he the low laugh of dalliance, heard he the rapture of sin.\\
Knew he her eyes were mystic with light that no man should see,\\
No man kindle and joy in, no man on earth save he.\\
And never for him would it kindle.  The bloodlust surged in his brain;\\
Through the senseless stone could he see them, wanton and warily fain.\\
Horrible!  Heaven he sought for, gained it and gloried and fell--\\
Oh, it was sudden--headlong into the nethermost hell. . . .

Was this he, Tellus, this marble?  Tellus . . . not dreaming a dream?\\
Ah! sharp-edged as a javelin, was that a woman's scream?\\
Was it a door that shattered, shell-like, under his blow?\\
Was it his saint, that strumpet, dishevelled and cowering low?\\
Was it her lover, that wild thing, that twisted and gouged and tore?\\
Was it a man he was crushing, whose head he beat on the floor?\\
Laughing the while at its weakness, till sudden he stayed his hand--\\
Through the red ring of his madness flamed the thought of the Brand.

Then bound he the naked Philo with thongs that cut in the flesh,\\
And the wife of his bosom, fear-frantic, he gagged with a silken mesh,\\
Choking her screams into silence; bound her down by the hair;\\
Dragged her lover unto her under her frenzied stare.\\
In the heat of the hearth-fire embers he heated the hideous Brand;\\
Twisting her fingers open, he forced its haft in her hand.\\
He pressed it downward and downward; she felt the living flesh sear;\\
She saw the throe of her lover; she heard the scream of his fear.\\
Once, twice and thrice he forced her, heedless of prayer and shriek--\\
Once on the forehead of Philo, twice in the soft of his cheek.\\
Then (for the thing was finished) he said to the woman:  "See\\
How you have branded your lover!  Now will I let him go free."\\
He severed the thongs that bound him, laughing:  "Revenge is sweet",\\
And Philo, sobbing in anguish, feebly rose to his feet.\\
The man who was fair as Apollo, god-like in woman's sight,\\
Hideous now as a satyr, fled to the pity of night.

\textit{
\idt Then came they before the Judgment Seat,\\
and thus spoke the Lord of the Land:\\
\idt "He who seeketh his neighbor's wife\\
shall suffer the doom of the Brand.\\
\idt Brutish and bold on his brow be it stamped,\\
deep in his cheek let it sear,\\
\idt That every man may look on his shame, and shudder and sicken and fear.\\
\idt He shall hear their mock in the market-place,\\
their fleering jibe at the feast;\\
\idt He shall seek the caves and the shroud of night,\\
and the fellowship of the beast.\\
\idt Outcast forever from homes of men, far and far shall he roam.\\
\idt Such be the doom, sadder than death, of him who shameth a home."
}
\end{poemblock}
