\poemchapter{The Ballad of Hard-Luck Henry}

\begin{poemblock}
 Now wouldn't you expect to find a man an awful crank
 That's staked out nigh three hundred claims, and every one a blank;
 That's followed every fool stampede, and seen the rise and fall
 Of camps where men got gold in chunks and he got none at all;
 That's prospected a bit of ground and sold it for a song
 To see it yield a fortune to some fool that came along;
 That's sunk a dozen bed-rock holes, and not a speck in sight,
 Yet sees them take a million from the claims to left and right?
 Now aren't things like that enough to drive a man to booze?
 But Hard-Luck Smith was hoodoo-proof--he knew the way to lose.

 'Twas in the fall of nineteen four--leap-year I've heard them say--
 When Hard-Luck came to Hunker Creek and took a hillside lay.
 And lo! as if to make amends for all the futile past,
 Late in the year he struck it rich, the real pay-streak at last.
 The riffles of his sluicing-box were choked with speckled earth,
 And night and day he worked that lay for all that he was worth.
 And when in chill December's gloom his lucky lease expired,
 He found that he had made a stake as big as he desired.

 One day while meditating on the waywardness of fate,
 He felt the ache of lonely man to find a fitting mate;
 A petticoated pard to cheer his solitary life,
 A woman with soft, soothing ways, a confidant, a wife.
 And while he cooked his supper on his little Yukon stove,
 He wished that he had staked a claim in Love's rich treasure-trove;
 When suddenly he paused and held aloft a Yukon egg,
 For there in pencilled letters was the magic name of Peg.

 You know these Yukon eggs of ours--some pink, some green, some blue--
 A dollar per, assorted tints, assorted flavors too.
 The supercilious cheechako might designate them high,
 But one acquires a taste for them and likes them by-and-by.
 Well, Hard-Luck Henry took this egg and held it to the light,
 And there was more faint pencilling that sorely taxed his sight.
 At last he made it out, and then the legend ran like this--
 "Will Klondike miner write to Peg, Plumhollow, Squashville, Wis.?"

 That night he got to thinking of this far-off, unknown fair;
 It seemed so sort of opportune, an answer to his prayer.
 She flitted sweetly through his dreams, she haunted him by day,
 She smiled through clouds of nicotine, she cheered his weary way.
 At last he yielded to the spell; his course of love he set--
 Wisconsin his objective point; his object, Margaret.

 With every mile of sea and land his longing grew and grew.
 He practised all his pretty words, and these, I fear, were few.
 At last, one frosty evening, with a cold chill down his spine,
 He found himself before her house, the threshold of the shrine.
 His courage flickered to a spark, then glowed with sudden flame--
 He knocked; he heard a welcome word; she came--his goddess came.
 Oh, she was fair as any flower, and huskily he spoke:
 "I'm all the way from Klondike, with a mighty heavy poke.
 I'm looking for a lassie, one whose Christian name is Peg,
 Who sought a Klondike miner, and who wrote it on an egg."

 The lassie gazed at him a space, her cheeks grew rosy red;
 She gazed at him with tear-bright eyes, then tenderly she said:
 "Yes, lonely Klondike miner, it is true my name is Peg.
 It's also true I longed for you and wrote it on an egg.
 My heart went out to someone in that land of night and cold;
 But oh, I fear that Yukon egg must have been mighty old.
 I waited long, I hoped and feared; you should have come before;
 I've been a wedded woman now for eighteen months or more.
 I'm sorry, since you've come so far, you ain't the one that wins;
 But won't you take a step inside--I'LL LET YOU SEE THE TWINS."
\end{poemblock}
