\poemchapter{L'Envoi}

\begin{poemblock}
  We talked of yesteryears, of trails and treasure,
   Of men who played the game and lost or won;
  Of mad stampedes, of toil beyond all measure,
   Of camp-fire comfort when the day was done.
  We talked of sullen nights by moon-dogs haunted,
   Of bird and beast and tree, of rod and gun;
  Of boat and tent, of hunting-trip enchanted
   Beneath the wonder of the midnight sun;
  Of bloody-footed dogs that gnawed the traces,
   Of prisoned seas, wind-lashed and winter-locked;
  The ice-gray dawn was pale upon our faces,
   Yet still we filled the cup and still we talked.

  The city street was dimmed.  We saw the glitter
   Of moon-picked brilliants on the virgin snow,
  And down the drifted canyon heard the bitter,
   Relentless slogan of the winds of woe.
  The city was forgot, and, parka-skirted,
   We trod that leagueless land that once we knew;
  We saw stream past, down valleys glacier-girted,
   The wolf-worn legions of the caribou.
  We smoked our pipes, o'er scenes of triumph dwelling;
   Of deeds of daring, dire defeats, we talked;
  And other tales that lost not in the telling,
   Ere to our beds uncertainly we walked.

  And so, dear friends, in gentler valleys roaming,
   Perhaps, when on my printed page you look,
  Your fancies by the firelight may go homing
   To that lone land that haply you forsook.
  And if perchance you hear the silence calling,
   The frozen music of star-yearning heights,
  Or, dreaming, see the seines of silver trawling
   Across the sky's abyss on vasty nights,
  You may recall that sweep of savage splendor,
   That land that measures each man at his worth,
  And feel in memory, half fierce, half tender,
   The brotherhood of men that know the North.

\end{poemblock}
