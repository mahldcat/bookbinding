\poemchapter{The Trail of Ninety-Eight}

\begin{poemblock}
 I.

 Gold!  We leapt from our benches.  Gold!  We sprang from our stools.
 Gold!  We wheeled in the furrow, fired with the faith of fools.
 Fearless, unfound, unfitted, far from the night and the cold,
 Heard we the clarion summons, followed the master-lure--Gold!

 Men from the sands of the Sunland; men from the woods of the West;
 Men from the farms and the cities, into the Northland we pressed.
 Graybeards and striplings and women, good men and bad men and bold,
 Leaving our homes and our loved ones, crying exultantly--"Gold!"

 Never was seen such an army, pitiful, futile, unfit;
 Never was seen such a spirit, manifold courage and grit.
 Never has been such a cohort under one banner unrolled
 As surged to the ragged-edged Arctic, urged by the arch-tempter--Gold.

 "Farewell!" we cried to our dearests; little we cared for their tears.
 "Farewell!" we cried to the humdrum and the yoke of the hireling years;
 Just like a pack of school-boys, and the big crowd cheered us good-bye.
 Never were hearts so uplifted, never were hopes so high.

 The spectral shores flitted past us, and every whirl of the screw
 Hurled us nearer to fortune, and ever we planned what we'd do--
 Do with the gold when we got it--big, shiny nuggets like plums,
 There in the sand of the river, gouging it out with our thumbs.

 And one man wanted a castle, another a racing stud;
 A third would cruise in a palace yacht like a red-necked prince of blood.
 And so we dreamed and we vaunted, millionaires to a man,
 Leaping to wealth in our visions long ere the trail began.


 II.

 We landed in wind-swept Skagway.  We joined the weltering mass,
 Clamoring over their outfits, waiting to climb the Pass.
 We tightened our girths and our pack-straps; we linked on the Human Chain,
 Struggling up to the summit, where every step was a pain.

 Gone was the joy of our faces, grim and haggard and pale;
 The heedless mirth of the shipboard was changed to the care of the trail.
 We flung ourselves in the struggle, packing our grub in relays,
 Step by step to the summit in the bale of the winter days.

 Floundering deep in the sump-holes, stumbling out again;
 Crying with cold and weakness, crazy with fear and pain.
 Then from the depths of our travail, ere our spirits were broke,
 Grim, tenacious and savage, the lust of the trail awoke.

 "Klondike or bust!" rang the slogan; every man for his own.
 Oh, how we flogged the horses, staggering skin and bone!
 Oh, how we cursed their weakness, anguish they could not tell,
 Breaking their hearts in our passion, lashing them on till they fell!

 For grub meant gold to our thinking, and all that could walk must pack;
 The sheep for the shambles stumbled, each with a load on its back;
 And even the swine were burdened, and grunted and squealed and rolled,
 And men went mad in the moment, huskily clamoring "Gold!"

 Oh, we were brutes and devils, goaded by lust and fear!
 Our eyes were strained to the summit; the weaklings dropped to the rear,
 Falling in heaps by the trail-side, heart-broken, limp and wan;
 But the gaps closed up in an instant, and heedless the chain went on.

 Never will I forget it, there on the mountain face,
 Antlike, men with their burdens, clinging in icy space;
 Dogged, determined and dauntless, cruel and callous and cold,
 Cursing, blaspheming, reviling, and ever that battle-cry--"Gold!"

 Thus toiled we, the army of fortune, in hunger and hope and despair,
 Till glacier, mountain and forest vanished, and, radiantly fair,
 There at our feet lay Lake Bennett, and down to its welcome we ran:
 The trail of the land was over, the trail of the water began.


 III.

 We built our boats and we launched them.  Never has been such a fleet;
 A packing-case for a bottom, a mackinaw for a sheet.
 Shapeless, grotesque, lopsided, flimsy, makeshift and crude,
 Each man after his fashion builded as best he could.

 Each man worked like a demon, as prow to rudder we raced;
 The winds of the Wild cried "Hurry!" the voice of the waters, "Haste!"
 We hated those driving before us; we dreaded those pressing behind;
 We cursed the slow current that bore us; we prayed to the God of the wind.

 Spring! and the hillsides flourished, vivid in jewelled green;
 Spring! and our hearts' blood nourished envy and hatred and spleen.
 Little cared we for the Spring-birth; much cared we to get on--
 Stake in the Great White Channel, stake ere the best be gone.

 The greed of the gold possessed us; pity and love were forgot;
 Covetous visions obsessed us; brother with brother fought.
 Partner with partner wrangled, each one claiming his due;
 Wrangled and halved their outfits, sawing their boats in two.

 Thuswise we voyaged Lake Bennett, Tagish, then Windy Arm,
 Sinister, savage and baleful, boding us hate and harm.
 Many a scow was shattered there on that iron shore;
 Many a heart was broken straining at sweep and oar.

 We roused Lake Marsh with a chorus, we drifted many a mile;
 There was the canyon before us--cave-like its dark defile;
 The shores swept faster and faster; the river narrowed to wrath;
 Waters that hissed disaster reared upright in our path.

 Beneath us the green tumult churning, above us the cavernous gloom;
 Around us, swift twisting and turning, the black, sullen walls of a tomb.
 We spun like a chip in a mill-race; our hearts hammered under the test;
 Then--oh, the relief on each chill face!--we soared into sunlight and rest.

 Hand sought for hand on the instant.  Cried we, "Our troubles are o'er!"
 Then, like a rumble of thunder, heard we a canorous roar.
 Leaping and boiling and seething, saw we a cauldron afume;
 There was the rage of the rapids, there was the menace of doom.

 The river springs like a racer, sweeps through a gash in the rock;
 Buts at the boulder-ribbed bottom, staggers and rears at the shock;
 Leaps like a terrified monster, writhes in its fury and pain;
 Then with the crash of a demon springs to the onset again.

 Dared we that ravening terror; heard we its din in our ears;
 Called on the Gods of our fathers, juggled forlorn with our fears;
 Sank to our waists in its fury, tossed to the sky like a fleece;
 Then, when our dread was the greatest, crashed into safety and peace.

 But what of the others that followed, losing their boats by the score?
 Well could we see them and hear them, strung down that desolate shore.
 What of the poor souls that perished?  Little of them shall be said--
 On to the Golden Valley, pause not to bury the dead.

 Then there were days of drifting, breezes soft as a sigh;
 Night trailed her robe of jewels over the floor of the sky.
 The moonlit stream was a python, silver, sinuous, vast,
 That writhed on a shroud of velvet--well, it was done at last.

 There were the tents of Dawson, there the scar of the slide;
 Swiftly we poled o'er the shallows, swiftly leapt o'er the side.
 Fires fringed the mouth of Bonanza; sunset gilded the dome;
 The test of the trail was over--thank God, thank God, we were Home!
\end{poemblock}
