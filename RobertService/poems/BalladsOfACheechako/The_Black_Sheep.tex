\poemchapter{The Black Sheep}

\begin{poemblock}
"The aristocratic ne'er-do-well in Canada frequently finds his way\\
into the ranks of the Royal North-West Mounted Police." --Extract.

\indentedblock\itshape{
Hark to the ewe that bore him:\\
\idt "What has muddied the strain?\\
Never his brothers before him\\
\idt  Showed the hint of a stain."\\
Hark to the tups and wethers;\\
\idt  Hark to the old gray ram:\\
"We're all of us white, but he's black as night,\\
\idt  And he'll never be worth a damn."
}

I'm up on the bally wood-pile at the back of the barracks yard;\\
"A damned disgrace to the force, sir", with a comrade standing guard;\\
Making the bluff I'm busy, doing my six months hard.

"Six months hard and dismissed, sir."  Isn't that rather hell?\\
And all because of the liquor laws and the wiles of a native belle--\\
Some "hooch" I gave to a siwash brave who swore that he wouldn't tell.

At least they SAY that I did it.  It's so in the town report.\\
All that I can recall is a night of revel and sport,\\
When I woke with a "head" in the guard-room,\\
\idt and they dragged me sick into court.

And the O. C. said:  "You are guilty", and I said never a word;\\
For, hang it, you see I couldn't--I didn't know WHAT had occurred,\\
And, under the circumstances, denial would be absurd.

But the one that cooked my bacon was Grubbe, of the City Patrol.\\
He fagged for my room at Eton, and didn't I devil his soul!\\
And now he is getting even, landing me down in the hole.

Plugging away on the wood-pile; doing chores round the square.\\
There goes an officer's lady--gives me a haughty stare--\\
Me that's an earl's own nephew--that is the hardest to bear.

To think of the poor old mater awaiting her prodigal son.\\
Tho' I broke her heart with my folly, I was always the white-haired one.\\
(That fatted calf that they're cooking will surely be overdone.)

I'll go back and yarn to the Bishop; I'll dance with the village belle;\\
I'll hand round tea to the ladies, and everything will be well.\\
Where I have been won't matter; what I have seen I won't tell.

I'll soar to their ken like a comet.  They'll see me with never a stain;\\
But will they reform me?--far from it.  We pay for our pleasure with pain;\\
But the dog will return to his vomit, the hog to his wallow again.

I've chewed on the rind of creation, and bitter I've tasted the same;\\
Stacked up against hell and damnation, I've managed to stay in the game;\\
I've had my moments of sorrow; I've had my seasons of shame.

That's past; when one's nature's a cracked one,\\
\idt it's too jolly hard to mend.\\
So long as the road is level, so long as I've cash to spend.\\
I'm bound to go to the devil, and it's all the same in the end.

The bugle is sounding for stables; the men troop off through the gloom;\\
An orderly laying the tables sings in the bright mess-room.\\
(I'll wash in the prison bucket, and brush with the prison broom.)

I'll lie in my cell and listen; I'll wish that I couldn't hear\\
The laugh and the chaff of the fellows swigging the canteen beer;\\
The nasal tone of the gramophone playing "The Bandolier".

And it seems to me, though it's misty, that night of the flowing bowl,\\
That the man who potlatched the whiskey and landed me into the hole\\
\textit{Was Grubbe, that Unmerciful Bounder, Grubbe, of the City Patrol}.

\end{poemblock}
