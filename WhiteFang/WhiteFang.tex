\documentclass[10pt]{book}
\usepackage[
  paperwidth=5.5in,
  paperheight=8.5in,
  inner=0.5in, % Margin on the binding side
  outer=0.75in,% Margin on the outside edge
  top=.75in,
  bottom=.75in,
  twoside  % Enables mirroring of inner/outer margins
]{geometry}

\usepackage[T1]{fontenc}
\usepackage[utf8]{inputenc}
\usepackage{microtype}
\usepackage{fancyhdr}
\usepackage{fontspec}
\usepackage{titlesec}

\setmainfont{Times New Roman}

\usepackage{csquotes}
\title{White Fang}
\author{Jack London}
\date{}

% Header/Footer style
\pagestyle{fancy}
\fancyhf{}
\fancyhead[LE,RO]{\thepage}
\fancyhead[RE]{\leftmark}
\fancyhead[LO]{\rightmark}

% Prevent chapter headings from adding "Chapter" prefix
\titleformat{\chapter}[display]
  {\normalfont\LARGE\bfseries\centering}{}{0pt}{}

\begin{document}

\frontmatter
\tableofcontents
\newpage

\mainmatter

%\part{}
\chapter{The Trail of the Meat}

Dark spruce forest frowned on either side the frozen waterway. The
trees had been stripped by a recent wind of their white covering of
frost, and they seemed to lean towards each other, black and ominous,
in the fading light. A vast silence reigned over the land. The land
itself was a desolation, lifeless, without movement, so lone and cold
that the spirit of it was not even that of sadness. There was a hint in
it of laughter, but of a laughter more terrible than any sadness—a
laughter that was mirthless as the smile of the sphinx, a laughter cold
as the frost and partaking of the grimness of infallibility. It was the
masterful and incommunicable wisdom of eternity laughing at the
futility of life and the effort of life. It was the Wild, the savage,
frozen-hearted Northland Wild.

But there \emph{was} life, abroad in the land and defiant. Down the frozen
waterway toiled a string of wolfish dogs. Their bristly fur was rimed
with frost. Their breath froze in the air as it left their mouths,
spouting forth in spumes of vapour that settled upon the hair of their
bodies and formed into crystals of frost. Leather harness was on the
dogs, and leather traces attached them to a sled which dragged along
behind. The sled was without runners. It was made of stout birch-bark,
and its full surface rested on the snow. The front end of the sled was
turned up, like a scroll, in order to force down and under the bore of
soft snow that surged like a wave before it. On the sled, securely
lashed, was a long and narrow oblong box. There were other things on
the sled—blankets, an axe, and a coffee-pot and frying-pan; but
prominent, occupying most of the space, was the long and narrow oblong
box.

In advance of the dogs, on wide snowshoes, toiled a man. At the rear of
the sled toiled a second man. On the sled, in the box, lay a third man
whose toil was over,—a man whom the Wild had conquered and beaten down
until he would never move nor struggle again. It is not the way of the
Wild to like movement. Life is an offence to it, for life is movement;
and the Wild aims always to destroy movement. It freezes the water to
prevent it running to the sea; it drives the sap out of the trees till
they are frozen to their mighty hearts; and most ferociously and
terribly of all does the Wild harry and crush into submission man—man
who is the most restless of life, ever in revolt against the dictum
that all movement must in the end come to the cessation of movement.

But at front and rear, unawed and indomitable, toiled the two men who
were not yet dead. Their bodies were covered with fur and soft-tanned
leather. Eyelashes and cheeks and lips were so coated with the crystals
from their frozen breath that their faces were not discernible. This
gave them the seeming of ghostly masques, undertakers in a spectral
world at the funeral of some ghost. But under it all they were men,
penetrating the land of desolation and mockery and silence, puny
adventurers bent on colossal adventure, pitting themselves against the
might of a world as remote and alien and pulseless as the abysses of
space.

They travelled on without speech, saving their breath for the work of
their bodies. On every side was the silence, pressing upon them with a
tangible presence. It affected their minds as the many atmospheres of
deep water affect the body of the diver. It crushed them with the
weight of unending vastness and unalterable decree. It crushed them
into the remotest recesses of their own minds, pressing out of them,
like juices from the grape, all the false ardours and exaltations and
undue self-values of the human soul, until they perceived themselves
finite and small, specks and motes, moving with weak cunning and little
wisdom amidst the play and inter-play of the great blind elements and
forces.

An hour went by, and a second hour. The pale light of the short sunless
day was beginning to fade, when a faint far cry arose on the still air.
It soared upward with a swift rush, till it reached its topmost note,
where it persisted, palpitant and tense, and then slowly died away. It
might have been a lost soul wailing, had it not been invested with a
certain sad fierceness and hungry eagerness. The front man turned his
head until his eyes met the eyes of the man behind. And then, across
the narrow oblong box, each nodded to the other.

A second cry arose, piercing the silence with needle-like shrillness.
Both men located the sound. It was to the rear, somewhere in the snow
expanse they had just traversed. A third and answering cry arose, also
to the rear and to the left of the second cry.

“They’re after us, Bill,” said the man at the front.

His voice sounded hoarse and unreal, and he had spoken with apparent
effort.

“Meat is scarce,” answered his comrade. “I ain’t seen a rabbit sign for
days.”

Thereafter they spoke no more, though their ears were keen for the
hunting-cries that continued to rise behind them.

At the fall of darkness they swung the dogs into a cluster of spruce
trees on the edge of the waterway and made a camp. The coffin, at the
side of the fire, served for seat and table. The wolf-dogs, clustered
on the far side of the fire, snarled and bickered among themselves, but
evinced no inclination to stray off into the darkness.

“Seems to me, Henry, they’re stayin’ remarkable close to camp,” Bill
commented.

Henry, squatting over the fire and settling the pot of coffee with a
piece of ice, nodded. Nor did he speak till he had taken his seat on
the coffin and begun to eat.

“They know where their hides is safe,” he said. “They’d sooner eat grub
than be grub. They’re pretty wise, them dogs.”

Bill shook his head. “Oh, I don’t know.”

His comrade looked at him curiously. “First time I ever heard you say
anything about their not bein’ wise.”

“Henry,” said the other, munching with deliberation the beans he was
eating, “did you happen to notice the way them dogs kicked up when I
was a-feedin’ ’em?”

“They did cut up more’n usual,” Henry acknowledged.

“How many dogs ’ve we got, Henry?”

“Six.”

“Well, Henry . . . ” Bill stopped for a moment, in order that his words
might gain greater significance. “As I was sayin’, Henry, we’ve got six
dogs. I took six fish out of the bag. I gave one fish to each dog, an’,
Henry, I was one fish short.”

“You counted wrong.”

“We’ve got six dogs,” the other reiterated dispassionately. “I took out
six fish. One Ear didn’t get no fish. I came back to the bag afterward
an’ got ’m his fish.”

“We’ve only got six dogs,” Henry said.

“Henry,” Bill went on. “I won’t say they was all dogs, but there was
seven of ’m that got fish.”

Henry stopped eating to glance across the fire and count the dogs.

“There’s only six now,” he said.

“I saw the other one run off across the snow,” Bill announced with cool
positiveness. “I saw seven.”

Henry looked at him commiseratingly, and said, “I’ll be almighty glad
when this trip’s over.”

“What d’ye mean by that?” Bill demanded.

“I mean that this load of ourn is gettin’ on your nerves, an’ that
you’re beginnin’ to see things.”

“I thought of that,” Bill answered gravely. “An’ so, when I saw it run
off across the snow, I looked in the snow an’ saw its tracks. Then I
counted the dogs an’ there was still six of ’em. The tracks is there in
the snow now. D’ye want to look at ’em? I’ll show ’em to you.”

Henry did not reply, but munched on in silence, until, the meal
finished, he topped it with a final cup of coffee. He wiped his mouth
with the back of his hand and said:

“Then you’re thinkin’ as it was—”

A long wailing cry, fiercely sad, from somewhere in the darkness, had
interrupted him. He stopped to listen to it, then he finished his
sentence with a wave of his hand toward the sound of the cry, “—one of
them?”

Bill nodded. “I’d a blame sight sooner think that than anything else.
You noticed yourself the row the dogs made.”

Cry after cry, and answering cries, were turning the silence into a
bedlam. From every side the cries arose, and the dogs betrayed their
fear by huddling together and so close to the fire that their hair was
scorched by the heat. Bill threw on more wood, before lighting his
pipe.

“I’m thinking you’re down in the mouth some,” Henry said.

“Henry . . . ” He sucked meditatively at his pipe for some time before
he went on. “Henry, I was a-thinkin’ what a blame sight luckier he is
than you an’ me’ll ever be.”

He indicated the third person by a downward thrust of the thumb to the
box on which they sat.

“You an’ me, Henry, when we die, we’ll be lucky if we get enough stones
over our carcases to keep the dogs off of us.”

“But we ain’t got people an’ money an’ all the rest, like him,” Henry
rejoined. “Long-distance funerals is somethin’ you an’ me can’t exactly
afford.”

“What gets me, Henry, is what a chap like this, that’s a lord or
something in his own country, and that’s never had to bother about grub
nor blankets; why he comes a-buttin’ round the Godforsaken ends of the
earth—that’s what I can’t exactly see.”

“He might have lived to a ripe old age if he’d stayed at home,” Henry
agreed.

Bill opened his mouth to speak, but changed his mind. Instead, he
pointed towards the wall of darkness that pressed about them from every
side. There was no suggestion of form in the utter blackness; only
could be seen a pair of eyes gleaming like live coals. Henry indicated
with his head a second pair, and a third. A circle of the gleaming eyes
had drawn about their camp. Now and again a pair of eyes moved, or
disappeared to appear again a moment later.

The unrest of the dogs had been increasing, and they stampeded, in a
surge of sudden fear, to the near side of the fire, cringing and
crawling about the legs of the men. In the scramble one of the dogs had
been overturned on the edge of the fire, and it had yelped with pain
and fright as the smell of its singed coat possessed the air. The
commotion caused the circle of eyes to shift restlessly for a moment
and even to withdraw a bit, but it settled down again as the dogs
became quiet.

“Henry, it’s a blame misfortune to be out of ammunition.”

Bill had finished his pipe and was helping his companion to spread the
bed of fur and blanket upon the spruce boughs which he had laid over
the snow before supper. Henry grunted, and began unlacing his
moccasins.

“How many cartridges did you say you had left?” he asked.

“Three,” came the answer. “An’ I wisht ’twas three hundred. Then I’d
show ’em what for, damn ’em!”

He shook his fist angrily at the gleaming eyes, and began securely to
prop his moccasins before the fire.

“An’ I wisht this cold snap’d break,” he went on. “It’s ben fifty below
for two weeks now. An’ I wisht I’d never started on this trip, Henry. I
don’t like the looks of it. I don’t feel right, somehow. An’ while I’m
wishin’, I wisht the trip was over an’ done with, an’ you an’ me
a-sittin’ by the fire in Fort McGurry just about now an’ playing
cribbage—that’s what I wisht.”

Henry grunted and crawled into bed. As he dozed off he was aroused by
his comrade’s voice.

“Say, Henry, that other one that come in an’ got a fish—why didn’t the
dogs pitch into it? That’s what’s botherin’ me.”

“You’re botherin’ too much, Bill,” came the sleepy response. “You was
never like this before. You jes’ shut up now, an’ go to sleep, an’
you’ll be all hunkydory in the mornin’. Your stomach’s sour, that’s
what’s botherin’ you.”

The men slept, breathing heavily, side by side, under the one covering.
The fire died down, and the gleaming eyes drew closer the circle they
had flung about the camp. The dogs clustered together in fear, now and
again snarling menacingly as a pair of eyes drew close. Once their
uproar became so loud that Bill woke up. He got out of bed carefully,
so as not to disturb the sleep of his comrade, and threw more wood on
the fire. As it began to flame up, the circle of eyes drew farther
back. He glanced casually at the huddling dogs. He rubbed his eyes and
looked at them more sharply. Then he crawled back into the blankets.

“Henry,” he said. “Oh, Henry.”

Henry groaned as he passed from sleep to waking, and demanded, “What’s
wrong now?”

“Nothin’,” came the answer; “only there’s seven of ’em again. I just
counted.”

Henry acknowledged receipt of the information with a grunt that slid
into a snore as he drifted back into sleep.

In the morning it was Henry who awoke first and routed his companion
out of bed. Daylight was yet three hours away, though it was already
six o’clock; and in the darkness Henry went about preparing breakfast,
while Bill rolled the blankets and made the sled ready for lashing.

“Say, Henry,” he asked suddenly, “how many dogs did you say we had?”

“Six.”

“Wrong,” Bill proclaimed triumphantly.

“Seven again?” Henry queried.

“No, five; one’s gone.”

“The hell!” Henry cried in wrath, leaving the cooking to come and count
the dogs.

“You’re right, Bill,” he concluded. “Fatty’s gone.”

“An’ he went like greased lightnin’ once he got started. Couldn’t ’ve
seen ’m for smoke.”

“No chance at all,” Henry concluded. “They jes’ swallowed ’m alive. I
bet he was yelpin’ as he went down their throats, damn ’em!”

“He always was a fool dog,” said Bill.

“But no fool dog ought to be fool enough to go off an’ commit suicide
that way.” He looked over the remainder of the team with a speculative
eye that summed up instantly the salient traits of each animal. “I bet
none of the others would do it.”

“Couldn’t drive ’em away from the fire with a club,” Bill agreed. “I
always did think there was somethin’ wrong with Fatty anyway.”

And this was the epitaph of a dead dog on the Northland trail—less
scant than the epitaph of many another dog, of many a man.

\chapter{The She-Wolf}

Breakfast eaten and the slim camp-outfit lashed to the sled, the men
turned their backs on the cheery fire and launched out into the
darkness. At once began to rise the cries that were fiercely sad—cries
that called through the darkness and cold to one another and answered
back. Conversation ceased. Daylight came at nine o’clock. At midday the
sky to the south warmed to rose-colour, and marked where the bulge of
the earth intervened between the meridian sun and the northern world.
But the rose-colour swiftly faded. The grey light of day that remained
lasted until three o’clock, when it, too, faded, and the pall of the
Arctic night descended upon the lone and silent land.

As darkness came on, the hunting-cries to right and left and rear drew
closer—so close that more than once they sent surges of fear through
the toiling dogs, throwing them into short-lived panics.

At the conclusion of one such panic, when he and Henry had got the dogs
back in the traces, Bill said:

“I wisht they’d strike game somewheres, an’ go away an’ leave us
alone.”

“They do get on the nerves horrible,” Henry sympathised.

They spoke no more until camp was made.

Henry was bending over and adding ice to the babbling pot of beans when
he was startled by the sound of a blow, an exclamation from Bill, and a
sharp snarling cry of pain from among the dogs. He straightened up in
time to see a dim form disappearing across the snow into the shelter of
the dark. Then he saw Bill, standing amid the dogs, half triumphant,
half crestfallen, in one hand a stout club, in the other the tail and
part of the body of a sun-cured salmon.

“It got half of it,” he announced; “but I got a whack at it jes’ the
same. D’ye hear it squeal?”

“What’d it look like?” Henry asked.

“Couldn’t see. But it had four legs an’ a mouth an’ hair an’ looked
like any dog.”

“Must be a tame wolf, I reckon.”

“It’s damned tame, whatever it is, comin’ in here at feedin’ time an’
gettin’ its whack of fish.”

That night, when supper was finished and they sat on the oblong box and
pulled at their pipes, the circle of gleaming eyes drew in even closer
than before.

“I wisht they’d spring up a bunch of moose or something, an’ go away
an’ leave us alone,” Bill said.

Henry grunted with an intonation that was not all sympathy, and for a
quarter of an hour they sat on in silence, Henry staring at the fire,
and Bill at the circle of eyes that burned in the darkness just beyond
the firelight.

“I wisht we was pullin’ into McGurry right now,” he began again.

“Shut up your wishin’ and your croakin’,” Henry burst out angrily.
“Your stomach’s sour. That’s what’s ailin’ you. Swallow a spoonful of
sody, an’ you’ll sweeten up wonderful an’ be more pleasant company.”

In the morning Henry was aroused by fervid blasphemy that proceeded
from the mouth of Bill. Henry propped himself up on an elbow and looked
to see his comrade standing among the dogs beside the replenished fire,
his arms raised in objurgation, his face distorted with passion.

“Hello!” Henry called. “What’s up now?”

“Frog’s gone,” came the answer.

“No.”

“I tell you yes.”

Henry leaped out of the blankets and to the dogs. He counted them with
care, and then joined his partner in cursing the power of the Wild that
had robbed them of another dog.

“Frog was the strongest dog of the bunch,” Bill pronounced finally.

“An’ he was no fool dog neither,” Henry added.

And so was recorded the second epitaph in two days.

A gloomy breakfast was eaten, and the four remaining dogs were
harnessed to the sled. The day was a repetition of the days that had
gone before. The men toiled without speech across the face of the
frozen world. The silence was unbroken save by the cries of their
pursuers, that, unseen, hung upon their rear. With the coming of night
in the mid-afternoon, the cries sounded closer as the pursuers drew in
according to their custom; and the dogs grew excited and frightened,
and were guilty of panics that tangled the traces and further depressed
the two men.

“There, that’ll fix you fool critters,” Bill said with satisfaction
that night, standing erect at completion of his task.

Henry left the cooking to come and see. Not only had his partner tied
the dogs up, but he had tied them, after the Indian fashion, with
sticks. About the neck of each dog he had fastened a leather thong. To
this, and so close to the neck that the dog could not get his teeth to
it, he had tied a stout stick four or five feet in length. The other
end of the stick, in turn, was made fast to a stake in the ground by
means of a leather thong. The dog was unable to gnaw through the
leather at his own end of the stick. The stick prevented him from
getting at the leather that fastened the other end.

Henry nodded his head approvingly.

“It’s the only contraption that’ll ever hold One Ear,” he said. “He can
gnaw through leather as clean as a knife an’ jes’ about half as quick.
They all’ll be here in the mornin’ hunkydory.”

“You jes’ bet they will,” Bill affirmed. “If one of em’ turns up
missin’, I’ll go without my coffee.”

“They jes’ know we ain’t loaded to kill,” Henry remarked at bed-time,
indicating the gleaming circle that hemmed them in. “If we could put a
couple of shots into ’em, they’d be more respectful. They come closer
every night. Get the firelight out of your eyes an’ look hard—there!
Did you see that one?”

For some time the two men amused themselves with watching the movement
of vague forms on the edge of the firelight. By looking closely and
steadily at where a pair of eyes burned in the darkness, the form of
the animal would slowly take shape. They could even see these forms
move at times.

A sound among the dogs attracted the men’s attention. One Ear was
uttering quick, eager whines, lunging at the length of his stick toward
the darkness, and desisting now and again in order to make frantic
attacks on the stick with his teeth.

“Look at that, Bill,” Henry whispered.

Full into the firelight, with a stealthy, sidelong movement, glided a
doglike animal. It moved with commingled mistrust and daring,
cautiously observing the men, its attention fixed on the dogs. One Ear
strained the full length of the stick toward the intruder and whined
with eagerness.

“That fool One Ear don’t seem scairt much,” Bill said in a low tone.

“It’s a she-wolf,” Henry whispered back, “an’ that accounts for Fatty
an’ Frog. She’s the decoy for the pack. She draws out the dog an’ then
all the rest pitches in an’ eats ’m up.”

The fire crackled. A log fell apart with a loud spluttering noise. At
the sound of it the strange animal leaped back into the darkness.

“Henry, I’m a-thinkin’,” Bill announced.

“Thinkin’ what?”

“I’m a-thinkin’ that was the one I lambasted with the club.”

“Ain’t the slightest doubt in the world,” was Henry’s response.

“An’ right here I want to remark,” Bill went on, “that that animal’s
familyarity with campfires is suspicious an’ immoral.”

“It knows for certain more’n a self-respectin’ wolf ought to know,”
Henry agreed. “A wolf that knows enough to come in with the dogs at
feedin’ time has had experiences.”

“Ol’ Villan had a dog once that run away with the wolves,” Bill
cogitates aloud. “I ought to know. I shot it out of the pack in a moose
pasture over ‘on Little Stick. An’ Ol’ Villan cried like a baby. Hadn’t
seen it for three years, he said. Ben with the wolves all that time.”

“I reckon you’ve called the turn, Bill. That wolf’s a dog, an’ it’s
eaten fish many’s the time from the hand of man.”

“An if I get a chance at it, that wolf that’s a dog’ll be jes’ meat,”
Bill declared. “We can’t afford to lose no more animals.”

“But you’ve only got three cartridges,” Henry objected.

“I’ll wait for a dead sure shot,” was the reply.

In the morning Henry renewed the fire and cooked breakfast to the
accompaniment of his partner’s snoring.

“You was sleepin’ jes’ too comfortable for anything,” Henry told him,
as he routed him out for breakfast. “I hadn’t the heart to rouse you.”

Bill began to eat sleepily. He noticed that his cup was empty and
started to reach for the pot. But the pot was beyond arm’s length and
beside Henry.

“Say, Henry,” he chided gently, “ain’t you forgot somethin’?”

Henry looked about with great carefulness and shook his head. Bill held
up the empty cup.

“You don’t get no coffee,” Henry announced.

“Ain’t run out?” Bill asked anxiously.

“Nope.”

“Ain’t thinkin’ it’ll hurt my digestion?”

“Nope.”

A flush of angry blood pervaded Bill’s face.

“Then it’s jes’ warm an’ anxious I am to be hearin’ you explain
yourself,” he said.

“Spanker’s gone,” Henry answered.

Without haste, with the air of one resigned to misfortune Bill turned
his head, and from where he sat counted the dogs.

“How’d it happen?” he asked apathetically.

Henry shrugged his shoulders. “Don’t know. Unless One Ear gnawed ’m
loose. He couldn’t a-done it himself, that’s sure.”

“The darned cuss.” Bill spoke gravely and slowly, with no hint of the
anger that was raging within. “Jes’ because he couldn’t chew himself
loose, he chews Spanker loose.”

“Well, Spanker’s troubles is over anyway; I guess he’s digested by this
time an’ cavortin’ over the landscape in the bellies of twenty
different wolves,” was Henry’s epitaph on this, the latest lost dog.
“Have some coffee, Bill.”

But Bill shook his head.

“Go on,” Henry pleaded, elevating the pot.

Bill shoved his cup aside. “I’ll be ding-dong-danged if I do. I said I
wouldn’t if ary dog turned up missin’, an’ I won’t.”

“It’s darn good coffee,” Henry said enticingly.

But Bill was stubborn, and he ate a dry breakfast washed down with
mumbled curses at One Ear for the trick he had played.

“I’ll tie ’em up out of reach of each other to-night,” Bill said, as
they took the trail.

They had travelled little more than a hundred yards, when Henry, who
was in front, bent down and picked up something with which his snowshoe
had collided. It was dark, and he could not see it, but he recognised
it by the touch. He flung it back, so that it struck the sled and
bounced along until it fetched up on Bill’s snowshoes.

“Mebbe you’ll need that in your business,” Henry said.

Bill uttered an exclamation. It was all that was left of Spanker—the
stick with which he had been tied.

“They ate ’m hide an’ all,” Bill announced. “The stick’s as clean as a
whistle. They’ve ate the leather offen both ends. They’re damn hungry,
Henry, an’ they’ll have you an’ me guessin’ before this trip’s over.”

Henry laughed defiantly. “I ain’t been trailed this way by wolves
before, but I’ve gone through a whole lot worse an’ kept my health.
Takes more’n a handful of them pesky critters to do for yours truly,
Bill, my son.”

“I don’t know, I don’t know,” Bill muttered ominously.

“Well, you’ll know all right when we pull into McGurry.”

“I ain’t feelin’ special enthusiastic,” Bill persisted.

“You’re off colour, that’s what’s the matter with you,” Henry
dogmatised. “What you need is quinine, an’ I’m goin’ to dose you up
stiff as soon as we make McGurry.”

Bill grunted his disagreement with the diagnosis, and lapsed into
silence. The day was like all the days. Light came at nine o’clock. At
twelve o’clock the southern horizon was warmed by the unseen sun; and
then began the cold grey of afternoon that would merge, three hours
later, into night.

It was just after the sun’s futile effort to appear, that Bill slipped
the rifle from under the sled-lashings and said:

“You keep right on, Henry, I’m goin’ to see what I can see.”

“You’d better stick by the sled,” his partner protested. “You’ve only
got three cartridges, an’ there’s no tellin’ what might happen.”

“Who’s croaking now?” Bill demanded triumphantly.

Henry made no reply, and plodded on alone, though often he cast anxious
glances back into the grey solitude where his partner had disappeared.
An hour later, taking advantage of the cut-offs around which the sled
had to go, Bill arrived.

“They’re scattered an’ rangin’ along wide,” he said: “keeping up with
us an’ lookin’ for game at the same time. You see, they’re sure of us,
only they know they’ve got to wait to get us. In the meantime they’re
willin’ to pick up anything eatable that comes handy.”

“You mean they \emph{think} they’re sure of us,” Henry objected pointedly.

But Bill ignored him. “I seen some of them. They’re pretty thin. They
ain’t had a bite in weeks I reckon, outside of Fatty an’ Frog an’
Spanker; an’ there’s so many of ’em that that didn’t go far. They’re
remarkable thin. Their ribs is like wash-boards, an’ their stomachs is
right up against their backbones. They’re pretty desperate, I can tell
you. They’ll be goin’ mad, yet, an’ then watch out.”

A few minutes later, Henry, who was now travelling behind the sled,
emitted a low, warning whistle. Bill turned and looked, then quietly
stopped the dogs. To the rear, from around the last bend and plainly
into view, on the very trail they had just covered, trotted a furry,
slinking form. Its nose was to the trail, and it trotted with a
peculiar, sliding, effortless gait. When they halted, it halted,
throwing up its head and regarding them steadily with nostrils that
twitched as it caught and studied the scent of them.

“It’s the she-wolf,” Bill answered.

The dogs had lain down in the snow, and he walked past them to join his
partner in the sled. Together they watched the strange animal that had
pursued them for days and that had already accomplished the destruction
of half their dog-team.

After a searching scrutiny, the animal trotted forward a few steps.
This it repeated several times, till it was a short hundred yards away.
It paused, head up, close by a clump of spruce trees, and with sight
and scent studied the outfit of the watching men. It looked at them in
a strangely wistful way, after the manner of a dog; but in its
wistfulness there was none of the dog affection. It was a wistfulness
bred of hunger, as cruel as its own fangs, as merciless as the frost
itself.

It was large for a wolf, its gaunt frame advertising the lines of an
animal that was among the largest of its kind.

“Stands pretty close to two feet an’ a half at the shoulders,” Henry
commented. “An’ I’ll bet it ain’t far from five feet long.”

“Kind of strange colour for a wolf,” was Bill’s criticism. “I never
seen a red wolf before. Looks almost cinnamon to me.”

The animal was certainly not cinnamon-coloured. Its coat was the true
wolf-coat. The dominant colour was grey, and yet there was to it a
faint reddish hue—a hue that was baffling, that appeared and
disappeared, that was more like an illusion of the vision, now grey,
distinctly grey, and again giving hints and glints of a vague redness
of colour not classifiable in terms of ordinary experience.

“Looks for all the world like a big husky sled-dog,” Bill said. “I
wouldn’t be s’prised to see it wag its tail.”

“Hello, you husky!” he called. “Come here, you whatever-your-name-is.”

“Ain’t a bit scairt of you,” Henry laughed.

Bill waved his hand at it threateningly and shouted loudly; but the
animal betrayed no fear. The only change in it that they could notice
was an accession of alertness. It still regarded them with the
merciless wistfulness of hunger. They were meat, and it was hungry; and
it would like to go in and eat them if it dared.

“Look here, Henry,” Bill said, unconsciously lowering his voice to a
whisper because of what he imitated. “We’ve got three cartridges. But
it’s a dead shot. Couldn’t miss it. It’s got away with three of our
dogs, an’ we oughter put a stop to it. What d’ye say?”

Henry nodded his consent. Bill cautiously slipped the gun from under
the sled-lashing. The gun was on the way to his shoulder, but it never
got there. For in that instant the she-wolf leaped sidewise from the
trail into the clump of spruce trees and disappeared.

The two men looked at each other. Henry whistled long and
comprehendingly.

“I might have knowed it,” Bill chided himself aloud as he replaced the
gun. “Of course a wolf that knows enough to come in with the dogs at
feedin’ time, ’d know all about shooting-irons. I tell you right now,
Henry, that critter’s the cause of all our trouble. We’d have six dogs
at the present time, ’stead of three, if it wasn’t for her. An’ I tell
you right now, Henry, I’m goin’ to get her. She’s too smart to be shot
in the open. But I’m goin’ to lay for her. I’ll bushwhack her as sure
as my name is Bill.”

“You needn’t stray off too far in doin’ it,” his partner admonished.
“If that pack ever starts to jump you, them three cartridges’d be wuth
no more’n three whoops in hell. Them animals is damn hungry, an’ once
they start in, they’ll sure get you, Bill.”

They camped early that night. Three dogs could not drag the sled so
fast nor for so long hours as could six, and they were showing
unmistakable signs of playing out. And the men went early to bed, Bill
first seeing to it that the dogs were tied out of gnawing-reach of one
another.

But the wolves were growing bolder, and the men were aroused more than
once from their sleep. So near did the wolves approach, that the dogs
became frantic with terror, and it was necessary to replenish the fire
from time to time in order to keep the adventurous marauders at safer
distance.

“I’ve hearn sailors talk of sharks followin’ a ship,” Bill remarked, as
he crawled back into the blankets after one such replenishing of the
fire. “Well, them wolves is land sharks. They know their business
better’n we do, an’ they ain’t a-holdin’ our trail this way for their
health. They’re goin’ to get us. They’re sure goin’ to get us, Henry.”

“They’ve half got you a’ready, a-talkin’ like that,” Henry retorted
sharply. “A man’s half licked when he says he is. An’ you’re half eaten
from the way you’re goin’ on about it.”

“They’ve got away with better men than you an’ me,” Bill answered.

“Oh, shet up your croakin’. You make me all-fired tired.”

Henry rolled over angrily on his side, but was surprised that Bill made
no similar display of temper. This was not Bill’s way, for he was
easily angered by sharp words. Henry thought long over it before he
went to sleep, and as his eyelids fluttered down and he dozed off, the
thought in his mind was: “There’s no mistakin’ it, Bill’s almighty
blue. I’ll have to cheer him up to-morrow.”

\chapter{The Hunger Cry}

The day began auspiciously. They had lost no dogs during the night, and
they swung out upon the trail and into the silence, the darkness, and
the cold with spirits that were fairly light. Bill seemed to have
forgotten his forebodings of the previous night, and even waxed
facetious with the dogs when, at midday, they overturned the sled on a
bad piece of trail.

It was an awkward mix-up. The sled was upside down and jammed between a
tree-trunk and a huge rock, and they were forced to unharness the dogs
in order to straighten out the tangle. The two men were bent over the
sled and trying to right it, when Henry observed One Ear sidling away.

“Here, you, One Ear!” he cried, straightening up and turning around on
the dog.

But One Ear broke into a run across the snow, his traces trailing
behind him. And there, out in the snow of their back track, was the
she-wolf waiting for him. As he neared her, he became suddenly
cautious. He slowed down to an alert and mincing walk and then stopped.
He regarded her carefully and dubiously, yet desirefully. She seemed to
smile at him, showing her teeth in an ingratiating rather than a
menacing way. She moved toward him a few steps, playfully, and then
halted. One Ear drew near to her, still alert and cautious, his tail
and ears in the air, his head held high.

He tried to sniff noses with her, but she retreated playfully and
coyly. Every advance on his part was accompanied by a corresponding
retreat on her part. Step by step she was luring him away from the
security of his human companionship. Once, as though a warning had in
vague ways flitted through his intelligence, he turned his head and
looked back at the overturned sled, at his team-mates, and at the two
men who were calling to him.

But whatever idea was forming in his mind, was dissipated by the
she-wolf, who advanced upon him, sniffed noses with him for a fleeting
instant, and then resumed her coy retreat before his renewed advances.

In the meantime, Bill had bethought himself of the rifle. But it was
jammed beneath the overturned sled, and by the time Henry had helped
him to right the load, One Ear and the she-wolf were too close together
and the distance too great to risk a shot.

Too late One Ear learned his mistake. Before they saw the cause, the
two men saw him turn and start to run back toward them. Then,
approaching at right angles to the trail and cutting off his retreat
they saw a dozen wolves, lean and grey, bounding across the snow. On
the instant, the she-wolf’s coyness and playfulness disappeared. With a
snarl she sprang upon One Ear. He thrust her off with his shoulder,
and, his retreat cut off and still intent on regaining the sled, he
altered his course in an attempt to circle around to it. More wolves
were appearing every moment and joining in the chase. The she-wolf was
one leap behind One Ear and holding her own.

“Where are you goin’?” Henry suddenly demanded, laying his hand on his
partner’s arm.

Bill shook it off. “I won’t stand it,” he said. “They ain’t a-goin’ to
get any more of our dogs if I can help it.”

Gun in hand, he plunged into the underbrush that lined the side of the
trail. His intention was apparent enough. Taking the sled as the centre
of the circle that One Ear was making, Bill planned to tap that circle
at a point in advance of the pursuit. With his rifle, in the broad
daylight, it might be possible for him to awe the wolves and save the
dog.

“Say, Bill!” Henry called after him. “Be careful! Don’t take no
chances!”

Henry sat down on the sled and watched. There was nothing else for him
to do. Bill had already gone from sight; but now and again, appearing
and disappearing amongst the underbrush and the scattered clumps of
spruce, could be seen One Ear. Henry judged his case to be hopeless.
The dog was thoroughly alive to its danger, but it was running on the
outer circle while the wolf-pack was running on the inner and shorter
circle. It was vain to think of One Ear so outdistancing his pursuers
as to be able to cut across their circle in advance of them and to
regain the sled.

The different lines were rapidly approaching a point. Somewhere out
there in the snow, screened from his sight by trees and thickets, Henry
knew that the wolf-pack, One Ear, and Bill were coming together. All
too quickly, far more quickly than he had expected, it happened. He
heard a shot, then two shots, in rapid succession, and he knew that
Bill’s ammunition was gone. Then he heard a great outcry of snarls and
yelps. He recognised One Ear’s yell of pain and terror, and he heard a
wolf-cry that bespoke a stricken animal. And that was all. The snarls
ceased. The yelping died away. Silence settled down again over the
lonely land.

He sat for a long while upon the sled. There was no need for him to go
and see what had happened. He knew it as though it had taken place
before his eyes. Once, he roused with a start and hastily got the axe
out from underneath the lashings. But for some time longer he sat and
brooded, the two remaining dogs crouching and trembling at his feet.

At last he arose in a weary manner, as though all the resilience had
gone out of his body, and proceeded to fasten the dogs to the sled. He
passed a rope over his shoulder, a man-trace, and pulled with the dogs.
He did not go far. At the first hint of darkness he hastened to make a
camp, and he saw to it that he had a generous supply of firewood. He
fed the dogs, cooked and ate his supper, and made his bed close to the
fire.

But he was not destined to enjoy that bed. Before his eyes closed the
wolves had drawn too near for safety. It no longer required an effort
of the vision to see them. They were all about him and the fire, in a
narrow circle, and he could see them plainly in the firelight lying
down, sitting up, crawling forward on their bellies, or slinking back
and forth. They even slept. Here and there he could see one curled up
in the snow like a dog, taking the sleep that was now denied himself.

He kept the fire brightly blazing, for he knew that it alone intervened
between the flesh of his body and their hungry fangs. His two dogs
stayed close by him, one on either side, leaning against him for
protection, crying and whimpering, and at times snarling desperately
when a wolf approached a little closer than usual. At such moments,
when his dogs snarled, the whole circle would be agitated, the wolves
coming to their feet and pressing tentatively forward, a chorus of
snarls and eager yelps rising about him. Then the circle would lie down
again, and here and there a wolf would resume its broken nap.

But this circle had a continuous tendency to draw in upon him. Bit by
bit, an inch at a time, with here a wolf bellying forward, and there a
wolf bellying forward, the circle would narrow until the brutes were
almost within springing distance. Then he would seize brands from the
fire and hurl them into the pack. A hasty drawing back always resulted,
accompanied by angry yelps and frightened snarls when a well-aimed
brand struck and scorched a too daring animal.

Morning found the man haggard and worn, wide-eyed from want of sleep.
He cooked breakfast in the darkness, and at nine o’clock, when, with
the coming of daylight, the wolf-pack drew back, he set about the task
he had planned through the long hours of the night. Chopping down young
saplings, he made them cross-bars of a scaffold by lashing them high up
to the trunks of standing trees. Using the sled-lashing for a heaving
rope, and with the aid of the dogs, he hoisted the coffin to the top of
the scaffold.

“They got Bill, an’ they may get me, but they’ll sure never get you,
young man,” he said, addressing the dead body in its tree-sepulchre.

Then he took the trail, the lightened sled bounding along behind the
willing dogs; for they, too, knew that safety lay open in the gaining
of Fort McGurry. The wolves were now more open in their pursuit,
trotting sedately behind and ranging along on either side, their red
tongues lolling out, their lean sides showing the undulating ribs with
every movement. They were very lean, mere skin-bags stretched over bony
frames, with strings for muscles—so lean that Henry found it in his
mind to marvel that they still kept their feet and did not collapse
forthright in the snow.

He did not dare travel until dark. At midday, not only did the sun warm
the southern horizon, but it even thrust its upper rim, pale and
golden, above the sky-line. He received it as a sign. The days were
growing longer. The sun was returning. But scarcely had the cheer of
its light departed, than he went into camp. There were still several
hours of grey daylight and sombre twilight, and he utilised them in
chopping an enormous supply of fire-wood.

With night came horror. Not only were the starving wolves growing
bolder, but lack of sleep was telling upon Henry. He dozed despite
himself, crouching by the fire, the blankets about his shoulders, the
axe between his knees, and on either side a dog pressing close against
him. He awoke once and saw in front of him, not a dozen feet away, a
big grey wolf, one of the largest of the pack. And even as he looked,
the brute deliberately stretched himself after the manner of a lazy
dog, yawning full in his face and looking upon him with a possessive
eye, as if, in truth, he were merely a delayed meal that was soon to be
eaten.

This certitude was shown by the whole pack. Fully a score he could
count, staring hungrily at him or calmly sleeping in the snow. They
reminded him of children gathered about a spread table and awaiting
permission to begin to eat. And he was the food they were to eat! He
wondered how and when the meal would begin.

As he piled wood on the fire he discovered an appreciation of his own
body which he had never felt before. He watched his moving muscles and
was interested in the cunning mechanism of his fingers. By the light of
the fire he crooked his fingers slowly and repeatedly now one at a
time, now all together, spreading them wide or making quick gripping
movements. He studied the nail-formation, and prodded the finger-tips,
now sharply, and again softly, gauging the while the nerve-sensations
produced. It fascinated him, and he grew suddenly fond of this subtle
flesh of his that worked so beautifully and smoothly and delicately.
Then he would cast a glance of fear at the wolf-circle drawn
expectantly about him, and like a blow the realisation would strike him
that this wonderful body of his, this living flesh, was no more than so
much meat, a quest of ravenous animals, to be torn and slashed by their
hungry fangs, to be sustenance to them as the moose and the rabbit had
often been sustenance to him.

He came out of a doze that was half nightmare, to see the red-hued
she-wolf before him. She was not more than half a dozen feet away
sitting in the snow and wistfully regarding him. The two dogs were
whimpering and snarling at his feet, but she took no notice of them.
She was looking at the man, and for some time he returned her look.
There was nothing threatening about her. She looked at him merely with
a great wistfulness, but he knew it to be the wistfulness of an equally
great hunger. He was the food, and the sight of him excited in her the
gustatory sensations. Her mouth opened, the saliva drooled forth, and
she licked her chops with the pleasure of anticipation.

A spasm of fear went through him. He reached hastily for a brand to
throw at her. But even as he reached, and before his fingers had closed
on the missile, she sprang back into safety; and he knew that she was
used to having things thrown at her. She had snarled as she sprang
away, baring her white fangs to their roots, all her wistfulness
vanishing, being replaced by a carnivorous malignity that made him
shudder. He glanced at the hand that held the brand, noticing the
cunning delicacy of the fingers that gripped it, how they adjusted
themselves to all the inequalities of the surface, curling over and
under and about the rough wood, and one little finger, too close to the
burning portion of the brand, sensitively and automatically writhing
back from the hurtful heat to a cooler gripping-place; and in the same
instant he seemed to see a vision of those same sensitive and delicate
fingers being crushed and torn by the white teeth of the she-wolf.
Never had he been so fond of this body of his as now when his tenure of
it was so precarious.

All night, with burning brands, he fought off the hungry pack. When he
dozed despite himself, the whimpering and snarling of the dogs aroused
him. Morning came, but for the first time the light of day failed to
scatter the wolves. The man waited in vain for them to go. They
remained in a circle about him and his fire, displaying an arrogance of
possession that shook his courage born of the morning light.

He made one desperate attempt to pull out on the trail. But the moment
he left the protection of the fire, the boldest wolf leaped for him,
but leaped short. He saved himself by springing back, the jaws snapping
together a scant six inches from his thigh. The rest of the pack was
now up and surging upon him, and a throwing of firebrands right and
left was necessary to drive them back to a respectful distance.

Even in the daylight he did not dare leave the fire to chop fresh wood.
Twenty feet away towered a huge dead spruce. He spent half the day
extending his campfire to the tree, at any moment a half dozen burning
faggots ready at hand to fling at his enemies. Once at the tree, he
studied the surrounding forest in order to fell the tree in the
direction of the most firewood.

The night was a repetition of the night before, save that the need for
sleep was becoming overpowering. The snarling of his dogs was losing
its efficacy. Besides, they were snarling all the time, and his
benumbed and drowsy senses no longer took note of changing pitch and
intensity. He awoke with a start. The she-wolf was less than a yard
from him. Mechanically, at short range, without letting go of it, he
thrust a brand full into her open and snarling mouth. She sprang away,
yelling with pain, and while he took delight in the smell of burning
flesh and hair, he watched her shaking her head and growling wrathfully
a score of feet away.

But this time, before he dozed again, he tied a burning pine-knot to
his right hand. His eyes were closed but few minutes when the burn of
the flame on his flesh awakened him. For several hours he adhered to
this programme. Every time he was thus awakened he drove back the
wolves with flying brands, replenished the fire, and rearranged the
pine-knot on his hand. All worked well, but there came a time when he
fastened the pine-knot insecurely. As his eyes closed it fell away from
his hand.

He dreamed. It seemed to him that he was in Fort McGurry. It was warm
and comfortable, and he was playing cribbage with the Factor. Also, it
seemed to him that the fort was besieged by wolves. They were howling
at the very gates, and sometimes he and the Factor paused from the game
to listen and laugh at the futile efforts of the wolves to get in. And
then, so strange was the dream, there was a crash. The door was burst
open. He could see the wolves flooding into the big living-room of the
fort. They were leaping straight for him and the Factor. With the
bursting open of the door, the noise of their howling had increased
tremendously. This howling now bothered him. His dream was merging into
something else—he knew not what; but through it all, following him,
persisted the howling.

And then he awoke to find the howling real. There was a great snarling
and yelping. The wolves were rushing him. They were all about him and
upon him. The teeth of one had closed upon his arm. Instinctively he
leaped into the fire, and as he leaped, he felt the sharp slash of
teeth that tore through the flesh of his leg. Then began a fire fight.
His stout mittens temporarily protected his hands, and he scooped live
coals into the air in all directions, until the campfire took on the
semblance of a volcano.

But it could not last long. His face was blistering in the heat, his
eyebrows and lashes were singed off, and the heat was becoming
unbearable to his feet. With a flaming brand in each hand, he sprang to
the edge of the fire. The wolves had been driven back. On every side,
wherever the live coals had fallen, the snow was sizzling, and every
little while a retiring wolf, with wild leap and snort and snarl,
announced that one such live coal had been stepped upon.

Flinging his brands at the nearest of his enemies, the man thrust his
smouldering mittens into the snow and stamped about to cool his feet.
His two dogs were missing, and he well knew that they had served as a
course in the protracted meal which had begun days before with Fatty,
the last course of which would likely be himself in the days to follow.

“You ain’t got me yet!” he cried, savagely shaking his fist at the
hungry beasts; and at the sound of his voice the whole circle was
agitated, there was a general snarl, and the she-wolf slid up close to
him across the snow and watched him with hungry wistfulness.

He set to work to carry out a new idea that had come to him. He
extended the fire into a large circle. Inside this circle he crouched,
his sleeping outfit under him as a protection against the melting snow.
When he had thus disappeared within his shelter of flame, the whole
pack came curiously to the rim of the fire to see what had become of
him. Hitherto they had been denied access to the fire, and they now
settled down in a close-drawn circle, like so many dogs, blinking and
yawning and stretching their lean bodies in the unaccustomed warmth.
Then the she-wolf sat down, pointed her nose at a star, and began to
howl. One by one the wolves joined her, till the whole pack, on
haunches, with noses pointed skyward, was howling its hunger cry.

Dawn came, and daylight. The fire was burning low. The fuel had run
out, and there was need to get more. The man attempted to step out of
his circle of flame, but the wolves surged to meet him. Burning brands
made them spring aside, but they no longer sprang back. In vain he
strove to drive them back. As he gave up and stumbled inside his
circle, a wolf leaped for him, missed, and landed with all four feet in
the coals. It cried out with terror, at the same time snarling, and
scrambled back to cool its paws in the snow.

The man sat down on his blankets in a crouching position. His body
leaned forward from the hips. His shoulders, relaxed and drooping, and
his head on his knees advertised that he had given up the struggle. Now
and again he raised his head to note the dying down of the fire. The
circle of flame and coals was breaking into segments with openings in
between. These openings grew in size, the segments diminished.

“I guess you can come an’ get me any time,” he mumbled. “Anyway, I’m
goin’ to sleep.”

Once he awakened, and in an opening in the circle, directly in front of
him, he saw the she-wolf gazing at him.

Again he awakened, a little later, though it seemed hours to him. A
mysterious change had taken place—so mysterious a change that he was
shocked wider awake. Something had happened. He could not understand at
first. Then he discovered it. The wolves were gone. Remained only the
trampled snow to show how closely they had pressed him. Sleep was
welling up and gripping him again, his head was sinking down upon his
knees, when he roused with a sudden start.

There were cries of men, and churn of sleds, the creaking of harnesses,
and the eager whimpering of straining dogs. Four sleds pulled in from
the river bed to the camp among the trees. Half a dozen men were about
the man who crouched in the centre of the dying fire. They were shaking
and prodding him into consciousness. He looked at them like a drunken
man and maundered in strange, sleepy speech.

“Red she-wolf. . . . Come in with the dogs at feedin’ time. . . . First
she ate the dog-food. . . . Then she ate the dogs. . . . An’ after that
she ate Bill. . . . ”

“Where’s Lord Alfred?” one of the men bellowed in his ear, shaking him
roughly.

He shook his head slowly. “No, she didn’t eat him. . . . He’s roostin’
in a tree at the last camp.”

“Dead?” the man shouted.

“An’ in a box,” Henry answered. He jerked his shoulder petulantly away
from the grip of his questioner. “Say, you lemme alone. . . . I’m jes’
plump tuckered out. . . . Goo’ night, everybody.”

His eyes fluttered and went shut. His chin fell forward on his chest.
And even as they eased him down upon the blankets his snores were
rising on the frosty air.

But there was another sound. Far and faint it was, in the remote
distance, the cry of the hungry wolf-pack as it took the trail of other
meat than the man it had just missed.

%\part{}
\chapter{The Battle of the Fangs}

It was the she-wolf who had first caught the sound of men’s voices and
the whining of the sled-dogs; and it was the she-wolf who was first to
spring away from the cornered man in his circle of dying flame. The
pack had been loath to forego the kill it had hunted down, and it
lingered for several minutes, making sure of the sounds, and then it,
too, sprang away on the trail made by the she-wolf.

Running at the forefront of the pack was a large grey wolf—one of its
several leaders. It was he who directed the pack’s course on the heels
of the she-wolf. It was he who snarled warningly at the younger members
of the pack or slashed at them with his fangs when they ambitiously
tried to pass him. And it was he who increased the pace when he sighted
the she-wolf, now trotting slowly across the snow.

She dropped in alongside by him, as though it were her appointed
position, and took the pace of the pack. He did not snarl at her, nor
show his teeth, when any leap of hers chanced to put her in advance of
him. On the contrary, he seemed kindly disposed toward her—too kindly
to suit her, for he was prone to run near to her, and when he ran too
near it was she who snarled and showed her teeth. Nor was she above
slashing his shoulder sharply on occasion. At such times he betrayed no
anger. He merely sprang to the side and ran stiffly ahead for several
awkward leaps, in carriage and conduct resembling an abashed country
swain.

This was his one trouble in the running of the pack; but she had other
troubles. On her other side ran a gaunt old wolf, grizzled and marked
with the scars of many battles. He ran always on her right side. The
fact that he had but one eye, and that the left eye, might account for
this. He, also, was addicted to crowding her, to veering toward her
till his scarred muzzle touched her body, or shoulder, or neck. As with
the running mate on the left, she repelled these attentions with her
teeth; but when both bestowed their attentions at the same time she was
roughly jostled, being compelled, with quick snaps to either side, to
drive both lovers away and at the same time to maintain her forward
leap with the pack and see the way of her feet before her. At such
times her running mates flashed their teeth and growled threateningly
across at each other. They might have fought, but even wooing and its
rivalry waited upon the more pressing hunger-need of the pack.

After each repulse, when the old wolf sheered abruptly away from the
sharp-toothed object of his desire, he shouldered against a young
three-year-old that ran on his blind right side. This young wolf had
attained his full size; and, considering the weak and famished
condition of the pack, he possessed more than the average vigour and
spirit. Nevertheless, he ran with his head even with the shoulder of
his one-eyed elder. When he ventured to run abreast of the older wolf
(which was seldom), a snarl and a snap sent him back even with the
shoulder again. Sometimes, however, he dropped cautiously and slowly
behind and edged in between the old leader and the she-wolf. This was
doubly resented, even triply resented. When she snarled her
displeasure, the old leader would whirl on the three-year-old.
Sometimes she whirled with him. And sometimes the young leader on the
left whirled, too.

At such times, confronted by three sets of savage teeth, the young wolf
stopped precipitately, throwing himself back on his haunches, with
fore-legs stiff, mouth menacing, and mane bristling. This confusion in
the front of the moving pack always caused confusion in the rear. The
wolves behind collided with the young wolf and expressed their
displeasure by administering sharp nips on his hind-legs and flanks. He
was laying up trouble for himself, for lack of food and short tempers
went together; but with the boundless faith of youth he persisted in
repeating the manoeuvre every little while, though it never succeeded
in gaining anything for him but discomfiture.

Had there been food, love-making and fighting would have gone on apace,
and the pack-formation would have been broken up. But the situation of
the pack was desperate. It was lean with long-standing hunger. It ran
below its ordinary speed. At the rear limped the weak members, the very
young and the very old. At the front were the strongest. Yet all were
more like skeletons than full-bodied wolves. Nevertheless, with the
exception of the ones that limped, the movements of the animals were
effortless and tireless. Their stringy muscles seemed founts of
inexhaustible energy. Behind every steel-like contraction of a muscle,
lay another steel-like contraction, and another, and another,
apparently without end.

They ran many miles that day. They ran through the night. And the next
day found them still running. They were running over the surface of a
world frozen and dead. No life stirred. They alone moved through the
vast inertness. They alone were alive, and they sought for other things
that were alive in order that they might devour them and continue to
live.

They crossed low divides and ranged a dozen small streams in a
lower-lying country before their quest was rewarded. Then they came
upon moose. It was a big bull they first found. Here was meat and life,
and it was guarded by no mysterious fires nor flying missiles of flame.
Splay hoofs and palmated antlers they knew, and they flung their
customary patience and caution to the wind. It was a brief fight and
fierce. The big bull was beset on every side. He ripped them open or
split their skulls with shrewdly driven blows of his great hoofs. He
crushed them and broke them on his large horns. He stamped them into
the snow under him in the wallowing struggle. But he was foredoomed,
and he went down with the she-wolf tearing savagely at his throat, and
with other teeth fixed everywhere upon him, devouring him alive, before
ever his last struggles ceased or his last damage had been wrought.

There was food in plenty. The bull weighed over eight hundred
pounds—fully twenty pounds of meat per mouth for the forty-odd wolves
of the pack. But if they could fast prodigiously, they could feed
prodigiously, and soon a few scattered bones were all that remained of
the splendid live brute that had faced the pack a few hours before.

There was now much resting and sleeping. With full stomachs, bickering
and quarrelling began among the younger males, and this continued
through the few days that followed before the breaking-up of the pack.
The famine was over. The wolves were now in the country of game, and
though they still hunted in pack, they hunted more cautiously, cutting
out heavy cows or crippled old bulls from the small moose-herds they
ran across.

There came a day, in this land of plenty, when the wolf-pack split in
half and went in different directions. The she-wolf, the young leader
on her left, and the one-eyed elder on her right, led their half of the
pack down to the Mackenzie River and across into the lake country to
the east. Each day this remnant of the pack dwindled. Two by two, male
and female, the wolves were deserting. Occasionally a solitary male was
driven out by the sharp teeth of his rivals. In the end there remained
only four: the she-wolf, the young leader, the one-eyed one, and the
ambitious three-year-old.

The she-wolf had by now developed a ferocious temper. Her three suitors
all bore the marks of her teeth. Yet they never replied in kind, never
defended themselves against her. They turned their shoulders to her
most savage slashes, and with wagging tails and mincing steps strove to
placate her wrath. But if they were all mildness toward her, they were
all fierceness toward one another. The three-year-old grew too
ambitious in his fierceness. He caught the one-eyed elder on his blind
side and ripped his ear into ribbons. Though the grizzled old fellow
could see only on one side, against the youth and vigour of the other
he brought into play the wisdom of long years of experience. His lost
eye and his scarred muzzle bore evidence to the nature of his
experience. He had survived too many battles to be in doubt for a
moment about what to do.

The battle began fairly, but it did not end fairly. There was no
telling what the outcome would have been, for the third wolf joined the
elder, and together, old leader and young leader, they attacked the
ambitious three-year-old and proceeded to destroy him. He was beset on
either side by the merciless fangs of his erstwhile comrades. Forgotten
were the days they had hunted together, the game they had pulled down,
the famine they had suffered. That business was a thing of the past.
The business of love was at hand—ever a sterner and crueller business
than that of food-getting.

And in the meanwhile, the she-wolf, the cause of it all, sat down
contentedly on her haunches and watched. She was even pleased. This was
her day—and it came not often—when manes bristled, and fang smote fang
or ripped and tore the yielding flesh, all for the possession of her.

And in the business of love the three-year-old, who had made this his
first adventure upon it, yielded up his life. On either side of his
body stood his two rivals. They were gazing at the she-wolf, who sat
smiling in the snow. But the elder leader was wise, very wise, in love
even as in battle. The younger leader turned his head to lick a wound
on his shoulder. The curve of his neck was turned toward his rival.
With his one eye the elder saw the opportunity. He darted in low and
closed with his fangs. It was a long, ripping slash, and deep as well.
His teeth, in passing, burst the wall of the great vein of the throat.
Then he leaped clear.

The young leader snarled terribly, but his snarl broke midmost into a
tickling cough. Bleeding and coughing, already stricken, he sprang at
the elder and fought while life faded from him, his legs going weak
beneath him, the light of day dulling on his eyes, his blows and
springs falling shorter and shorter.

And all the while the she-wolf sat on her haunches and smiled. She was
made glad in vague ways by the battle, for this was the love-making of
the Wild, the sex-tragedy of the natural world that was tragedy only to
those that died. To those that survived it was not tragedy, but
realisation and achievement.

When the young leader lay in the snow and moved no more, One Eye
stalked over to the she-wolf. His carriage was one of mingled triumph
and caution. He was plainly expectant of a rebuff, and he was just as
plainly surprised when her teeth did not flash out at him in anger. For
the first time she met him with a kindly manner. She sniffed noses with
him, and even condescended to leap about and frisk and play with him in
quite puppyish fashion. And he, for all his grey years and sage
experience, behaved quite as puppyishly and even a little more
foolishly.

Forgotten already were the vanquished rivals and the love-tale
red-written on the snow. Forgotten, save once, when old One Eye stopped
for a moment to lick his stiffening wounds. Then it was that his lips
half writhed into a snarl, and the hair of his neck and shoulders
involuntarily bristled, while he half crouched for a spring, his claws
spasmodically clutching into the snow-surface for firmer footing. But
it was all forgotten the next moment, as he sprang after the she-wolf,
who was coyly leading him a chase through the woods.

After that they ran side by side, like good friends who have come to an
understanding. The days passed by, and they kept together, hunting
their meat and killing and eating it in common. After a time the
she-wolf began to grow restless. She seemed to be searching for
something that she could not find. The hollows under fallen trees
seemed to attract her, and she spent much time nosing about among the
larger snow-piled crevices in the rocks and in the caves of overhanging
banks. Old One Eye was not interested at all, but he followed her
good-naturedly in her quest, and when her investigations in particular
places were unusually protracted, he would lie down and wait until she
was ready to go on.

They did not remain in one place, but travelled across country until
they regained the Mackenzie River, down which they slowly went, leaving
it often to hunt game along the small streams that entered it, but
always returning to it again. Sometimes they chanced upon other wolves,
usually in pairs; but there was no friendliness of intercourse
displayed on either side, no gladness at meeting, no desire to return
to the pack-formation. Several times they encountered solitary wolves.
These were always males, and they were pressingly insistent on joining
with One Eye and his mate. This he resented, and when she stood
shoulder to shoulder with him, bristling and showing her teeth, the
aspiring solitary ones would back off, turn-tail, and continue on their
lonely way.

One moonlight night, running through the quiet forest, One Eye suddenly
halted. His muzzle went up, his tail stiffened, and his nostrils
dilated as he scented the air. One foot also he held up, after the
manner of a dog. He was not satisfied, and he continued to smell the
air, striving to understand the message borne upon it to him. One
careless sniff had satisfied his mate, and she trotted on to reassure
him. Though he followed her, he was still dubious, and he could not
forbear an occasional halt in order more carefully to study the
warning.

She crept out cautiously on the edge of a large open space in the midst
of the trees. For some time she stood alone. Then One Eye, creeping and
crawling, every sense on the alert, every hair radiating infinite
suspicion, joined her. They stood side by side, watching and listening
and smelling.

To their ears came the sounds of dogs wrangling and scuffling, the
guttural cries of men, the sharper voices of scolding women, and once
the shrill and plaintive cry of a child. With the exception of the huge
bulks of the skin-lodges, little could be seen save the flames of the
fire, broken by the movements of intervening bodies, and the smoke
rising slowly on the quiet air. But to their nostrils came the myriad
smells of an Indian camp, carrying a story that was largely
incomprehensible to One Eye, but every detail of which the she-wolf
knew.

She was strangely stirred, and sniffed and sniffed with an increasing
delight. But old One Eye was doubtful. He betrayed his apprehension,
and started tentatively to go. She turned and touched his neck with her
muzzle in a reassuring way, then regarded the camp again. A new
wistfulness was in her face, but it was not the wistfulness of hunger.
She was thrilling to a desire that urged her to go forward, to be in
closer to that fire, to be squabbling with the dogs, and to be avoiding
and dodging the stumbling feet of men.

One Eye moved impatiently beside her; her unrest came back upon her,
and she knew again her pressing need to find the thing for which she
searched. She turned and trotted back into the forest, to the great
relief of One Eye, who trotted a little to the fore until they were
well within the shelter of the trees.

As they slid along, noiseless as shadows, in the moonlight, they came
upon a run-way. Both noses went down to the footprints in the snow.
These footprints were very fresh. One Eye ran ahead cautiously, his
mate at his heels. The broad pads of their feet were spread wide and in
contact with the snow were like velvet. One Eye caught sight of a dim
movement of white in the midst of the white. His sliding gait had been
deceptively swift, but it was as nothing to the speed at which he now
ran. Before him was bounding the faint patch of white he had
discovered.

They were running along a narrow alley flanked on either side by a
growth of young spruce. Through the trees the mouth of the alley could
be seen, opening out on a moonlit glade. Old One Eye was rapidly
overhauling the fleeing shape of white. Bound by bound he gained. Now
he was upon it. One leap more and his teeth would be sinking into it.
But that leap was never made. High in the air, and straight up, soared
the shape of white, now a struggling snowshoe rabbit that leaped and
bounded, executing a fantastic dance there above him in the air and
never once returning to earth.

One Eye sprang back with a snort of sudden fright, then shrank down to
the snow and crouched, snarling threats at this thing of fear he did
not understand. But the she-wolf coolly thrust past him. She poised for
a moment, then sprang for the dancing rabbit. She, too, soared high,
but not so high as the quarry, and her teeth clipped emptily together
with a metallic snap. She made another leap, and another.

Her mate had slowly relaxed from his crouch and was watching her. He
now evinced displeasure at her repeated failures, and himself made a
mighty spring upward. His teeth closed upon the rabbit, and he bore it
back to earth with him. But at the same time there was a suspicious
crackling movement beside him, and his astonished eye saw a young
spruce sapling bending down above him to strike him. His jaws let go
their grip, and he leaped backward to escape this strange danger, his
lips drawn back from his fangs, his throat snarling, every hair
bristling with rage and fright. And in that moment the sapling reared
its slender length upright and the rabbit soared dancing in the air
again.

The she-wolf was angry. She sank her fangs into her mate’s shoulder in
reproof; and he, frightened, unaware of what constituted this new
onslaught, struck back ferociously and in still greater fright, ripping
down the side of the she-wolf’s muzzle. For him to resent such reproof
was equally unexpected to her, and she sprang upon him in snarling
indignation. Then he discovered his mistake and tried to placate her.
But she proceeded to punish him roundly, until he gave over all
attempts at placation, and whirled in a circle, his head away from her,
his shoulders receiving the punishment of her teeth.

In the meantime the rabbit danced above them in the air. The she-wolf
sat down in the snow, and old One Eye, now more in fear of his mate
than of the mysterious sapling, again sprang for the rabbit. As he sank
back with it between his teeth, he kept his eye on the sapling. As
before, it followed him back to earth. He crouched down under the
impending blow, his hair bristling, but his teeth still keeping tight
hold of the rabbit. But the blow did not fall. The sapling remained
bent above him. When he moved it moved, and he growled at it through
his clenched jaws; when he remained still, it remained still, and he
concluded it was safer to continue remaining still. Yet the warm blood
of the rabbit tasted good in his mouth.

It was his mate who relieved him from the quandary in which he found
himself. She took the rabbit from him, and while the sapling swayed and
teetered threateningly above her she calmly gnawed off the rabbit’s
head. At once the sapling shot up, and after that gave no more trouble,
remaining in the decorous and perpendicular position in which nature
had intended it to grow. Then, between them, the she-wolf and One Eye
devoured the game which the mysterious sapling had caught for them.

There were other run-ways and alleys where rabbits were hanging in the
air, and the wolf-pair prospected them all, the she-wolf leading the
way, old One Eye following and observant, learning the method of
robbing snares—a knowledge destined to stand him in good stead in the
days to come.

\chapter{The Lair}

For two days the she-wolf and One Eye hung about the Indian camp. He
was worried and apprehensive, yet the camp lured his mate and she was
loath to depart. But when, one morning, the air was rent with the
report of a rifle close at hand, and a bullet smashed against a tree
trunk several inches from One Eye’s head, they hesitated no more, but
went off on a long, swinging lope that put quick miles between them and
the danger.

They did not go far—a couple of days’ journey. The she-wolf’s need to
find the thing for which she searched had now become imperative. She
was getting very heavy, and could run but slowly. Once, in the pursuit
of a rabbit, which she ordinarily would have caught with ease, she gave
over and lay down and rested. One Eye came to her; but when he touched
her neck gently with his muzzle she snapped at him with such quick
fierceness that he tumbled over backward and cut a ridiculous figure in
his effort to escape her teeth. Her temper was now shorter than ever;
but he had become more patient than ever and more solicitous.

And then she found the thing for which she sought. It was a few miles
up a small stream that in the summer time flowed into the Mackenzie,
but that then was frozen over and frozen down to its rocky bottom—a
dead stream of solid white from source to mouth. The she-wolf was
trotting wearily along, her mate well in advance, when she came upon
the overhanging, high clay-bank. She turned aside and trotted over to
it. The wear and tear of spring storms and melting snows had
underwashed the bank and in one place had made a small cave out of a
narrow fissure.

She paused at the mouth of the cave and looked the wall over carefully.
Then, on one side and the other, she ran along the base of the wall to
where its abrupt bulk merged from the softer-lined landscape. Returning
to the cave, she entered its narrow mouth. For a short three feet she
was compelled to crouch, then the walls widened and rose higher in a
little round chamber nearly six feet in diameter. The roof barely
cleared her head. It was dry and cosey. She inspected it with
painstaking care, while One Eye, who had returned, stood in the
entrance and patiently watched her. She dropped her head, with her nose
to the ground and directed toward a point near to her closely bunched
feet, and around this point she circled several times; then, with a
tired sigh that was almost a grunt, she curled her body in, relaxed her
legs, and dropped down, her head toward the entrance. One Eye, with
pointed, interested ears, laughed at her, and beyond, outlined against
the white light, she could see the brush of his tail waving
good-naturedly. Her own ears, with a snuggling movement, laid their
sharp points backward and down against the head for a moment, while her
mouth opened and her tongue lolled peaceably out, and in this way she
expressed that she was pleased and satisfied.

One Eye was hungry. Though he lay down in the entrance and slept, his
sleep was fitful. He kept awaking and cocking his ears at the bright
world without, where the April sun was blazing across the snow. When he
dozed, upon his ears would steal the faint whispers of hidden trickles
of running water, and he would rouse and listen intently. The sun had
come back, and all the awakening Northland world was calling to him.
Life was stirring. The feel of spring was in the air, the feel of
growing life under the snow, of sap ascending in the trees, of buds
bursting the shackles of the frost.

He cast anxious glances at his mate, but she showed no desire to get
up. He looked outside, and half a dozen snow-birds fluttered across his
field of vision. He started to get up, then looked back to his mate
again, and settled down and dozed. A shrill and minute singing stole
upon his hearing. Once, and twice, he sleepily brushed his nose with
his paw. Then he woke up. There, buzzing in the air at the tip of his
nose, was a lone mosquito. It was a full-grown mosquito, one that had
lain frozen in a dry log all winter and that had now been thawed out by
the sun. He could resist the call of the world no longer. Besides, he
was hungry.

He crawled over to his mate and tried to persuade her to get up. But
she only snarled at him, and he walked out alone into the bright
sunshine to find the snow-surface soft under foot and the travelling
difficult. He went up the frozen bed of the stream, where the snow,
shaded by the trees, was yet hard and crystalline. He was gone eight
hours, and he came back through the darkness hungrier than when he had
started. He had found game, but he had not caught it. He had broken
through the melting snow crust, and wallowed, while the snowshoe
rabbits had skimmed along on top lightly as ever.

He paused at the mouth of the cave with a sudden shock of suspicion.
Faint, strange sounds came from within. They were sounds not made by
his mate, and yet they were remotely familiar. He bellied cautiously
inside and was met by a warning snarl from the she-wolf. This he
received without perturbation, though he obeyed it by keeping his
distance; but he remained interested in the other sounds—faint, muffled
sobbings and slubberings.

His mate warned him irritably away, and he curled up and slept in the
entrance. When morning came and a dim light pervaded the lair, he again
sought after the source of the remotely familiar sounds. There was a
new note in his mate’s warning snarl. It was a jealous note, and he was
very careful in keeping a respectful distance. Nevertheless, he made
out, sheltering between her legs against the length of her body, five
strange little bundles of life, very feeble, very helpless, making tiny
whimpering noises, with eyes that did not open to the light. He was
surprised. It was not the first time in his long and successful life
that this thing had happened. It had happened many times, yet each time
it was as fresh a surprise as ever to him.

His mate looked at him anxiously. Every little while she emitted a low
growl, and at times, when it seemed to her he approached too near, the
growl shot up in her throat to a sharp snarl. Of her own experience she
had no memory of the thing happening; but in her instinct, which was
the experience of all the mothers of wolves, there lurked a memory of
fathers that had eaten their new-born and helpless progeny. It
manifested itself as a fear strong within her, that made her prevent
One Eye from more closely inspecting the cubs he had fathered.

But there was no danger. Old One Eye was feeling the urge of an
impulse, that was, in turn, an instinct that had come down to him from
all the fathers of wolves. He did not question it, nor puzzle over it.
It was there, in the fibre of his being; and it was the most natural
thing in the world that he should obey it by turning his back on his
new-born family and by trotting out and away on the meat-trail whereby
he lived.

Five or six miles from the lair, the stream divided, its forks going
off among the mountains at a right angle. Here, leading up the left
fork, he came upon a fresh track. He smelled it and found it so recent
that he crouched swiftly, and looked in the direction in which it
disappeared. Then he turned deliberately and took the right fork. The
footprint was much larger than the one his own feet made, and he knew
that in the wake of such a trail there was little meat for him.

Half a mile up the right fork, his quick ears caught the sound of
gnawing teeth. He stalked the quarry and found it to be a porcupine,
standing upright against a tree and trying his teeth on the bark. One
Eye approached carefully but hopelessly. He knew the breed, though he
had never met it so far north before; and never in his long life had
porcupine served him for a meal. But he had long since learned that
there was such a thing as Chance, or Opportunity, and he continued to
draw near. There was never any telling what might happen, for with live
things events were somehow always happening differently.

The porcupine rolled itself into a ball, radiating long, sharp needles
in all directions that defied attack. In his youth One Eye had once
sniffed too near a similar, apparently inert ball of quills, and had
the tail flick out suddenly in his face. One quill he had carried away
in his muzzle, where it had remained for weeks, a rankling flame, until
it finally worked out. So he lay down, in a comfortable crouching
position, his nose fully a foot away, and out of the line of the tail.
Thus he waited, keeping perfectly quiet. There was no telling.
Something might happen. The porcupine might unroll. There might be
opportunity for a deft and ripping thrust of paw into the tender,
unguarded belly.

But at the end of half an hour he arose, growled wrathfully at the
motionless ball, and trotted on. He had waited too often and futilely
in the past for porcupines to unroll, to waste any more time. He
continued up the right fork. The day wore along, and nothing rewarded
his hunt.

The urge of his awakened instinct of fatherhood was strong upon him. He
must find meat. In the afternoon he blundered upon a ptarmigan. He came
out of a thicket and found himself face to face with the slow-witted
bird. It was sitting on a log, not a foot beyond the end of his nose.
Each saw the other. The bird made a startled rise, but he struck it
with his paw, and smashed it down to earth, then pounced upon it, and
caught it in his teeth as it scuttled across the snow trying to rise in
the air again. As his teeth crunched through the tender flesh and
fragile bones, he began naturally to eat. Then he remembered, and,
turning on the back-track, started for home, carrying the ptarmigan in
his mouth.

A mile above the forks, running velvet-footed as was his custom, a
gliding shadow that cautiously prospected each new vista of the trail,
he came upon later imprints of the large tracks he had discovered in
the early morning. As the track led his way, he followed, prepared to
meet the maker of it at every turn of the stream.

He slid his head around a corner of rock, where began an unusually
large bend in the stream, and his quick eyes made out something that
sent him crouching swiftly down. It was the maker of the track, a large
female lynx. She was crouching as he had crouched once that day, in
front of her the tight-rolled ball of quills. If he had been a gliding
shadow before, he now became the ghost of such a shadow, as he crept
and circled around, and came up well to leeward of the silent,
motionless pair.

He lay down in the snow, depositing the ptarmigan beside him, and with
eyes peering through the needles of a low-growing spruce he watched the
play of life before him—the waiting lynx and the waiting porcupine,
each intent on life; and, such was the curiousness of the game, the way
of life for one lay in the eating of the other, and the way of life for
the other lay in being not eaten. While old One Eye, the wolf crouching
in the covert, played his part, too, in the game, waiting for some
strange freak of Chance, that might help him on the meat-trail which
was his way of life.

Half an hour passed, an hour; and nothing happened. The ball of quills
might have been a stone for all it moved; the lynx might have been
frozen to marble; and old One Eye might have been dead. Yet all three
animals were keyed to a tenseness of living that was almost painful,
and scarcely ever would it come to them to be more alive than they were
then in their seeming petrifaction.

One Eye moved slightly and peered forth with increased eagerness.
Something was happening. The porcupine had at last decided that its
enemy had gone away. Slowly, cautiously, it was unrolling its ball of
impregnable armour. It was agitated by no tremor of anticipation.
Slowly, slowly, the bristling ball straightened out and lengthened. One
Eye watching, felt a sudden moistness in his mouth and a drooling of
saliva, involuntary, excited by the living meat that was spreading
itself like a repast before him.

Not quite entirely had the porcupine unrolled when it discovered its
enemy. In that instant the lynx struck. The blow was like a flash of
light. The paw, with rigid claws curving like talons, shot under the
tender belly and came back with a swift ripping movement. Had the
porcupine been entirely unrolled, or had it not discovered its enemy a
fraction of a second before the blow was struck, the paw would have
escaped unscathed; but a side-flick of the tail sank sharp quills into
it as it was withdrawn.

Everything had happened at once—the blow, the counter-blow, the squeal
of agony from the porcupine, the big cat’s squall of sudden hurt and
astonishment. One Eye half arose in his excitement, his ears up, his
tail straight out and quivering behind him. The lynx’s bad temper got
the best of her. She sprang savagely at the thing that had hurt her.
But the porcupine, squealing and grunting, with disrupted anatomy
trying feebly to roll up into its ball-protection, flicked out its tail
again, and again the big cat squalled with hurt and astonishment. Then
she fell to backing away and sneezing, her nose bristling with quills
like a monstrous pin-cushion. She brushed her nose with her paws,
trying to dislodge the fiery darts, thrust it into the snow, and rubbed
it against twigs and branches, and all the time leaping about, ahead,
sidewise, up and down, in a frenzy of pain and fright.

She sneezed continually, and her stub of a tail was doing its best
toward lashing about by giving quick, violent jerks. She quit her
antics, and quieted down for a long minute. One Eye watched. And even
he could not repress a start and an involuntary bristling of hair along
his back when she suddenly leaped, without warning, straight up in the
air, at the same time emitting a long and most terrible squall. Then
she sprang away, up the trail, squalling with every leap she made.

It was not until her racket had faded away in the distance and died out
that One Eye ventured forth. He walked as delicately as though all the
snow were carpeted with porcupine quills, erect and ready to pierce the
soft pads of his feet. The porcupine met his approach with a furious
squealing and a clashing of its long teeth. It had managed to roll up
in a ball again, but it was not quite the old compact ball; its muscles
were too much torn for that. It had been ripped almost in half, and was
still bleeding profusely.

One Eye scooped out mouthfuls of the blood-soaked snow, and chewed and
tasted and swallowed. This served as a relish, and his hunger increased
mightily; but he was too old in the world to forget his caution. He
waited. He lay down and waited, while the porcupine grated its teeth
and uttered grunts and sobs and occasional sharp little squeals. In a
little while, One Eye noticed that the quills were drooping and that a
great quivering had set up. The quivering came to an end suddenly.
There was a final defiant clash of the long teeth. Then all the quills
drooped quite down, and the body relaxed and moved no more.

With a nervous, shrinking paw, One Eye stretched out the porcupine to
its full length and turned it over on its back. Nothing had happened.
It was surely dead. He studied it intently for a moment, then took a
careful grip with his teeth and started off down the stream, partly
carrying, partly dragging the porcupine, with head turned to the side
so as to avoid stepping on the prickly mass. He recollected something,
dropped the burden, and trotted back to where he had left the
ptarmigan. He did not hesitate a moment. He knew clearly what was to be
done, and this he did by promptly eating the ptarmigan. Then he
returned and took up his burden.

When he dragged the result of his day’s hunt into the cave, the
she-wolf inspected it, turned her muzzle to him, and lightly licked him
on the neck. But the next instant she was warning him away from the
cubs with a snarl that was less harsh than usual and that was more
apologetic than menacing. Her instinctive fear of the father of her
progeny was toning down. He was behaving as a wolf-father should, and
manifesting no unholy desire to devour the young lives she had brought
into the world.

\chapter{The Grey Cub}

He was different from his brothers and sisters. Their hair already
betrayed the reddish hue inherited from their mother, the she-wolf;
while he alone, in this particular, took after his father. He was the
one little grey cub of the litter. He had bred true to the straight
wolf-stock—in fact, he had bred true to old One Eye himself,
physically, with but a single exception, and that was he had two eyes
to his father’s one.

The grey cub’s eyes had not been open long, yet already he could see
with steady clearness. And while his eyes were still closed, he had
felt, tasted, and smelled. He knew his two brothers and his two sisters
very well. He had begun to romp with them in a feeble, awkward way, and
even to squabble, his little throat vibrating with a queer rasping
noise (the forerunner of the growl), as he worked himself into a
passion. And long before his eyes had opened he had learned by touch,
taste, and smell to know his mother—a fount of warmth and liquid food
and tenderness. She possessed a gentle, caressing tongue that soothed
him when it passed over his soft little body, and that impelled him to
snuggle close against her and to doze off to sleep.

Most of the first month of his life had been passed thus in sleeping;
but now he could see quite well, and he stayed awake for longer periods
of time, and he was coming to learn his world quite well. His world was
gloomy; but he did not know that, for he knew no other world. It was
dim-lighted; but his eyes had never had to adjust themselves to any
other light. His world was very small. Its limits were the walls of the
lair; but as he had no knowledge of the wide world outside, he was
never oppressed by the narrow confines of his existence.

But he had early discovered that one wall of his world was different
from the rest. This was the mouth of the cave and the source of light.
He had discovered that it was different from the other walls long
before he had any thoughts of his own, any conscious volitions. It had
been an irresistible attraction before ever his eyes opened and looked
upon it. The light from it had beat upon his sealed lids, and the eyes
and the optic nerves had pulsated to little, sparklike flashes,
warm-coloured and strangely pleasing. The life of his body, and of
every fibre of his body, the life that was the very substance of his
body and that was apart from his own personal life, had yearned toward
this light and urged his body toward it in the same way that the
cunning chemistry of a plant urges it toward the sun.

Always, in the beginning, before his conscious life dawned, he had
crawled toward the mouth of the cave. And in this his brothers and
sisters were one with him. Never, in that period, did any of them crawl
toward the dark corners of the back-wall. The light drew them as if
they were plants; the chemistry of the life that composed them demanded
the light as a necessity of being; and their little puppet-bodies
crawled blindly and chemically, like the tendrils of a vine. Later on,
when each developed individuality and became personally conscious of
impulsions and desires, the attraction of the light increased. They
were always crawling and sprawling toward it, and being driven back
from it by their mother.

It was in this way that the grey cub learned other attributes of his
mother than the soft, soothing, tongue. In his insistent crawling
toward the light, he discovered in her a nose that with a sharp nudge
administered rebuke, and later, a paw, that crushed him down and rolled
him over and over with swift, calculating stroke. Thus he learned hurt;
and on top of it he learned to avoid hurt, first, by not incurring the
risk of it; and second, when he had incurred the risk, by dodging and
by retreating. These were conscious actions, and were the results of
his first generalisations upon the world. Before that he had recoiled
automatically from hurt, as he had crawled automatically toward the
light. After that he recoiled from hurt because he \emph{knew} that it was
hurt.

He was a fierce little cub. So were his brothers and sisters. It was to
be expected. He was a carnivorous animal. He came of a breed of
meat-killers and meat-eaters. His father and mother lived wholly upon
meat. The milk he had sucked with his first flickering life, was milk
transformed directly from meat, and now, at a month old, when his eyes
had been open for but a week, he was beginning himself to eat meat—meat
half-digested by the she-wolf and disgorged for the five growing cubs
that already made too great demand upon her breast.

But he was, further, the fiercest of the litter. He could make a louder
rasping growl than any of them. His tiny rages were much more terrible
than theirs. It was he that first learned the trick of rolling a
fellow-cub over with a cunning paw-stroke. And it was he that first
gripped another cub by the ear and pulled and tugged and growled
through jaws tight-clenched. And certainly it was he that caused the
mother the most trouble in keeping her litter from the mouth of the
cave.

The fascination of the light for the grey cub increased from day to
day. He was perpetually departing on yard-long adventures toward the
cave’s entrance, and as perpetually being driven back. Only he did not
know it for an entrance. He did not know anything about
entrances—passages whereby one goes from one place to another place. He
did not know any other place, much less of a way to get there. So to
him the entrance of the cave was a wall—a wall of light. As the sun was
to the outside dweller, this wall was to him the sun of his world. It
attracted him as a candle attracts a moth. He was always striving to
attain it. The life that was so swiftly expanding within him, urged him
continually toward the wall of light. The life that was within him knew
that it was the one way out, the way he was predestined to tread. But
he himself did not know anything about it. He did not know there was
any outside at all.

There was one strange thing about this wall of light. His father (he
had already come to recognise his father as the one other dweller in
the world, a creature like his mother, who slept near the light and was
a bringer of meat)—his father had a way of walking right into the white
far wall and disappearing. The grey cub could not understand this.
Though never permitted by his mother to approach that wall, he had
approached the other walls, and encountered hard obstruction on the end
of his tender nose. This hurt. And after several such adventures, he
left the walls alone. Without thinking about it, he accepted this
disappearing into the wall as a peculiarity of his father, as milk and
half-digested meat were peculiarities of his mother.

In fact, the grey cub was not given to thinking—at least, to the kind
of thinking customary of men. His brain worked in dim ways. Yet his
conclusions were as sharp and distinct as those achieved by men. He had
a method of accepting things, without questioning the why and
wherefore. In reality, this was the act of classification. He was never
disturbed over why a thing happened. How it happened was sufficient for
him. Thus, when he had bumped his nose on the back-wall a few times, he
accepted that he would not disappear into walls. In the same way he
accepted that his father could disappear into walls. But he was not in
the least disturbed by desire to find out the reason for the difference
between his father and himself. Logic and physics were no part of his
mental make-up.

Like most creatures of the Wild, he early experienced famine. There
came a time when not only did the meat-supply cease, but the milk no
longer came from his mother’s breast. At first, the cubs whimpered and
cried, but for the most part they slept. It was not long before they
were reduced to a coma of hunger. There were no more spats and
squabbles, no more tiny rages nor attempts at growling; while the
adventures toward the far white wall ceased altogether. The cubs slept,
while the life that was in them flickered and died down.

One Eye was desperate. He ranged far and wide, and slept but little in
the lair that had now become cheerless and miserable. The she-wolf,
too, left her litter and went out in search of meat. In the first days
after the birth of the cubs, One Eye had journeyed several times back
to the Indian camp and robbed the rabbit snares; but, with the melting
of the snow and the opening of the streams, the Indian camp had moved
away, and that source of supply was closed to him.

When the grey cub came back to life and again took interest in the far
white wall, he found that the population of his world had been reduced.
Only one sister remained to him. The rest were gone. As he grew
stronger, he found himself compelled to play alone, for the sister no
longer lifted her head nor moved about. His little body rounded out
with the meat he now ate; but the food had come too late for her. She
slept continuously, a tiny skeleton flung round with skin in which the
flame flickered lower and lower and at last went out.

Then there came a time when the grey cub no longer saw his father
appearing and disappearing in the wall nor lying down asleep in the
entrance. This had happened at the end of a second and less severe
famine. The she-wolf knew why One Eye never came back, but there was no
way by which she could tell what she had seen to the grey cub. Hunting
herself for meat, up the left fork of the stream where lived the lynx,
she had followed a day-old trail of One Eye. And she had found him, or
what remained of him, at the end of the trail. There were many signs of
the battle that had been fought, and of the lynx’s withdrawal to her
lair after having won the victory. Before she went away, the she-wolf
had found this lair, but the signs told her that the lynx was inside,
and she had not dared to venture in.

After that, the she-wolf in her hunting avoided the left fork. For she
knew that in the lynx’s lair was a litter of kittens, and she knew the
lynx for a fierce, bad-tempered creature and a terrible fighter. It was
all very well for half a dozen wolves to drive a lynx, spitting and
bristling, up a tree; but it was quite a different matter for a lone
wolf to encounter a lynx—especially when the lynx was known to have a
litter of hungry kittens at her back.

But the Wild is the Wild, and motherhood is motherhood, at all times
fiercely protective whether in the Wild or out of it; and the time was
to come when the she-wolf, for her grey cub’s sake, would venture the
left fork, and the lair in the rocks, and the lynx’s wrath.

\chapter{The Wall of the World}

By the time his mother began leaving the cave on hunting expeditions,
the cub had learned well the law that forbade his approaching the
entrance. Not only had this law been forcibly and many times impressed
on him by his mother’s nose and paw, but in him the instinct of fear
was developing. Never, in his brief cave-life, had he encountered
anything of which to be afraid. Yet fear was in him. It had come down
to him from a remote ancestry through a thousand thousand lives. It was
a heritage he had received directly from One Eye and the she-wolf; but
to them, in turn, it had been passed down through all the generations
of wolves that had gone before. Fear!—that legacy of the Wild which no
animal may escape nor exchange for pottage.

So the grey cub knew fear, though he knew not the stuff of which fear
was made. Possibly he accepted it as one of the restrictions of life.
For he had already learned that there were such restrictions. Hunger he
had known; and when he could not appease his hunger he had felt
restriction. The hard obstruction of the cave-wall, the sharp nudge of
his mother’s nose, the smashing stroke of her paw, the hunger
unappeased of several famines, had borne in upon him that all was not
freedom in the world, that to life there was limitations and
restraints. These limitations and restraints were laws. To be obedient
to them was to escape hurt and make for happiness.

He did not reason the question out in this man fashion. He merely
classified the things that hurt and the things that did not hurt. And
after such classification he avoided the things that hurt, the
restrictions and restraints, in order to enjoy the satisfactions and
the remunerations of life.

Thus it was that in obedience to the law laid down by his mother, and
in obedience to the law of that unknown and nameless thing, fear, he
kept away from the mouth of the cave. It remained to him a white wall
of light. When his mother was absent, he slept most of the time, while
during the intervals that he was awake he kept very quiet, suppressing
the whimpering cries that tickled in his throat and strove for noise.

Once, lying awake, he heard a strange sound in the white wall. He did
not know that it was a wolverine, standing outside, all a-trembling
with its own daring, and cautiously scenting out the contents of the
cave. The cub knew only that the sniff was strange, a something
unclassified, therefore unknown and terrible—for the unknown was one of
the chief elements that went into the making of fear.

The hair bristled upon the grey cub’s back, but it bristled silently.
How was he to know that this thing that sniffed was a thing at which to
bristle? It was not born of any knowledge of his, yet it was the
visible expression of the fear that was in him, and for which, in his
own life, there was no accounting. But fear was accompanied by another
instinct—that of concealment. The cub was in a frenzy of terror, yet he
lay without movement or sound, frozen, petrified into immobility, to
all appearances dead. His mother, coming home, growled as she smelt the
wolverine’s track, and bounded into the cave and licked and nozzled him
with undue vehemence of affection. And the cub felt that somehow he had
escaped a great hurt.

But there were other forces at work in the cub, the greatest of which
was growth. Instinct and law demanded of him obedience. But growth
demanded disobedience. His mother and fear impelled him to keep away
from the white wall. Growth is life, and life is for ever destined to
make for light. So there was no damming up the tide of life that was
rising within him—rising with every mouthful of meat he swallowed, with
every breath he drew. In the end, one day, fear and obedience were
swept away by the rush of life, and the cub straddled and sprawled
toward the entrance.

Unlike any other wall with which he had had experience, this wall
seemed to recede from him as he approached. No hard surface collided
with the tender little nose he thrust out tentatively before him. The
substance of the wall seemed as permeable and yielding as light. And as
condition, in his eyes, had the seeming of form, so he entered into
what had been wall to him and bathed in the substance that composed it.

It was bewildering. He was sprawling through solidity. And ever the
light grew brighter. Fear urged him to go back, but growth drove him
on. Suddenly he found himself at the mouth of the cave. The wall,
inside which he had thought himself, as suddenly leaped back before him
to an immeasurable distance. The light had become painfully bright. He
was dazzled by it. Likewise he was made dizzy by this abrupt and
tremendous extension of space. Automatically, his eyes were adjusting
themselves to the brightness, focusing themselves to meet the increased
distance of objects. At first, the wall had leaped beyond his vision.
He now saw it again; but it had taken upon itself a remarkable
remoteness. Also, its appearance had changed. It was now a variegated
wall, composed of the trees that fringed the stream, the opposing
mountain that towered above the trees, and the sky that out-towered the
mountain.

A great fear came upon him. This was more of the terrible unknown. He
crouched down on the lip of the cave and gazed out on the world. He was
very much afraid. Because it was unknown, it was hostile to him.
Therefore the hair stood up on end along his back and his lips wrinkled
weakly in an attempt at a ferocious and intimidating snarl. Out of his
puniness and fright he challenged and menaced the whole wide world.

Nothing happened. He continued to gaze, and in his interest he forgot
to snarl. Also, he forgot to be afraid. For the time, fear had been
routed by growth, while growth had assumed the guise of curiosity. He
began to notice near objects—an open portion of the stream that flashed
in the sun, the blasted pine-tree that stood at the base of the slope,
and the slope itself, that ran right up to him and ceased two feet
beneath the lip of the cave on which he crouched.

Now the grey cub had lived all his days on a level floor. He had never
experienced the hurt of a fall. He did not know what a fall was. So he
stepped boldly out upon the air. His hind-legs still rested on the
cave-lip, so he fell forward head downward. The earth struck him a
harsh blow on the nose that made him yelp. Then he began rolling down
the slope, over and over. He was in a panic of terror. The unknown had
caught him at last. It had gripped savagely hold of him and was about
to wreak upon him some terrific hurt. Growth was now routed by fear,
and he ki-yi’d like any frightened puppy.

The unknown bore him on he knew not to what frightful hurt, and he
yelped and ki-yi’d unceasingly. This was a different proposition from
crouching in frozen fear while the unknown lurked just alongside. Now
the unknown had caught tight hold of him. Silence would do no good.
Besides, it was not fear, but terror, that convulsed him.

But the slope grew more gradual, and its base was grass-covered. Here
the cub lost momentum. When at last he came to a stop, he gave one last
agonised yell and then a long, whimpering wail. Also, and quite as a
matter of course, as though in his life he had already made a thousand
toilets, he proceeded to lick away the dry clay that soiled him.

After that he sat up and gazed about him, as might the first man of the
earth who landed upon Mars. The cub had broken through the wall of the
world, the unknown had let go its hold of him, and here he was without
hurt. But the first man on Mars would have experienced less
unfamiliarity than did he. Without any antecedent knowledge, without
any warning whatever that such existed, he found himself an explorer in
a totally new world.

Now that the terrible unknown had let go of him, he forgot that the
unknown had any terrors. He was aware only of curiosity in all the
things about him. He inspected the grass beneath him, the moss-berry
plant just beyond, and the dead trunk of the blasted pine that stood on
the edge of an open space among the trees. A squirrel, running around
the base of the trunk, came full upon him, and gave him a great fright.
He cowered down and snarled. But the squirrel was as badly scared. It
ran up the tree, and from a point of safety chattered back savagely.

This helped the cub’s courage, and though the woodpecker he next
encountered gave him a start, he proceeded confidently on his way. Such
was his confidence, that when a moose-bird impudently hopped up to him,
he reached out at it with a playful paw. The result was a sharp peck on
the end of his nose that made him cower down and ki-yi. The noise he
made was too much for the moose-bird, who sought safety in flight.

But the cub was learning. His misty little mind had already made an
unconscious classification. There were live things and things not
alive. Also, he must watch out for the live things. The things not
alive remained always in one place, but the live things moved about,
and there was no telling what they might do. The thing to expect of
them was the unexpected, and for this he must be prepared.

He travelled very clumsily. He ran into sticks and things. A twig that
he thought a long way off, would the next instant hit him on the nose
or rake along his ribs. There were inequalities of surface. Sometimes
he overstepped and stubbed his nose. Quite as often he understepped and
stubbed his feet. Then there were the pebbles and stones that turned
under him when he trod upon them; and from them he came to know that
the things not alive were not all in the same state of stable
equilibrium as was his cave—also, that small things not alive were more
liable than large things to fall down or turn over. But with every
mishap he was learning. The longer he walked, the better he walked. He
was adjusting himself. He was learning to calculate his own muscular
movements, to know his physical limitations, to measure distances
between objects, and between objects and himself.

His was the luck of the beginner. Born to be a hunter of meat (though
he did not know it), he blundered upon meat just outside his own
cave-door on his first foray into the world. It was by sheer blundering
that he chanced upon the shrewdly hidden ptarmigan nest. He fell into
it. He had essayed to walk along the trunk of a fallen pine. The rotten
bark gave way under his feet, and with a despairing yelp he pitched
down the rounded crescent, smashed through the leafage and stalks of a
small bush, and in the heart of the bush, on the ground, fetched up in
the midst of seven ptarmigan chicks.

They made noises, and at first he was frightened at them. Then he
perceived that they were very little, and he became bolder. They moved.
He placed his paw on one, and its movements were accelerated. This was
a source of enjoyment to him. He smelled it. He picked it up in his
mouth. It struggled and tickled his tongue. At the same time he was
made aware of a sensation of hunger. His jaws closed together. There
was a crunching of fragile bones, and warm blood ran in his mouth. The
taste of it was good. This was meat, the same as his mother gave him,
only it was alive between his teeth and therefore better. So he ate the
ptarmigan. Nor did he stop till he had devoured the whole brood. Then
he licked his chops in quite the same way his mother did, and began to
crawl out of the bush.

He encountered a feathered whirlwind. He was confused and blinded by
the rush of it and the beat of angry wings. He hid his head between his
paws and yelped. The blows increased. The mother ptarmigan was in a
fury. Then he became angry. He rose up, snarling, striking out with his
paws. He sank his tiny teeth into one of the wings and pulled and
tugged sturdily. The ptarmigan struggled against him, showering blows
upon him with her free wing. It was his first battle. He was elated. He
forgot all about the unknown. He no longer was afraid of anything. He
was fighting, tearing at a live thing that was striking at him. Also,
this live thing was meat. The lust to kill was on him. He had just
destroyed little live things. He would now destroy a big live thing. He
was too busy and happy to know that he was happy. He was thrilling and
exulting in ways new to him and greater to him than any he had known
before.

He held on to the wing and growled between his tight-clenched teeth.
The ptarmigan dragged him out of the bush. When she turned and tried to
drag him back into the bush’s shelter, he pulled her away from it and
on into the open. And all the time she was making outcry and striking
with her free wing, while feathers were flying like a snow-fall. The
pitch to which he was aroused was tremendous. All the fighting blood of
his breed was up in him and surging through him. This was living,
though he did not know it. He was realising his own meaning in the
world; he was doing that for which he was made—killing meat and
battling to kill it. He was justifying his existence, than which life
can do no greater; for life achieves its summit when it does to the
uttermost that which it was equipped to do.

After a time, the ptarmigan ceased her struggling. He still held her by
the wing, and they lay on the ground and looked at each other. He tried
to growl threateningly, ferociously. She pecked on his nose, which by
now, what of previous adventures was sore. He winced but held on. She
pecked him again and again. From wincing he went to whimpering. He
tried to back away from her, oblivious to the fact that by his hold on
her he dragged her after him. A rain of pecks fell on his ill-used
nose. The flood of fight ebbed down in him, and, releasing his prey, he
turned tail and scampered on across the open in inglorious retreat.

He lay down to rest on the other side of the open, near the edge of the
bushes, his tongue lolling out, his chest heaving and panting, his nose
still hurting him and causing him to continue his whimper. But as he
lay there, suddenly there came to him a feeling as of something
terrible impending. The unknown with all its terrors rushed upon him,
and he shrank back instinctively into the shelter of the bush. As he
did so, a draught of air fanned him, and a large, winged body swept
ominously and silently past. A hawk, driving down out of the blue, had
barely missed him.

While he lay in the bush, recovering from his fright and peering
fearfully out, the mother-ptarmigan on the other side of the open space
fluttered out of the ravaged nest. It was because of her loss that she
paid no attention to the winged bolt of the sky. But the cub saw, and
it was a warning and a lesson to him—the swift downward swoop of the
hawk, the short skim of its body just above the ground, the strike of
its talons in the body of the ptarmigan, the ptarmigan’s squawk of
agony and fright, and the hawk’s rush upward into the blue, carrying
the ptarmigan away with it.

It was a long time before the cub left its shelter. He had learned
much. Live things were meat. They were good to eat. Also, live things
when they were large enough, could give hurt. It was better to eat
small live things like ptarmigan chicks, and to let alone large live
things like ptarmigan hens. Nevertheless he felt a little prick of
ambition, a sneaking desire to have another battle with that ptarmigan
hen—only the hawk had carried her away. May be there were other
ptarmigan hens. He would go and see.

He came down a shelving bank to the stream. He had never seen water
before. The footing looked good. There were no inequalities of surface.
He stepped boldly out on it; and went down, crying with fear, into the
embrace of the unknown. It was cold, and he gasped, breathing quickly.
The water rushed into his lungs instead of the air that had always
accompanied his act of breathing. The suffocation he experienced was
like the pang of death. To him it signified death. He had no conscious
knowledge of death, but like every animal of the Wild, he possessed the
instinct of death. To him it stood as the greatest of hurts. It was the
very essence of the unknown; it was the sum of the terrors of the
unknown, the one culminating and unthinkable catastrophe that could
happen to him, about which he knew nothing and about which he feared
everything.

He came to the surface, and the sweet air rushed into his open mouth.
He did not go down again. Quite as though it had been a
long-established custom of his he struck out with all his legs and
began to swim. The near bank was a yard away; but he had come up with
his back to it, and the first thing his eyes rested upon was the
opposite bank, toward which he immediately began to swim. The stream
was a small one, but in the pool it widened out to a score of feet.

Midway in the passage, the current picked up the cub and swept him
downstream. He was caught in the miniature rapid at the bottom of the
pool. Here was little chance for swimming. The quiet water had become
suddenly angry. Sometimes he was under, sometimes on top. At all times
he was in violent motion, now being turned over or around, and again,
being smashed against a rock. And with every rock he struck, he yelped.
His progress was a series of yelps, from which might have been adduced
the number of rocks he encountered.

Below the rapid was a second pool, and here, captured by the eddy, he
was gently borne to the bank, and as gently deposited on a bed of
gravel. He crawled frantically clear of the water and lay down. He had
learned some more about the world. Water was not alive. Yet it moved.
Also, it looked as solid as the earth, but was without any solidity at
all. His conclusion was that things were not always what they appeared
to be. The cub’s fear of the unknown was an inherited distrust, and it
had now been strengthened by experience. Thenceforth, in the nature of
things, he would possess an abiding distrust of appearances. He would
have to learn the reality of a thing before he could put his faith into
it.

One other adventure was destined for him that day. He had recollected
that there was such a thing in the world as his mother. And then there
came to him a feeling that he wanted her more than all the rest of the
things in the world. Not only was his body tired with the adventures it
had undergone, but his little brain was equally tired. In all the days
he had lived it had not worked so hard as on this one day. Furthermore,
he was sleepy. So he started out to look for the cave and his mother,
feeling at the same time an overwhelming rush of loneliness and
helplessness.

He was sprawling along between some bushes, when he heard a sharp
intimidating cry. There was a flash of yellow before his eyes. He saw a
weasel leaping swiftly away from him. It was a small live thing, and he
had no fear. Then, before him, at his feet, he saw an extremely small
live thing, only several inches long, a young weasel, that, like
himself, had disobediently gone out adventuring. It tried to retreat
before him. He turned it over with his paw. It made a queer, grating
noise. The next moment the flash of yellow reappeared before his eyes.
He heard again the intimidating cry, and at the same instant received a
sharp blow on the side of the neck and felt the sharp teeth of the
mother-weasel cut into his flesh.

While he yelped and ki-yi’d and scrambled backward, he saw the
mother-weasel leap upon her young one and disappear with it into the
neighbouring thicket. The cut of her teeth in his neck still hurt, but
his feelings were hurt more grievously, and he sat down and weakly
whimpered. This mother-weasel was so small and so savage. He was yet to
learn that for size and weight the weasel was the most ferocious,
vindictive, and terrible of all the killers of the Wild. But a portion
of this knowledge was quickly to be his.

He was still whimpering when the mother-weasel reappeared. She did not
rush him, now that her young one was safe. She approached more
cautiously, and the cub had full opportunity to observe her lean,
snakelike body, and her head, erect, eager, and snake-like itself. Her
sharp, menacing cry sent the hair bristling along his back, and he
snarled warningly at her. She came closer and closer. There was a leap,
swifter than his unpractised sight, and the lean, yellow body
disappeared for a moment out of the field of his vision. The next
moment she was at his throat, her teeth buried in his hair and flesh.

At first he snarled and tried to fight; but he was very young, and this
was only his first day in the world, and his snarl became a whimper,
his fight a struggle to escape. The weasel never relaxed her hold. She
hung on, striving to press down with her teeth to the great vein where
his life-blood bubbled. The weasel was a drinker of blood, and it was
ever her preference to drink from the throat of life itself.

The grey cub would have died, and there would have been no story to
write about him, had not the she-wolf come bounding through the bushes.
The weasel let go the cub and flashed at the she-wolf’s throat,
missing, but getting a hold on the jaw instead. The she-wolf flirted
her head like the snap of a whip, breaking the weasel’s hold and
flinging it high in the air. And, still in the air, the she-wolf’s jaws
closed on the lean, yellow body, and the weasel knew death between the
crunching teeth.

The cub experienced another access of affection on the part of his
mother. Her joy at finding him seemed even greater than his joy at
being found. She nozzled him and caressed him and licked the cuts made
in him by the weasel’s teeth. Then, between them, mother and cub, they
ate the blood-drinker, and after that went back to the cave and slept.

\chapter{The Law of the Meat}

The cub’s development was rapid. He rested for two days, and then
ventured forth from the cave again. It was on this adventure that he
found the young weasel whose mother he had helped eat, and he saw to it
that the young weasel went the way of its mother. But on this trip he
did not get lost. When he grew tired, he found his way back to the cave
and slept. And every day thereafter found him out and ranging a wider
area.

He began to get accurate measurement of his strength and his weakness,
and to know when to be bold and when to be cautious. He found it
expedient to be cautious all the time, except for the rare moments,
when, assured of his own intrepidity, he abandoned himself to petty
rages and lusts.

He was always a little demon of fury when he chanced upon a stray
ptarmigan. Never did he fail to respond savagely to the chatter of the
squirrel he had first met on the blasted pine. While the sight of a
moose-bird almost invariably put him into the wildest of rages; for he
never forgot the peck on the nose he had received from the first of
that ilk he encountered.

But there were times when even a moose-bird failed to affect him, and
those were times when he felt himself to be in danger from some other
prowling meat hunter. He never forgot the hawk, and its moving shadow
always sent him crouching into the nearest thicket. He no longer
sprawled and straddled, and already he was developing the gait of his
mother, slinking and furtive, apparently without exertion, yet sliding
along with a swiftness that was as deceptive as it was imperceptible.

In the matter of meat, his luck had been all in the beginning. The
seven ptarmigan chicks and the baby weasel represented the sum of his
killings. His desire to kill strengthened with the days, and he
cherished hungry ambitions for the squirrel that chattered so volubly
and always informed all wild creatures that the wolf-cub was
approaching. But as birds flew in the air, squirrels could climb trees,
and the cub could only try to crawl unobserved upon the squirrel when
it was on the ground.

The cub entertained a great respect for his mother. She could get meat,
and she never failed to bring him his share. Further, she was unafraid
of things. It did not occur to him that this fearlessness was founded
upon experience and knowledge. Its effect on him was that of an
impression of power. His mother represented power; and as he grew older
he felt this power in the sharper admonishment of her paw; while the
reproving nudge of her nose gave place to the slash of her fangs. For
this, likewise, he respected his mother. She compelled obedience from
him, and the older he grew the shorter grew her temper.

Famine came again, and the cub with clearer consciousness knew once
more the bite of hunger. The she-wolf ran herself thin in the quest for
meat. She rarely slept any more in the cave, spending most of her time
on the meat-trail, and spending it vainly. This famine was not a long
one, but it was severe while it lasted. The cub found no more milk in
his mother’s breast, nor did he get one mouthful of meat for himself.

Before, he had hunted in play, for the sheer joyousness of it; now he
hunted in deadly earnestness, and found nothing. Yet the failure of it
accelerated his development. He studied the habits of the squirrel with
greater carefulness, and strove with greater craft to steal upon it and
surprise it. He studied the wood-mice and tried to dig them out of
their burrows; and he learned much about the ways of moose-birds and
woodpeckers. And there came a day when the hawk’s shadow did not drive
him crouching into the bushes. He had grown stronger and wiser, and
more confident. Also, he was desperate. So he sat on his haunches,
conspicuously in an open space, and challenged the hawk down out of the
sky. For he knew that there, floating in the blue above him, was meat,
the meat his stomach yearned after so insistently. But the hawk refused
to come down and give battle, and the cub crawled away into a thicket
and whimpered his disappointment and hunger.

The famine broke. The she-wolf brought home meat. It was strange meat,
different from any she had ever brought before. It was a lynx kitten,
partly grown, like the cub, but not so large. And it was all for him.
His mother had satisfied her hunger elsewhere; though he did not know
that it was the rest of the lynx litter that had gone to satisfy her.
Nor did he know the desperateness of her deed. He knew only that the
velvet-furred kitten was meat, and he ate and waxed happier with every
mouthful.

A full stomach conduces to inaction, and the cub lay in the cave,
sleeping against his mother’s side. He was aroused by her snarling.
Never had he heard her snarl so terribly. Possibly in her whole life it
was the most terrible snarl she ever gave. There was reason for it, and
none knew it better than she. A lynx’s lair is not despoiled with
impunity. In the full glare of the afternoon light, crouching in the
entrance of the cave, the cub saw the lynx-mother. The hair rippled up
along his back at the sight. Here was fear, and it did not require his
instinct to tell him of it. And if sight alone were not sufficient, the
cry of rage the intruder gave, beginning with a snarl and rushing
abruptly upward into a hoarse screech, was convincing enough in itself.

The cub felt the prod of the life that was in him, and stood up and
snarled valiantly by his mother’s side. But she thrust him
ignominiously away and behind her. Because of the low-roofed entrance
the lynx could not leap in, and when she made a crawling rush of it the
she-wolf sprang upon her and pinned her down. The cub saw little of the
battle. There was a tremendous snarling and spitting and screeching.
The two animals threshed about, the lynx ripping and tearing with her
claws and using her teeth as well, while the she-wolf used her teeth
alone.

Once, the cub sprang in and sank his teeth into the hind leg of the
lynx. He clung on, growling savagely. Though he did not know it, by the
weight of his body he clogged the action of the leg and thereby saved
his mother much damage. A change in the battle crushed him under both
their bodies and wrenched loose his hold. The next moment the two
mothers separated, and, before they rushed together again, the lynx
lashed out at the cub with a huge fore-paw that ripped his shoulder
open to the bone and sent him hurtling sidewise against the wall. Then
was added to the uproar the cub’s shrill yelp of pain and fright. But
the fight lasted so long that he had time to cry himself out and to
experience a second burst of courage; and the end of the battle found
him again clinging to a hind-leg and furiously growling between his
teeth.

The lynx was dead. But the she-wolf was very weak and sick. At first
she caressed the cub and licked his wounded shoulder; but the blood she
had lost had taken with it her strength, and for all of a day and a
night she lay by her dead foe’s side, without movement, scarcely
breathing. For a week she never left the cave, except for water, and
then her movements were slow and painful. At the end of that time the
lynx was devoured, while the she-wolf’s wounds had healed sufficiently
to permit her to take the meat-trail again.

The cub’s shoulder was stiff and sore, and for some time he limped from
the terrible slash he had received. But the world now seemed changed.
He went about in it with greater confidence, with a feeling of prowess
that had not been his in the days before the battle with the lynx. He
had looked upon life in a more ferocious aspect; he had fought; he had
buried his teeth in the flesh of a foe; and he had survived. And
because of all this, he carried himself more boldly, with a touch of
defiance that was new in him. He was no longer afraid of minor things,
and much of his timidity had vanished, though the unknown never ceased
to press upon him with its mysteries and terrors, intangible and
ever-menacing.

He began to accompany his mother on the meat-trail, and he saw much of
the killing of meat and began to play his part in it. And in his own
dim way he learned the law of meat. There were two kinds of life—his
own kind and the other kind. His own kind included his mother and
himself. The other kind included all live things that moved. But the
other kind was divided. One portion was what his own kind killed and
ate. This portion was composed of the non-killers and the small
killers. The other portion killed and ate his own kind, or was killed
and eaten by his own kind. And out of this classification arose the
law. The aim of life was meat. Life itself was meat. Life lived on
life. There were the eaters and the eaten. The law was: EAT OR BE
EATEN. He did not formulate the law in clear, set terms and moralise
about it. He did not even think the law; he merely lived the law
without thinking about it at all.

He saw the law operating around him on every side. He had eaten the
ptarmigan chicks. The hawk had eaten the ptarmigan-mother. The hawk
would also have eaten him. Later, when he had grown more formidable, he
wanted to eat the hawk. He had eaten the lynx kitten. The lynx-mother
would have eaten him had she not herself been killed and eaten. And so
it went. The law was being lived about him by all live things, and he
himself was part and parcel of the law. He was a killer. His only food
was meat, live meat, that ran away swiftly before him, or flew into the
air, or climbed trees, or hid in the ground, or faced him and fought
with him, or turned the tables and ran after him.

Had the cub thought in man-fashion, he might have epitomised life as a
voracious appetite and the world as a place wherein ranged a multitude
of appetites, pursuing and being pursued, hunting and being hunted,
eating and being eaten, all in blindness and confusion, with violence
and disorder, a chaos of gluttony and slaughter, ruled over by chance,
merciless, planless, endless.

But the cub did not think in man-fashion. He did not look at things
with wide vision. He was single-purposed, and entertained but one
thought or desire at a time. Besides the law of meat, there were a
myriad other and lesser laws for him to learn and obey. The world was
filled with surprise. The stir of the life that was in him, the play of
his muscles, was an unending happiness. To run down meat was to
experience thrills and elations. His rages and battles were pleasures.
Terror itself, and the mystery of the unknown, led to his living.

And there were easements and satisfactions. To have a full stomach, to
doze lazily in the sunshine—such things were remuneration in full for
his ardours and toils, while his ardours and tolls were in themselves
self-remunerative. They were expressions of life, and life is always
happy when it is expressing itself. So the cub had no quarrel with his
hostile environment. He was very much alive, very happy, and very proud
of himself.

%\part{}
\chapter{The Makers of Fire}

The cub came upon it suddenly. It was his own fault. He had been
careless. He had left the cave and run down to the stream to drink. It
might have been that he took no notice because he was heavy with sleep.
(He had been out all night on the meat-trail, and had but just then
awakened.) And his carelessness might have been due to the familiarity
of the trail to the pool. He had travelled it often, and nothing had
ever happened on it.

He went down past the blasted pine, crossed the open space, and trotted
in amongst the trees. Then, at the same instant, he saw and smelt.
Before him, sitting silently on their haunches, were five live things,
the like of which he had never seen before. It was his first glimpse of
mankind. But at the sight of him the five men did not spring to their
feet, nor show their teeth, nor snarl. They did not move, but sat
there, silent and ominous.

Nor did the cub move. Every instinct of his nature would have impelled
him to dash wildly away, had there not suddenly and for the first time
arisen in him another and counter instinct. A great awe descended upon
him. He was beaten down to movelessness by an overwhelming sense of his
own weakness and littleness. Here was mastery and power, something far
and away beyond him.

The cub had never seen man, yet the instinct concerning man was his. In
dim ways he recognised in man the animal that had fought itself to
primacy over the other animals of the Wild. Not alone out of his own
eyes, but out of the eyes of all his ancestors was the cub now looking
upon man—out of eyes that had circled in the darkness around countless
winter camp-fires, that had peered from safe distances and from the
hearts of thickets at the strange, two-legged animal that was lord over
living things. The spell of the cub’s heritage was upon him, the fear
and the respect born of the centuries of struggle and the accumulated
experience of the generations. The heritage was too compelling for a
wolf that was only a cub. Had he been full-grown, he would have run
away. As it was, he cowered down in a paralysis of fear, already half
proffering the submission that his kind had proffered from the first
time a wolf came in to sit by man’s fire and be made warm.

One of the Indians arose and walked over to him and stooped above him.
The cub cowered closer to the ground. It was the unknown, objectified
at last, in concrete flesh and blood, bending over him and reaching
down to seize hold of him. His hair bristled involuntarily; his lips
writhed back and his little fangs were bared. The hand, poised like
doom above him, hesitated, and the man spoke laughing, “«EMPH[Wabam wabisca
ip pit tah]».” (“Look! The white fangs!”)

The other Indians laughed loudly, and urged the man on to pick up the
cub. As the hand descended closer and closer, there raged within the
cub a battle of the instincts. He experienced two great impulsions—to
yield and to fight. The resulting action was a compromise. He did both.
He yielded till the hand almost touched him. Then he fought, his teeth
flashing in a snap that sank them into the hand. The next moment he
received a clout alongside the head that knocked him over on his side.
Then all fight fled out of him. His puppyhood and the instinct of
submission took charge of him. He sat up on his haunches and ki-yi’d.
But the man whose hand he had bitten was angry. The cub received a
clout on the other side of his head. Whereupon he sat up and ki-yi’d
louder than ever.

The four Indians laughed more loudly, while even the man who had been
bitten began to laugh. They surrounded the cub and laughed at him,
while he wailed out his terror and his hurt. In the midst of it, he
heard something. The Indians heard it too. But the cub knew what it
was, and with a last, long wail that had in it more of triumph than
grief, he ceased his noise and waited for the coming of his mother, of
his ferocious and indomitable mother who fought and killed all things
and was never afraid. She was snarling as she ran. She had heard the
cry of her cub and was dashing to save him.

She bounded in amongst them, her anxious and militant motherhood making
her anything but a pretty sight. But to the cub the spectacle of her
protective rage was pleasing. He uttered a glad little cry and bounded
to meet her, while the man-animals went back hastily several steps. The
she-wolf stood over against her cub, facing the men, with bristling
hair, a snarl rumbling deep in her throat. Her face was distorted and
malignant with menace, even the bridge of the nose wrinkling from tip
to eyes so prodigious was her snarl.

Then it was that a cry went up from one of the men. “Kiche!” was what
he uttered. It was an exclamation of surprise. The cub felt his mother
wilting at the sound.

“Kiche!” the man cried again, this time with sharpness and authority.

And then the cub saw his mother, the she-wolf, the fearless one,
crouching down till her belly touched the ground, whimpering, wagging
her tail, making peace signs. The cub could not understand. He was
appalled. The awe of man rushed over him again. His instinct had been
true. His mother verified it. She, too, rendered submission to the
man-animals.

The man who had spoken came over to her. He put his hand upon her head,
and she only crouched closer. She did not snap, nor threaten to snap.
The other men came up, and surrounded her, and felt her, and pawed her,
which actions she made no attempt to resent. They were greatly excited,
and made many noises with their mouths. These noises were not
indication of danger, the cub decided, as he crouched near his mother
still bristling from time to time but doing his best to submit.

“It is not strange,” an Indian was saying. “Her father was a wolf. It
is true, her mother was a dog; but did not my brother tie her out in
the woods all of three nights in the mating season? Therefore was the
father of Kiche a wolf.”

“It is a year, Grey Beaver, since she ran away,” spoke a second Indian.

“It is not strange, Salmon Tongue,” Grey Beaver answered. “It was the
time of the famine, and there was no meat for the dogs.”

“She has lived with the wolves,” said a third Indian.

“So it would seem, Three Eagles,” Grey Beaver answered, laying his hand
on the cub; “and this be the sign of it.”

The cub snarled a little at the touch of the hand, and the hand flew
back to administer a clout. Whereupon the cub covered its fangs, and
sank down submissively, while the hand, returning, rubbed behind his
ears, and up and down his back.

“This be the sign of it,” Grey Beaver went on. “It is plain that his
mother is Kiche. But his father was a wolf. Wherefore is there in him
little dog and much wolf. His fangs be white, and White Fang shall be
his name. I have spoken. He is my dog. For was not Kiche my brother’s
dog? And is not my brother dead?”

The cub, who had thus received a name in the world, lay and watched.
For a time the man-animals continued to make their mouth-noises. Then
Grey Beaver took a knife from a sheath that hung around his neck, and
went into the thicket and cut a stick. White Fang watched him. He
notched the stick at each end and in the notches fastened strings of
raw-hide. One string he tied around the throat of Kiche. Then he led
her to a small pine, around which he tied the other string.

White Fang followed and lay down beside her. Salmon Tongue’s hand
reached out to him and rolled him over on his back. Kiche looked on
anxiously. White Fang felt fear mounting in him again. He could not
quite suppress a snarl, but he made no offer to snap. The hand, with
fingers crooked and spread apart, rubbed his stomach in a playful way
and rolled him from side to side. It was ridiculous and ungainly, lying
there on his back with legs sprawling in the air. Besides, it was a
position of such utter helplessness that White Fang’s whole nature
revolted against it. He could do nothing to defend himself. If this
man-animal intended harm, White Fang knew that he could not escape it.
How could he spring away with his four legs in the air above him? Yet
submission made him master his fear, and he only growled softly. This
growl he could not suppress; nor did the man-animal resent it by giving
him a blow on the head. And furthermore, such was the strangeness of
it, White Fang experienced an unaccountable sensation of pleasure as
the hand rubbed back and forth. When he was rolled on his side he
ceased to growl, when the fingers pressed and prodded at the base of
his ears the pleasurable sensation increased; and when, with a final
rub and scratch, the man left him alone and went away, all fear had
died out of White Fang. He was to know fear many times in his dealing
with man; yet it was a token of the fearless companionship with man
that was ultimately to be his.

After a time, White Fang heard strange noises approaching. He was quick
in his classification, for he knew them at once for man-animal noises.
A few minutes later the remainder of the tribe, strung out as it was on
the march, trailed in. There were more men and many women and children,
forty souls of them, and all heavily burdened with camp equipage and
outfit. Also there were many dogs; and these, with the exception of the
part-grown puppies, were likewise burdened with camp outfit. On their
backs, in bags that fastened tightly around underneath, the dogs
carried from twenty to thirty pounds of weight.

White Fang had never seen dogs before, but at sight of them he felt
that they were his own kind, only somehow different. But they displayed
little difference from the wolf when they discovered the cub and his
mother. There was a rush. White Fang bristled and snarled and snapped
in the face of the open-mouthed oncoming wave of dogs, and went down
and under them, feeling the sharp slash of teeth in his body, himself
biting and tearing at the legs and bellies above him. There was a great
uproar. He could hear the snarl of Kiche as she fought for him; and he
could hear the cries of the man-animals, the sound of clubs striking
upon bodies, and the yelps of pain from the dogs so struck.

Only a few seconds elapsed before he was on his feet again. He could
now see the man-animals driving back the dogs with clubs and stones,
defending him, saving him from the savage teeth of his kind that
somehow was not his kind. And though there was no reason in his brain
for a clear conception of so abstract a thing as justice, nevertheless,
in his own way, he felt the justice of the man-animals, and he knew
them for what they were—makers of law and executors of law. Also, he
appreciated the power with which they administered the law. Unlike any
animals he had ever encountered, they did not bite nor claw. They
enforced their live strength with the power of dead things. Dead things
did their bidding. Thus, sticks and stones, directed by these strange
creatures, leaped through the air like living things, inflicting
grievous hurts upon the dogs.

To his mind this was power unusual, power inconceivable and beyond the
natural, power that was godlike. White Fang, in the very nature of him,
could never know anything about gods; at the best he could know only
things that were beyond knowing—but the wonder and awe that he had of
these man-animals in ways resembled what would be the wonder and awe of
man at sight of some celestial creature, on a mountain top, hurling
thunderbolts from either hand at an astonished world.

The last dog had been driven back. The hubbub died down. And White Fang
licked his hurts and meditated upon this, his first taste of
pack-cruelty and his introduction to the pack. He had never dreamed
that his own kind consisted of more than One Eye, his mother, and
himself. They had constituted a kind apart, and here, abruptly, he had
discovered many more creatures apparently of his own kind. And there
was a subconscious resentment that these, his kind, at first sight had
pitched upon him and tried to destroy him. In the same way he resented
his mother being tied with a stick, even though it was done by the
superior man-animals. It savoured of the trap, of bondage. Yet of the
trap and of bondage he knew nothing. Freedom to roam and run and lie
down at will, had been his heritage; and here it was being infringed
upon. His mother’s movements were restricted to the length of a stick,
and by the length of that same stick was he restricted, for he had not
yet got beyond the need of his mother’s side.

He did not like it. Nor did he like it when the man-animals arose and
went on with their march; for a tiny man-animal took the other end of
the stick and led Kiche captive behind him, and behind Kiche followed
White Fang, greatly perturbed and worried by this new adventure he had
entered upon.

They went down the valley of the stream, far beyond White Fang’s widest
ranging, until they came to the end of the valley, where the stream ran
into the Mackenzie River. Here, where canoes were cached on poles high
in the air and where stood fish-racks for the drying of fish, camp was
made; and White Fang looked on with wondering eyes. The superiority of
these man-animals increased with every moment. There was their mastery
over all these sharp-fanged dogs. It breathed of power. But greater
than that, to the wolf-cub, was their mastery over things not alive;
their capacity to communicate motion to unmoving things; their capacity
to change the very face of the world.

It was this last that especially affected him. The elevation of frames
of poles caught his eye; yet this in itself was not so remarkable,
being done by the same creatures that flung sticks and stones to great
distances. But when the frames of poles were made into tepees by being
covered with cloth and skins, White Fang was astounded. It was the
colossal bulk of them that impressed him. They arose around him, on
every side, like some monstrous quick-growing form of life. They
occupied nearly the whole circumference of his field of vision. He was
afraid of them. They loomed ominously above him; and when the breeze
stirred them into huge movements, he cowered down in fear, keeping his
eyes warily upon them, and prepared to spring away if they attempted to
precipitate themselves upon him.

But in a short while his fear of the tepees passed away. He saw the
women and children passing in and out of them without harm, and he saw
the dogs trying often to get into them, and being driven away with
sharp words and flying stones. After a time, he left Kiche’s side and
crawled cautiously toward the wall of the nearest tepee. It was the
curiosity of growth that urged him on—the necessity of learning and
living and doing that brings experience. The last few inches to the
wall of the tepee were crawled with painful slowness and precaution.
The day’s events had prepared him for the unknown to manifest itself in
most stupendous and unthinkable ways. At last his nose touched the
canvas. He waited. Nothing happened. Then he smelled the strange
fabric, saturated with the man-smell. He closed on the canvas with his
teeth and gave a gentle tug. Nothing happened, though the adjacent
portions of the tepee moved. He tugged harder. There was a greater
movement. It was delightful. He tugged still harder, and repeatedly,
until the whole tepee was in motion. Then the sharp cry of a squaw
inside sent him scampering back to Kiche. But after that he was afraid
no more of the looming bulks of the tepees.

A moment later he was straying away again from his mother. Her stick
was tied to a peg in the ground and she could not follow him. A
part-grown puppy, somewhat larger and older than he, came toward him
slowly, with ostentatious and belligerent importance. The puppy’s name,
as White Fang was afterward to hear him called, was Lip-lip. He had had
experience in puppy fights and was already something of a bully.

Lip-lip was White Fang’s own kind, and, being only a puppy, did not
seem dangerous; so White Fang prepared to meet him in a friendly
spirit. But when the strangers walk became stiff-legged and his lips
lifted clear of his teeth, White Fang stiffened too, and answered with
lifted lips. They half circled about each other, tentatively, snarling
and bristling. This lasted several minutes, and White Fang was
beginning to enjoy it, as a sort of game. But suddenly, with remarkable
swiftness, Lip-lip leaped in, delivering a slashing snap, and leaped
away again. The snap had taken effect on the shoulder that had been
hurt by the lynx and that was still sore deep down near the bone. The
surprise and hurt of it brought a yelp out of White Fang; but the next
moment, in a rush of anger, he was upon Lip-lip and snapping viciously.

But Lip-lip had lived his life in camp and had fought many puppy
fights. Three times, four times, and half a dozen times, his sharp
little teeth scored on the newcomer, until White Fang, yelping
shamelessly, fled to the protection of his mother. It was the first of
the many fights he was to have with Lip-lip, for they were enemies from
the start, born so, with natures destined perpetually to clash.

Kiche licked White Fang soothingly with her tongue, and tried to
prevail upon him to remain with her. But his curiosity was rampant, and
several minutes later he was venturing forth on a new quest. He came
upon one of the man-animals, Grey Beaver, who was squatting on his hams
and doing something with sticks and dry moss spread before him on the
ground. White Fang came near to him and watched. Grey Beaver made
mouth-noises which White Fang interpreted as not hostile, so he came
still nearer.

Women and children were carrying more sticks and branches to Grey
Beaver. It was evidently an affair of moment. White Fang came in until
he touched Grey Beaver’s knee, so curious was he, and already forgetful
that this was a terrible man-animal. Suddenly he saw a strange thing
like mist beginning to arise from the sticks and moss beneath Grey
Beaver’s hands. Then, amongst the sticks themselves, appeared a live
thing, twisting and turning, of a colour like the colour of the sun in
the sky. White Fang knew nothing about fire. It drew him as the light,
in the mouth of the cave had drawn him in his early puppyhood. He
crawled the several steps toward the flame. He heard Grey Beaver
chuckle above him, and he knew the sound was not hostile. Then his nose
touched the flame, and at the same instant his little tongue went out
to it.

For a moment he was paralysed. The unknown, lurking in the midst of the
sticks and moss, was savagely clutching him by the nose. He scrambled
backward, bursting out in an astonished explosion of ki-yi’s. At the
sound, Kiche leaped snarling to the end of her stick, and there raged
terribly because she could not come to his aid. But Grey Beaver laughed
loudly, and slapped his thighs, and told the happening to all the rest
of the camp, till everybody was laughing uproariously. But White Fang
sat on his haunches and ki-yi’d and ki-yi’d, a forlorn and pitiable
little figure in the midst of the man-animals.

It was the worst hurt he had ever known. Both nose and tongue had been
scorched by the live thing, sun-coloured, that had grown up under Grey
Beaver’s hands. He cried and cried interminably, and every fresh wail
was greeted by bursts of laughter on the part of the man-animals. He
tried to soothe his nose with his tongue, but the tongue was burnt too,
and the two hurts coming together produced greater hurt; whereupon he
cried more hopelessly and helplessly than ever.

And then shame came to him. He knew laughter and the meaning of it. It
is not given us to know how some animals know laughter, and know when
they are being laughed at; but it was this same way that White Fang
knew it. And he felt shame that the man-animals should be laughing at
him. He turned and fled away, not from the hurt of the fire, but from
the laughter that sank even deeper, and hurt in the spirit of him. And
he fled to Kiche, raging at the end of her stick like an animal gone
mad—to Kiche, the one creature in the world who was not laughing at
him.

Twilight drew down and night came on, and White Fang lay by his
mother’s side. His nose and tongue still hurt, but he was perplexed by
a greater trouble. He was homesick. He felt a vacancy in him, a need
for the hush and quietude of the stream and the cave in the cliff. Life
had become too populous. There were so many of the man-animals, men,
women, and children, all making noises and irritations. And there were
the dogs, ever squabbling and bickering, bursting into uproars and
creating confusions. The restful loneliness of the only life he had
known was gone. Here the very air was palpitant with life. It hummed
and buzzed unceasingly. Continually changing its intensity and abruptly
variant in pitch, it impinged on his nerves and senses, made him
nervous and restless and worried him with a perpetual imminence of
happening.

He watched the man-animals coming and going and moving about the camp.
In fashion distantly resembling the way men look upon the gods they
create, so looked White Fang upon the man-animals before him. They were
superior creatures, of a verity, gods. To his dim comprehension they
were as much wonder-workers as gods are to men. They were creatures of
mastery, possessing all manner of unknown and impossible potencies,
overlords of the alive and the not alive—making obey that which moved,
imparting movement to that which did not move, and making life,
sun-coloured and biting life, to grow out of dead moss and wood. They
were fire-makers! They were gods.

\chapter{The Bondage}

The days were thronged with experience for White Fang. During the time
that Kiche was tied by the stick, he ran about over all the camp,
inquiring, investigating, learning. He quickly came to know much of the
ways of the man-animals, but familiarity did not breed contempt. The
more he came to know them, the more they vindicated their superiority,
the more they displayed their mysterious powers, the greater loomed
their god-likeness.

To man has been given the grief, often, of seeing his gods overthrown
and his altars crumbling; but to the wolf and the wild dog that have
come in to crouch at man’s feet, this grief has never come. Unlike man,
whose gods are of the unseen and the overguessed, vapours and mists of
fancy eluding the garmenture of reality, wandering wraiths of desired
goodness and power, intangible out-croppings of self into the realm of
spirit—unlike man, the wolf and the wild dog that have come in to the
fire find their gods in the living flesh, solid to the touch, occupying
earth-space and requiring time for the accomplishment of their ends and
their existence. No effort of faith is necessary to believe in such a
god; no effort of will can possibly induce disbelief in such a god.
There is no getting away from it. There it stands, on its two
hind-legs, club in hand, immensely potential, passionate and wrathful
and loving, god and mystery and power all wrapped up and around by
flesh that bleeds when it is torn and that is good to eat like any
flesh.

And so it was with White Fang. The man-animals were gods unmistakable
and unescapable. As his mother, Kiche, had rendered her allegiance to
them at the first cry of her name, so he was beginning to render his
allegiance. He gave them the trail as a privilege indubitably theirs.
When they walked, he got out of their way. When they called, he came.
When they threatened, he cowered down. When they commanded him to go,
he went away hurriedly. For behind any wish of theirs was power to
enforce that wish, power that hurt, power that expressed itself in
clouts and clubs, in flying stones and stinging lashes of whips.

He belonged to them as all dogs belonged to them. His actions were
theirs to command. His body was theirs to maul, to stamp upon, to
tolerate. Such was the lesson that was quickly borne in upon him. It
came hard, going as it did, counter to much that was strong and
dominant in his own nature; and, while he disliked it in the learning
of it, unknown to himself he was learning to like it. It was a placing
of his destiny in another’s hands, a shifting of the responsibilities
of existence. This in itself was compensation, for it is always easier
to lean upon another than to stand alone.

But it did not all happen in a day, this giving over of himself, body
and soul, to the man-animals. He could not immediately forego his wild
heritage and his memories of the Wild. There were days when he crept to
the edge of the forest and stood and listened to something calling him
far and away. And always he returned, restless and uncomfortable, to
whimper softly and wistfully at Kiche’s side and to lick her face with
eager, questioning tongue.

White Fang learned rapidly the ways of the camp. He knew the injustice
and greediness of the older dogs when meat or fish was thrown out to be
eaten. He came to know that men were more just, children more cruel,
and women more kindly and more likely to toss him a bit of meat or
bone. And after two or three painful adventures with the mothers of
part-grown puppies, he came into the knowledge that it was always good
policy to let such mothers alone, to keep away from them as far as
possible, and to avoid them when he saw them coming.

But the bane of his life was Lip-lip. Larger, older, and stronger,
Lip-lip had selected White Fang for his special object of persecution.
White Fang fought willingly enough, but he was outclassed. His enemy
was too big. Lip-lip became a nightmare to him. Whenever he ventured
away from his mother, the bully was sure to appear, trailing at his
heels, snarling at him, picking upon him, and watchful of an
opportunity, when no man-animal was near, to spring upon him and force
a fight. As Lip-lip invariably won, he enjoyed it hugely. It became his
chief delight in life, as it became White Fang’s chief torment.

But the effect upon White Fang was not to cow him. Though he suffered
most of the damage and was always defeated, his spirit remained
unsubdued. Yet a bad effect was produced. He became malignant and
morose. His temper had been savage by birth, but it became more savage
under this unending persecution. The genial, playful, puppyish side of
him found little expression. He never played and gambolled about with
the other puppies of the camp. Lip-lip would not permit it. The moment
White Fang appeared near them, Lip-lip was upon him, bullying and
hectoring him, or fighting with him until he had driven him away.

The effect of all this was to rob White Fang of much of his puppyhood
and to make him in his comportment older than his age. Denied the
outlet, through play, of his energies, he recoiled upon himself and
developed his mental processes. He became cunning; he had idle time in
which to devote himself to thoughts of trickery. Prevented from
obtaining his share of meat and fish when a general feed was given to
the camp-dogs, he became a clever thief. He had to forage for himself,
and he foraged well, though he was oft-times a plague to the squaws in
consequence. He learned to sneak about camp, to be crafty, to know what
was going on everywhere, to see and to hear everything and to reason
accordingly, and successfully to devise ways and means of avoiding his
implacable persecutor.

It was early in the days of his persecution that he played his first
really big crafty game and got therefrom his first taste of revenge.
As Kiche, when with the wolves, had lured out to destruction dogs from
the camps of men, so White Fang, in manner somewhat similar, lured
Lip-lip into Kiche’s avenging jaws. Retreating before Lip-lip, White
Fang made an indirect flight that led in and out and around the various
tepees of the camp. He was a good runner, swifter than any puppy of his
size, and swifter than Lip-lip. But he did not run his best in this
chase. He barely held his own, one leap ahead of his pursuer.

Lip-lip, excited by the chase and by the persistent nearness of his
victim, forgot caution and locality. When he remembered locality, it
was too late. Dashing at top speed around a tepee, he ran full tilt
into Kiche lying at the end of her stick. He gave one yelp of
consternation, and then her punishing jaws closed upon him. She was
tied, but he could not get away from her easily. She rolled him off his
legs so that he could not run, while she repeatedly ripped and slashed
him with her fangs.

When at last he succeeded in rolling clear of her, he crawled to his
feet, badly dishevelled, hurt both in body and in spirit. His hair was
standing out all over him in tufts where her teeth had mauled. He stood
where he had arisen, opened his mouth, and broke out the long,
heart-broken puppy wail. But even this he was not allowed to complete.
In the middle of it, White Fang, rushing in, sank his teeth into
Lip-lip’s hind leg. There was no fight left in Lip-lip, and he ran away
shamelessly, his victim hot on his heels and worrying him all the way
back to his own tepee. Here the squaws came to his aid, and White Fang,
transformed into a raging demon, was finally driven off only by a
fusillade of stones.

Came the day when Grey Beaver, deciding that the liability of her
running away was past, released Kiche. White Fang was delighted with
his mother’s freedom. He accompanied her joyfully about the camp; and,
so long as he remained close by her side, Lip-lip kept a respectful
distance. White-Fang even bristled up to him and walked stiff-legged,
but Lip-lip ignored the challenge. He was no fool himself, and whatever
vengeance he desired to wreak, he could wait until he caught White Fang
alone.

Later on that day, Kiche and White Fang strayed into the edge of the
woods next to the camp. He had led his mother there, step by step, and
now when she stopped, he tried to inveigle her farther. The stream, the
lair, and the quiet woods were calling to him, and he wanted her to
come. He ran on a few steps, stopped, and looked back. She had not
moved. He whined pleadingly, and scurried playfully in and out of the
underbrush. He ran back to her, licked her face, and ran on again. And
still she did not move. He stopped and regarded her, all of an
intentness and eagerness, physically expressed, that slowly faded out
of him as she turned her head and gazed back at the camp.

There was something calling to him out there in the open. His mother
heard it too. But she heard also that other and louder call, the call
of the fire and of man—the call which has been given alone of all
animals to the wolf to answer, to the wolf and the wild-dog, who are
brothers.

Kiche turned and slowly trotted back toward camp. Stronger than the
physical restraint of the stick was the clutch of the camp upon her.
Unseen and occultly, the gods still gripped with their power and would
not let her go. White Fang sat down in the shadow of a birch and
whimpered softly. There was a strong smell of pine, and subtle wood
fragrances filled the air, reminding him of his old life of freedom
before the days of his bondage. But he was still only a part-grown
puppy, and stronger than the call either of man or of the Wild was the
call of his mother. All the hours of his short life he had depended
upon her. The time was yet to come for independence. So he arose and
trotted forlornly back to camp, pausing once, and twice, to sit down
and whimper and to listen to the call that still sounded in the depths
of the forest.

In the Wild the time of a mother with her young is short; but under the
dominion of man it is sometimes even shorter. Thus it was with White
Fang. Grey Beaver was in the debt of Three Eagles. Three Eagles was
going away on a trip up the Mackenzie to the Great Slave Lake. A strip
of scarlet cloth, a bearskin, twenty cartridges, and Kiche, went to pay
the debt. White Fang saw his mother taken aboard Three Eagles’ canoe,
and tried to follow her. A blow from Three Eagles knocked him backward
to the land. The canoe shoved off. He sprang into the water and swam
after it, deaf to the sharp cries of Grey Beaver to return. Even a
man-animal, a god, White Fang ignored, such was the terror he was in of
losing his mother.

But gods are accustomed to being obeyed, and Grey Beaver wrathfully
launched a canoe in pursuit. When he overtook White Fang, he reached
down and by the nape of the neck lifted him clear of the water. He did
not deposit him at once in the bottom of the canoe. Holding him
suspended with one hand, with the other hand he proceeded to give him a
beating. And it \emph{was} a beating. His hand was heavy. Every blow was
shrewd to hurt; and he delivered a multitude of blows.

Impelled by the blows that rained upon him, now from this side, now
from that, White Fang swung back and forth like an erratic and jerky
pendulum. Varying were the emotions that surged through him. At first,
he had known surprise. Then came a momentary fear, when he yelped
several times to the impact of the hand. But this was quickly followed
by anger. His free nature asserted itself, and he showed his teeth and
snarled fearlessly in the face of the wrathful god. This but served to
make the god more wrathful. The blows came faster, heavier, more shrewd
to hurt.

Grey Beaver continued to beat, White Fang continued to snarl. But this
could not last for ever. One or the other must give over, and that one
was White Fang. Fear surged through him again. For the first time he
was being really man-handled. The occasional blows of sticks and stones
he had previously experienced were as caresses compared with this. He
broke down and began to cry and yelp. For a time each blow brought a
yelp from him; but fear passed into terror, until finally his yelps
were voiced in unbroken succession, unconnected with the rhythm of the
punishment.

At last Grey Beaver withheld his hand. White Fang, hanging limply,
continued to cry. This seemed to satisfy his master, who flung him down
roughly in the bottom of the canoe. In the meantime the canoe had
drifted down the stream. Grey Beaver picked up the paddle. White Fang
was in his way. He spurned him savagely with his foot. In that moment
White Fang’s free nature flashed forth again, and he sank his teeth
into the moccasined foot.

The beating that had gone before was as nothing compared with the
beating he now received. Grey Beaver’s wrath was terrible; likewise was
White Fang’s fright. Not only the hand, but the hard wooden paddle was
used upon him; and he was bruised and sore in all his small body when
he was again flung down in the canoe. Again, and this time with
purpose, did Grey Beaver kick him. White Fang did not repeat his attack
on the foot. He had learned another lesson of his bondage. Never, no
matter what the circumstance, must he dare to bite the god who was lord
and master over him; the body of the lord and master was sacred, not to
be defiled by the teeth of such as he. That was evidently the crime of
crimes, the one offence there was no condoning nor overlooking.

When the canoe touched the shore, White Fang lay whimpering and
motionless, waiting the will of Grey Beaver. It was Grey Beaver’s will
that he should go ashore, for ashore he was flung, striking heavily on
his side and hurting his bruises afresh. He crawled tremblingly to his
feet and stood whimpering. Lip-lip, who had watched the whole
proceeding from the bank, now rushed upon him, knocking him over and
sinking his teeth into him. White Fang was too helpless to defend
himself, and it would have gone hard with him had not Grey Beaver’s
foot shot out, lifting Lip-lip into the air with its violence so that
he smashed down to earth a dozen feet away. This was the man-animal’s
justice; and even then, in his own pitiable plight, White Fang
experienced a little grateful thrill. At Grey Beaver’s heels he limped
obediently through the village to the tepee. And so it came that White
Fang learned that the right to punish was something the gods reserved
for themselves and denied to the lesser creatures under them.

That night, when all was still, White Fang remembered his mother and
sorrowed for her. He sorrowed too loudly and woke up Grey Beaver, who
beat him. After that he mourned gently when the gods were around. But
sometimes, straying off to the edge of the woods by himself, he gave
vent to his grief, and cried it out with loud whimperings and wailings.

It was during this period that he might have harkened to the memories
of the lair and the stream and run back to the Wild. But the memory of
his mother held him. As the hunting man-animals went out and came back,
so she would come back to the village some time. So he remained in his
bondage waiting for her.

But it was not altogether an unhappy bondage. There was much to
interest him. Something was always happening. There was no end to the
strange things these gods did, and he was always curious to see.
Besides, he was learning how to get along with Grey Beaver. Obedience,
rigid, undeviating obedience, was what was exacted of him; and in
return he escaped beatings and his existence was tolerated.

Nay, Grey Beaver himself sometimes tossed him a piece of meat, and
defended him against the other dogs in the eating of it. And such a
piece of meat was of value. It was worth more, in some strange way,
then a dozen pieces of meat from the hand of a squaw. Grey Beaver never
petted nor caressed. Perhaps it was the weight of his hand, perhaps his
justice, perhaps the sheer power of him, and perhaps it was all these
things that influenced White Fang; for a certain tie of attachment was
forming between him and his surly lord.

Insidiously, and by remote ways, as well as by the power of stick and
stone and clout of hand, were the shackles of White Fang’s bondage
being riveted upon him. The qualities in his kind that in the beginning
made it possible for them to come in to the fires of men, were
qualities capable of development. They were developing in him, and the
camp-life, replete with misery as it was, was secretly endearing itself
to him all the time. But White Fang was unaware of it. He knew only
grief for the loss of Kiche, hope for her return, and a hungry yearning
for the free life that had been his.

\chapter{The Outcast}

Lip-lip continued so to darken his days that White Fang became wickeder
and more ferocious than it was his natural right to be. Savageness was
a part of his make-up, but the savageness thus developed exceeded his
make-up. He acquired a reputation for wickedness amongst the
man-animals themselves. Wherever there was trouble and uproar in camp,
fighting and squabbling or the outcry of a squaw over a bit of stolen
meat, they were sure to find White Fang mixed up in it and usually at
the bottom of it. They did not bother to look after the causes of his
conduct. They saw only the effects, and the effects were bad. He was a
sneak and a thief, a mischief-maker, a fomenter of trouble; and irate
squaws told him to his face, the while he eyed them alert and ready to
dodge any quick-flung missile, that he was a wolf and worthless and
bound to come to an evil end.

He found himself an outcast in the midst of the populous camp. All the
young dogs followed Lip-lip’s lead. There was a difference between
White Fang and them. Perhaps they sensed his wild-wood breed, and
instinctively felt for him the enmity that the domestic dog feels for
the wolf. But be that as it may, they joined with Lip-lip in the
persecution. And, once declared against him, they found good reason to
continue declared against him. One and all, from time to time, they
felt his teeth; and to his credit, he gave more than he received. Many
of them he could whip in single fight; but single fight was denied him.
The beginning of such a fight was a signal for all the young dogs in
camp to come running and pitch upon him.

Out of this pack-persecution he learned two important things: how to
take care of himself in a mass-fight against him—and how, on a single
dog, to inflict the greatest amount of damage in the briefest space of
time. To keep one’s feet in the midst of the hostile mass meant life,
and this he learnt well. He became cat-like in his ability to stay on
his feet. Even grown dogs might hurtle him backward or sideways with
the impact of their heavy bodies; and backward or sideways he would go,
in the air or sliding on the ground, but always with his legs under him
and his feet downward to the mother earth.

When dogs fight, there are usually preliminaries to the actual
combat—snarlings and bristlings and stiff-legged struttings. But White
Fang learned to omit these preliminaries. Delay meant the coming
against him of all the young dogs. He must do his work quickly and get
away. So he learnt to give no warning of his intention. He rushed in
and snapped and slashed on the instant, without notice, before his foe
could prepare to meet him. Thus he learned how to inflict quick and
severe damage. Also he learned the value of surprise. A dog, taken off
its guard, its shoulder slashed open or its ear ripped in ribbons
before it knew what was happening, was a dog half whipped.

Furthermore, it was remarkably easy to overthrow a dog taken by
surprise; while a dog, thus overthrown, invariably exposed for a moment
the soft underside of its neck—the vulnerable point at which to strike
for its life. White Fang knew this point. It was a knowledge bequeathed
to him directly from the hunting generation of wolves. So it was that
White Fang’s method when he took the offensive, was: first to find a
young dog alone; second, to surprise it and knock it off its feet; and
third, to drive in with his teeth at the soft throat.

Being but partly grown his jaws had not yet become large enough nor
strong enough to make his throat-attack deadly; but many a young dog
went around camp with a lacerated throat in token of White Fang’s
intention. And one day, catching one of his enemies alone on the edge
of the woods, he managed, by repeatedly overthrowing him and attacking
the throat, to cut the great vein and let out the life. There was a
great row that night. He had been observed, the news had been carried
to the dead dog’s master, the squaws remembered all the instances of
stolen meat, and Grey Beaver was beset by many angry voices. But he
resolutely held the door of his tepee, inside which he had placed the
culprit, and refused to permit the vengeance for which his tribespeople
clamoured.

White Fang became hated by man and dog. During this period of his
development he never knew a moment’s security. The tooth of every dog
was against him, the hand of every man. He was greeted with snarls by
his kind, with curses and stones by his gods. He lived tensely. He was
always keyed up, alert for attack, wary of being attacked, with an eye
for sudden and unexpected missiles, prepared to act precipitately and
coolly, to leap in with a flash of teeth, or to leap away with a
menacing snarl.

As for snarling he could snarl more terribly than any dog, young or
old, in camp. The intent of the snarl is to warn or frighten, and
judgment is required to know when it should be used. White Fang knew
how to make it and when to make it. Into his snarl he incorporated all
that was vicious, malignant, and horrible. With nose serrulated by
continuous spasms, hair bristling in recurrent waves, tongue whipping
out like a red snake and whipping back again, ears flattened down, eyes
gleaming hatred, lips wrinkled back, and fangs exposed and dripping, he
could compel a pause on the part of almost any assailant. A temporary
pause, when taken off his guard, gave him the vital moment in which to
think and determine his action. But often a pause so gained lengthened
out until it evolved into a complete cessation from the attack. And
before more than one of the grown dogs White Fang’s snarl enabled him
to beat an honourable retreat.

An outcast himself from the pack of the part-grown dogs, his sanguinary
methods and remarkable efficiency made the pack pay for its persecution
of him. Not permitted himself to run with the pack, the curious state
of affairs obtained that no member of the pack could run outside the
pack. White Fang would not permit it. What of his bushwhacking and
waylaying tactics, the young dogs were afraid to run by themselves.
With the exception of Lip-lip, they were compelled to hunch together
for mutual protection against the terrible enemy they had made. A puppy
alone by the river bank meant a puppy dead or a puppy that aroused the
camp with its shrill pain and terror as it fled back from the wolf-cub
that had waylaid it.

But White Fang’s reprisals did not cease, even when the young dogs had
learned thoroughly that they must stay together. He attacked them when
he caught them alone, and they attacked him when they were bunched. The
sight of him was sufficient to start them rushing after him, at which
times his swiftness usually carried him into safety. But woe the dog
that outran his fellows in such pursuit! White Fang had learned to turn
suddenly upon the pursuer that was ahead of the pack and thoroughly to
rip him up before the pack could arrive. This occurred with great
frequency, for, once in full cry, the dogs were prone to forget
themselves in the excitement of the chase, while White Fang never
forgot himself. Stealing backward glances as he ran, he was always
ready to whirl around and down the overzealous pursuer that outran his
fellows.

Young dogs are bound to play, and out of the exigencies of the
situation they realised their play in this mimic warfare. Thus it was
that the hunt of White Fang became their chief game—a deadly game,
withal, and at all times a serious game. He, on the other hand, being
the fastest-footed, was unafraid to venture anywhere. During the period
that he waited vainly for his mother to come back, he led the pack many
a wild chase through the adjacent woods. But the pack invariably lost
him. Its noise and outcry warned him of its presence, while he ran
alone, velvet-footed, silently, a moving shadow among the trees after
the manner of his father and mother before him. Further he was more
directly connected with the Wild than they; and he knew more of its
secrets and stratagems. A favourite trick of his was to lose his trail
in running water and then lie quietly in a near-by thicket while their
baffled cries arose around him.

Hated by his kind and by mankind, indomitable, perpetually warred upon
and himself waging perpetual war, his development was rapid and
one-sided. This was no soil for kindliness and affection to blossom in.
Of such things he had not the faintest glimmering. The code he learned
was to obey the strong and to oppress the weak. Grey Beaver was a god,
and strong. Therefore White Fang obeyed him. But the dog younger or
smaller than himself was weak, a thing to be destroyed. His development
was in the direction of power. In order to face the constant danger of
hurt and even of destruction, his predatory and protective faculties
were unduly developed. He became quicker of movement than the other
dogs, swifter of foot, craftier, deadlier, more lithe, more lean with
ironlike muscle and sinew, more enduring, more cruel, more ferocious,
and more intelligent. He had to become all these things, else he would
not have held his own nor survive the hostile environment in which he
found himself.

\chapter{The Trail of the Gods}

In the fall of the year, when the days were shortening and the bite of
the frost was coming into the air, White Fang got his chance for
liberty. For several days there had been a great hubbub in the village.
The summer camp was being dismantled, and the tribe, bag and baggage,
was preparing to go off to the fall hunting. White Fang watched it all
with eager eyes, and when the tepees began to come down and the canoes
were loading at the bank, he understood. Already the canoes were
departing, and some had disappeared down the river.

Quite deliberately he determined to stay behind. He waited his
opportunity to slink out of camp to the woods. Here, in the running
stream where ice was beginning to form, he hid his trail. Then he
crawled into the heart of a dense thicket and waited. The time passed
by, and he slept intermittently for hours. Then he was aroused by Grey
Beaver’s voice calling him by name. There were other voices. White Fang
could hear Grey Beaver’s squaw taking part in the search, and Mit-sah,
who was Grey Beaver’s son.

White Fang trembled with fear, and though the impulse came to crawl out
of his hiding-place, he resisted it. After a time the voices died away,
and some time after that he crept out to enjoy the success of his
undertaking. Darkness was coming on, and for a while he played about
among the trees, pleasuring in his freedom. Then, and quite suddenly,
he became aware of loneliness. He sat down to consider, listening to
the silence of the forest and perturbed by it. That nothing moved nor
sounded, seemed ominous. He felt the lurking of danger, unseen and
unguessed. He was suspicious of the looming bulks of the trees and of
the dark shadows that might conceal all manner of perilous things.

Then it was cold. Here was no warm side of a tepee against which to
snuggle. The frost was in his feet, and he kept lifting first one
fore-foot and then the other. He curved his bushy tail around to cover
them, and at the same time he saw a vision. There was nothing strange
about it. Upon his inward sight was impressed a succession of
memory-pictures. He saw the camp again, the tepees, and the blaze of
the fires. He heard the shrill voices of the women, the gruff basses of
the men, and the snarling of the dogs. He was hungry, and he remembered
pieces of meat and fish that had been thrown him. Here was no meat,
nothing but a threatening and inedible silence.

His bondage had softened him. Irresponsibility had weakened him. He had
forgotten how to shift for himself. The night yawned about him. His
senses, accustomed to the hum and bustle of the camp, used to the
continuous impact of sights and sounds, were now left idle. There was
nothing to do, nothing to see nor hear. They strained to catch some
interruption of the silence and immobility of nature. They were
appalled by inaction and by the feel of something terrible impending.

He gave a great start of fright. A colossal and formless something was
rushing across the field of his vision. It was a tree-shadow flung by
the moon, from whose face the clouds had been brushed away. Reassured,
he whimpered softly; then he suppressed the whimper for fear that it
might attract the attention of the lurking dangers.

A tree, contracting in the cool of the night, made a loud noise. It was
directly above him. He yelped in his fright. A panic seized him, and he
ran madly toward the village. He knew an overpowering desire for the
protection and companionship of man. In his nostrils was the smell of
the camp-smoke. In his ears the camp-sounds and cries were ringing
loud. He passed out of the forest and into the moonlit open where were
no shadows nor darknesses. But no village greeted his eyes. He had
forgotten. The village had gone away.

His wild flight ceased abruptly. There was no place to which to flee.
He slunk forlornly through the deserted camp, smelling the
rubbish-heaps and the discarded rags and tags of the gods. He would
have been glad for the rattle of stones about him, flung by an angry
squaw, glad for the hand of Grey Beaver descending upon him in wrath;
while he would have welcomed with delight Lip-lip and the whole
snarling, cowardly pack.

He came to where Grey Beaver’s tepee had stood. In the centre of the
space it had occupied, he sat down. He pointed his nose at the moon.
His throat was afflicted by rigid spasms, his mouth opened, and in a
heart-broken cry bubbled up his loneliness and fear, his grief for
Kiche, all his past sorrows and miseries as well as his apprehension of
sufferings and dangers to come. It was the long wolf-howl,
full-throated and mournful, the first howl he had ever uttered.

The coming of daylight dispelled his fears but increased his
loneliness. The naked earth, which so shortly before had been so
populous; thrust his loneliness more forcibly upon him. It did not take
him long to make up his mind. He plunged into the forest and followed
the river bank down the stream. All day he ran. He did not rest. He
seemed made to run on for ever. His iron-like body ignored fatigue. And
even after fatigue came, his heritage of endurance braced him to
endless endeavour and enabled him to drive his complaining body onward.

Where the river swung in against precipitous bluffs, he climbed the
high mountains behind. Rivers and streams that entered the main river
he forded or swam. Often he took to the rim-ice that was beginning to
form, and more than once he crashed through and struggled for life in
the icy current. Always he was on the lookout for the trail of the gods
where it might leave the river and proceed inland.

White Fang was intelligent beyond the average of his kind; yet his
mental vision was not wide enough to embrace the other bank of the
Mackenzie. What if the trail of the gods led out on that side? It never
entered his head. Later on, when he had travelled more and grown older
and wiser and come to know more of trails and rivers, it might be that
he could grasp and apprehend such a possibility. But that mental power
was yet in the future. Just now he ran blindly, his own bank of the
Mackenzie alone entering into his calculations.

All night he ran, blundering in the darkness into mishaps and obstacles
that delayed but did not daunt. By the middle of the second day he had
been running continuously for thirty hours, and the iron of his flesh
was giving out. It was the endurance of his mind that kept him going.
He had not eaten in forty hours, and he was weak with hunger. The
repeated drenchings in the icy water had likewise had their effect on
him. His handsome coat was draggled. The broad pads of his feet were
bruised and bleeding. He had begun to limp, and this limp increased
with the hours. To make it worse, the light of the sky was obscured and
snow began to fall—a raw, moist, melting, clinging snow, slippery under
foot, that hid from him the landscape he traversed, and that covered
over the inequalities of the ground so that the way of his feet was
more difficult and painful.

Grey Beaver had intended camping that night on the far bank of the
Mackenzie, for it was in that direction that the hunting lay. But on
the near bank, shortly before dark, a moose coming down to drink, had
been espied by Kloo-kooch, who was Grey Beaver’s squaw. Now, had not
the moose come down to drink, had not Mit-sah been steering out of the
course because of the snow, had not Kloo-kooch sighted the moose, and
had not Grey Beaver killed it with a lucky shot from his rifle, all
subsequent things would have happened differently. Grey Beaver would
not have camped on the near side of the Mackenzie, and White Fang would
have passed by and gone on, either to die or to find his way to his
wild brothers and become one of them—a wolf to the end of his days.

Night had fallen. The snow was flying more thickly, and White Fang,
whimpering softly to himself as he stumbled and limped along, came upon
a fresh trail in the snow. So fresh was it that he knew it immediately
for what it was. Whining with eagerness, he followed back from the
river bank and in among the trees. The camp-sounds came to his ears. He
saw the blaze of the fire, Kloo-kooch cooking, and Grey Beaver
squatting on his hams and mumbling a chunk of raw tallow. There was
fresh meat in camp!

White Fang expected a beating. He crouched and bristled a little at the
thought of it. Then he went forward again. He feared and disliked the
beating he knew to be waiting for him. But he knew, further, that the
comfort of the fire would be his, the protection of the gods, the
companionship of the dogs—the last, a companionship of enmity, but none
the less a companionship and satisfying to his gregarious needs.

He came cringing and crawling into the firelight. Grey Beaver saw him,
and stopped munching the tallow. White Fang crawled slowly, cringing
and grovelling in the abjectness of his abasement and submission. He
crawled straight toward Grey Beaver, every inch of his progress
becoming slower and more painful. At last he lay at the master’s feet,
into whose possession he now surrendered himself, voluntarily, body and
soul. Of his own choice, he came in to sit by man’s fire and to be
ruled by him. White Fang trembled, waiting for the punishment to fall
upon him. There was a movement of the hand above him. He cringed
involuntarily under the expected blow. It did not fall. He stole a
glance upward. Grey Beaver was breaking the lump of tallow in half!
Grey Beaver was offering him one piece of the tallow! Very gently and
somewhat suspiciously, he first smelled the tallow and then proceeded
to eat it. Grey Beaver ordered meat to be brought to him, and guarded
him from the other dogs while he ate. After that, grateful and content,
White Fang lay at Grey Beaver’s feet, gazing at the fire that warmed
him, blinking and dozing, secure in the knowledge that the morrow would
find him, not wandering forlorn through bleak forest-stretches, but in
the camp of the man-animals, with the gods to whom he had given himself
and upon whom he was now dependent.

\chapter{The Covenant}

When December was well along, Grey Beaver went on a journey up the
Mackenzie. Mit-sah and Kloo-kooch went with him. One sled he drove
himself, drawn by dogs he had traded for or borrowed. A second and
smaller sled was driven by Mit-sah, and to this was harnessed a team of
puppies. It was more of a toy affair than anything else, yet it was the
delight of Mit-sah, who felt that he was beginning to do a man’s work
in the world. Also, he was learning to drive dogs and to train dogs;
while the puppies themselves were being broken in to the harness.
Furthermore, the sled was of some service, for it carried nearly two
hundred pounds of outfit and food.

White Fang had seen the camp-dogs toiling in the harness, so that he
did not resent overmuch the first placing of the harness upon himself.
About his neck was put a moss-stuffed collar, which was connected by
two pulling-traces to a strap that passed around his chest and over his
back. It was to this that was fastened the long rope by which he pulled
at the sled.

There were seven puppies in the team. The others had been born earlier
in the year and were nine and ten months old, while White Fang was only
eight months old. Each dog was fastened to the sled by a single rope.
No two ropes were of the same length, while the difference in length
between any two ropes was at least that of a dog’s body. Every rope was
brought to a ring at the front end of the sled. The sled itself was
without runners, being a birch-bark toboggan, with upturned forward end
to keep it from ploughing under the snow. This construction enabled the
weight of the sled and load to be distributed over the largest
snow-surface; for the snow was crystal-powder and very soft. Observing
the same principle of widest distribution of weight, the dogs at the
ends of their ropes radiated fan-fashion from the nose of the sled, so
that no dog trod in another’s footsteps.

There was, furthermore, another virtue in the fan-formation. The ropes
of varying length prevented the dogs attacking from the rear those that
ran in front of them. For a dog to attack another, it would have to
turn upon one at a shorter rope. In which case it would find itself
face to face with the dog attacked, and also it would find itself
facing the whip of the driver. But the most peculiar virtue of all lay
in the fact that the dog that strove to attack one in front of him must
pull the sled faster, and that the faster the sled travelled, the
faster could the dog attacked run away. Thus, the dog behind could
never catch up with the one in front. The faster he ran, the faster ran
the one he was after, and the faster ran all the dogs. Incidentally,
the sled went faster, and thus, by cunning indirection, did man
increase his mastery over the beasts.

Mit-sah resembled his father, much of whose grey wisdom he possessed.
In the past he had observed Lip-lip’s persecution of White Fang; but at
that time Lip-lip was another man’s dog, and Mit-sah had never dared
more than to shy an occasional stone at him. But now Lip-lip was his
dog, and he proceeded to wreak his vengeance on him by putting him at
the end of the longest rope. This made Lip-lip the leader, and was
apparently an honour! but in reality it took away from him all honour,
and instead of being bully and master of the pack, he now found himself
hated and persecuted by the pack.

Because he ran at the end of the longest rope, the dogs had always the
view of him running away before them. All that they saw of him was his
bushy tail and fleeing hind legs—a view far less ferocious and
intimidating than his bristling mane and gleaming fangs. Also, dogs
being so constituted in their mental ways, the sight of him running
away gave desire to run after him and a feeling that he ran away from
them.

The moment the sled started, the team took after Lip-lip in a chase
that extended throughout the day. At first he had been prone to turn
upon his pursuers, jealous of his dignity and wrathful; but at such
times Mit-sah would throw the stinging lash of the thirty-foot
cariboo-gut whip into his face and compel him to turn tail and run on.
Lip-lip might face the pack, but he could not face that whip, and all
that was left him to do was to keep his long rope taut and his flanks
ahead of the teeth of his mates.

But a still greater cunning lurked in the recesses of the Indian mind.
To give point to unending pursuit of the leader, Mit-sah favoured him
over the other dogs. These favours aroused in them jealousy and hatred.
In their presence Mit-sah would give him meat and would give it to him
only. This was maddening to them. They would rage around just outside
the throwing-distance of the whip, while Lip-lip devoured the meat and
Mit-sah protected him. And when there was no meat to give, Mit-sah
would keep the team at a distance and make believe to give meat to
Lip-lip.

White Fang took kindly to the work. He had travelled a greater distance
than the other dogs in the yielding of himself to the rule of the gods,
and he had learned more thoroughly the futility of opposing their will.
In addition, the persecution he had suffered from the pack had made the
pack less to him in the scheme of things, and man more. He had not
learned to be dependent on his kind for companionship. Besides, Kiche
was well-nigh forgotten; and the chief outlet of expression that
remained to him was in the allegiance he tendered the gods he had
accepted as masters. So he worked hard, learned discipline, and was
obedient. Faithfulness and willingness characterised his toil. These
are essential traits of the wolf and the wild-dog when they have become
domesticated, and these traits White Fang possessed in unusual measure.

A companionship did exist between White Fang and the other dogs, but it
was one of warfare and enmity. He had never learned to play with them.
He knew only how to fight, and fight with them he did, returning to
them a hundred-fold the snaps and slashes they had given him in the
days when Lip-lip was leader of the pack. But Lip-lip was no longer
leader—except when he fled away before his mates at the end of his
rope, the sled bounding along behind. In camp he kept close to Mit-sah
or Grey Beaver or Kloo-kooch. He did not dare venture away from the
gods, for now the fangs of all dogs were against him, and he tasted to
the dregs the persecution that had been White Fang’s.

With the overthrow of Lip-lip, White Fang could have become leader of
the pack. But he was too morose and solitary for that. He merely
thrashed his team-mates. Otherwise he ignored them. They got out of his
way when he came along; nor did the boldest of them ever dare to rob
him of his meat. On the contrary, they devoured their own meat
hurriedly, for fear that he would take it away from them. White Fang
knew the law well: \emph{to oppress the weak and obey the strong}. He ate
his share of meat as rapidly as he could. And then woe the dog that had
not yet finished! A snarl and a flash of fangs, and that dog would wail
his indignation to the uncomforting stars while White Fang finished his
portion for him.

Every little while, however, one dog or another would flame up in
revolt and be promptly subdued. Thus White Fang was kept in training.
He was jealous of the isolation in which he kept himself in the midst
of the pack, and he fought often to maintain it. But such fights were
of brief duration. He was too quick for the others. They were slashed
open and bleeding before they knew what had happened, were whipped
almost before they had begun to fight.

As rigid as the sled-discipline of the gods, was the discipline
maintained by White Fang amongst his fellows. He never allowed them any
latitude. He compelled them to an unremitting respect for him. They
might do as they pleased amongst themselves. That was no concern of
his. But it \emph{was} his concern that they leave him alone in his
isolation, get out of his way when he elected to walk among them, and
at all times acknowledge his mastery over them. A hint of
stiff-leggedness on their part, a lifted lip or a bristle of hair, and
he would be upon them, merciless and cruel, swiftly convincing them of
the error of their way.

He was a monstrous tyrant. His mastery was rigid as steel. He oppressed
the weak with a vengeance. Not for nothing had he been exposed to the
pitiless struggles for life in the day of his cubhood, when his mother
and he, alone and unaided, held their own and survived in the ferocious
environment of the Wild. And not for nothing had he learned to walk
softly when superior strength went by. He oppressed the weak, but he
respected the strong. And in the course of the long journey with Grey
Beaver he walked softly indeed amongst the full-grown dogs in the camps
of the strange man-animals they encountered.

The months passed by. Still continued the journey of Grey Beaver. White
Fang’s strength was developed by the long hours on trail and the steady
toil at the sled; and it would have seemed that his mental development
was well-nigh complete. He had come to know quite thoroughly the world
in which he lived. His outlook was bleak and materialistic. The world
as he saw it was a fierce and brutal world, a world without warmth, a
world in which caresses and affection and the bright sweetnesses of the
spirit did not exist.

He had no affection for Grey Beaver. True, he was a god, but a most
savage god. White Fang was glad to acknowledge his lordship, but it was
a lordship based upon superior intelligence and brute strength. There
was something in the fibre of White Fang’s being that made his lordship
a thing to be desired, else he would not have come back from the Wild
when he did to tender his allegiance. There were deeps in his nature
which had never been sounded. A kind word, a caressing touch of the
hand, on the part of Grey Beaver, might have sounded these deeps; but
Grey Beaver did not caress, nor speak kind words. It was not his way.
His primacy was savage, and savagely he ruled, administering justice
with a club, punishing transgression with the pain of a blow, and
rewarding merit, not by kindness, but by withholding a blow.

So White Fang knew nothing of the heaven a man’s hand might contain for
him. Besides, he did not like the hands of the man-animals. He was
suspicious of them. It was true that they sometimes gave meat, but more
often they gave hurt. Hands were things to keep away from. They hurled
stones, wielded sticks and clubs and whips, administered slaps and
clouts, and, when they touched him, were cunning to hurt with pinch and
twist and wrench. In strange villages he had encountered the hands of
the children and learned that they were cruel to hurt. Also, he had
once nearly had an eye poked out by a toddling papoose. From these
experiences he became suspicious of all children. He could not tolerate
them. When they came near with their ominous hands, he got up.

It was in a village at the Great Slave Lake, that, in the course of
resenting the evil of the hands of the man-animals, he came to modify
the law that he had learned from Grey Beaver: namely, that the
unpardonable crime was to bite one of the gods. In this village, after
the custom of all dogs in all villages, White Fang went foraging, for
food. A boy was chopping frozen moose-meat with an axe, and the chips
were flying in the snow. White Fang, sliding by in quest of meat,
stopped and began to eat the chips. He observed the boy lay down the
axe and take up a stout club. White Fang sprang clear, just in time to
escape the descending blow. The boy pursued him, and he, a stranger in
the village, fled between two tepees to find himself cornered against a
high earth bank.

There was no escape for White Fang. The only way out was between the
two tepees, and this the boy guarded. Holding his club prepared to
strike, he drew in on his cornered quarry. White Fang was furious. He
faced the boy, bristling and snarling, his sense of justice outraged.
He knew the law of forage. All the wastage of meat, such as the frozen
chips, belonged to the dog that found it. He had done no wrong, broken
no law, yet here was this boy preparing to give him a beating. White
Fang scarcely knew what happened. He did it in a surge of rage. And he
did it so quickly that the boy did not know either. All the boy knew
was that he had in some unaccountable way been overturned into the
snow, and that his club-hand had been ripped wide open by White Fang’s
teeth.

But White Fang knew that he had broken the law of the gods. He had
driven his teeth into the sacred flesh of one of them, and could expect
nothing but a most terrible punishment. He fled away to Grey Beaver,
behind whose protecting legs he crouched when the bitten boy and the
boy’s family came, demanding vengeance. But they went away with
vengeance unsatisfied. Grey Beaver defended White Fang. So did Mit-sah
and Kloo-kooch. White Fang, listening to the wordy war and watching the
angry gestures, knew that his act was justified. And so it came that he
learned there were gods and gods. There were his gods, and there were
other gods, and between them there was a difference. Justice or
injustice, it was all the same, he must take all things from the hands
of his own gods. But he was not compelled to take injustice from the
other gods. It was his privilege to resent it with his teeth. And this
also was a law of the gods.

Before the day was out, White Fang was to learn more about this law.
Mit-sah, alone, gathering firewood in the forest, encountered the boy
that had been bitten. With him were other boys. Hot words passed. Then
all the boys attacked Mit-sah. It was going hard with him. Blows were
raining upon him from all sides. White Fang looked on at first. This
was an affair of the gods, and no concern of his. Then he realised that
this was Mit-sah, one of his own particular gods, who was being
maltreated. It was no reasoned impulse that made White Fang do what he
then did. A mad rush of anger sent him leaping in amongst the
combatants. Five minutes later the landscape was covered with fleeing
boys, many of whom dripped blood upon the snow in token that White
Fang’s teeth had not been idle. When Mit-sah told the story in camp,
Grey Beaver ordered meat to be given to White Fang. He ordered much
meat to be given, and White Fang, gorged and sleepy by the fire, knew
that the law had received its verification.

It was in line with these experiences that White Fang came to learn the
law of property and the duty of the defence of property. From the
protection of his god’s body to the protection of his god’s possessions
was a step, and this step he made. What was his god’s was to be
defended against all the world—even to the extent of biting other gods.
Not only was such an act sacrilegious in its nature, but it was fraught
with peril. The gods were all-powerful, and a dog was no match against
them; yet White Fang learned to face them, fiercely belligerent and
unafraid. Duty rose above fear, and thieving gods learned to leave Grey
Beaver’s property alone.

One thing, in this connection, White Fang quickly learnt, and that was
that a thieving god was usually a cowardly god and prone to run away at
the sounding of the alarm. Also, he learned that but brief time elapsed
between his sounding of the alarm and Grey Beaver coming to his aid. He
came to know that it was not fear of him that drove the thief away, but
fear of Grey Beaver. White Fang did not give the alarm by barking. He
never barked. His method was to drive straight at the intruder, and to
sink his teeth in if he could. Because he was morose and solitary,
having nothing to do with the other dogs, he was unusually fitted to
guard his master’s property; and in this he was encouraged and trained
by Grey Beaver. One result of this was to make White Fang more
ferocious and indomitable, and more solitary.

The months went by, binding stronger and stronger the covenant between
dog and man. This was the ancient covenant that the first wolf that
came in from the Wild entered into with man. And, like all succeeding
wolves and wild dogs that had done likewise, White Fang worked the
covenant out for himself. The terms were simple. For the possession of
a flesh-and-blood god, he exchanged his own liberty. Food and fire,
protection and companionship, were some of the things he received from
the god. In return, he guarded the god’s property, defended his body,
worked for him, and obeyed him.

The possession of a god implies service. White Fang’s was a service of
duty and awe, but not of love. He did not know what love was. He had no
experience of love. Kiche was a remote memory. Besides, not only had he
abandoned the Wild and his kind when he gave himself up to man, but the
terms of the covenant were such that if ever he met Kiche again he
would not desert his god to go with her. His allegiance to man seemed
somehow a law of his being greater than the love of liberty, of kind
and kin.

\chapter{The Famine}

The spring of the year was at hand when Grey Beaver finished his long
journey. It was April, and White Fang was a year old when he pulled
into the home villages and was loosed from the harness by Mit-sah.
Though a long way from his full growth, White Fang, next to Lip-lip,
was the largest yearling in the village. Both from his father, the
wolf, and from Kiche, he had inherited stature and strength, and
already he was measuring up alongside the full-grown dogs. But he had
not yet grown compact. His body was slender and rangy, and his strength
more stringy than massive, His coat was the true wolf-grey, and to all
appearances he was true wolf himself. The quarter-strain of dog he had
inherited from Kiche had left no mark on him physically, though it had
played its part in his mental make-up.

He wandered through the village, recognising with staid satisfaction
the various gods he had known before the long journey. Then there were
the dogs, puppies growing up like himself, and grown dogs that did not
look so large and formidable as the memory pictures he retained of
them. Also, he stood less in fear of them than formerly, stalking among
them with a certain careless ease that was as new to him as it was
enjoyable.

There was Baseek, a grizzled old fellow that in his younger days had
but to uncover his fangs to send White Fang cringing and crouching to
the right about. From him White Fang had learned much of his own
insignificance; and from him he was now to learn much of the change and
development that had taken place in himself. While Baseek had been
growing weaker with age, White Fang had been growing stronger with
youth.

It was at the cutting-up of a moose, fresh-killed, that White Fang
learned of the changed relations in which he stood to the dog-world. He
had got for himself a hoof and part of the shin-bone, to which quite a
bit of meat was attached. Withdrawn from the immediate scramble of the
other dogs—in fact out of sight behind a thicket—he was devouring his
prize, when Baseek rushed in upon him. Before he knew what he was
doing, he had slashed the intruder twice and sprung clear. Baseek was
surprised by the other’s temerity and swiftness of attack. He stood,
gazing stupidly across at White Fang, the raw, red shin-bone between
them.

Baseek was old, and already he had come to know the increasing valour
of the dogs it had been his wont to bully. Bitter experiences these,
which, perforce, he swallowed, calling upon all his wisdom to cope with
them. In the old days he would have sprung upon White Fang in a fury of
righteous wrath. But now his waning powers would not permit such a
course. He bristled fiercely and looked ominously across the shin-bone
at White Fang. And White Fang, resurrecting quite a deal of the old
awe, seemed to wilt and to shrink in upon himself and grow small, as he
cast about in his mind for a way to beat a retreat not too inglorious.

And right here Baseek erred. Had he contented himself with looking
fierce and ominous, all would have been well. White Fang, on the verge
of retreat, would have retreated, leaving the meat to him. But Baseek
did not wait. He considered the victory already his and stepped forward
to the meat. As he bent his head carelessly to smell it, White Fang
bristled slightly. Even then it was not too late for Baseek to retrieve
the situation. Had he merely stood over the meat, head up and
glowering, White Fang would ultimately have slunk away. But the fresh
meat was strong in Baseek’s nostrils, and greed urged him to take a
bite of it.

This was too much for White Fang. Fresh upon his months of mastery over
his own team-mates, it was beyond his self-control to stand idly by
while another devoured the meat that belonged to him. He struck, after
his custom, without warning. With the first slash, Baseek’s right ear
was ripped into ribbons. He was astounded at the suddenness of it. But
more things, and most grievous ones, were happening with equal
suddenness. He was knocked off his feet. His throat was bitten. While
he was struggling to his feet the young dog sank teeth twice into his
shoulder. The swiftness of it was bewildering. He made a futile rush at
White Fang, clipping the empty air with an outraged snap. The next
moment his nose was laid open, and he was staggering backward away from
the meat.

The situation was now reversed. White Fang stood over the shin-bone,
bristling and menacing, while Baseek stood a little way off, preparing
to retreat. He dared not risk a fight with this young lightning-flash,
and again he knew, and more bitterly, the enfeeblement of oncoming age.
His attempt to maintain his dignity was heroic. Calmly turning his back
upon young dog and shin-bone, as though both were beneath his notice
and unworthy of his consideration, he stalked grandly away. Nor, until
well out of sight, did he stop to lick his bleeding wounds.

The effect on White Fang was to give him a greater faith in himself,
and a greater pride. He walked less softly among the grown dogs; his
attitude toward them was less compromising. Not that he went out of his
way looking for trouble. Far from it. But upon his way he demanded
consideration. He stood upon his right to go his way unmolested and to
give trail to no dog. He had to be taken into account, that was all. He
was no longer to be disregarded and ignored, as was the lot of puppies,
and as continued to be the lot of the puppies that were his team-mates.
They got out of the way, gave trail to the grown dogs, and gave up meat
to them under compulsion. But White Fang, uncompanionable, solitary,
morose, scarcely looking to right or left, redoubtable, forbidding of
aspect, remote and alien, was accepted as an equal by his puzzled
elders. They quickly learned to leave him alone, neither venturing
hostile acts nor making overtures of friendliness. If they left him
alone, he left them alone—a state of affairs that they found, after a
few encounters, to be pre-eminently desirable.

In midsummer White Fang had an experience. Trotting along in his silent
way to investigate a new tepee which had been erected on the edge of
the village while he was away with the hunters after moose, he came
full upon Kiche. He paused and looked at her. He remembered her
vaguely, but he \emph{remembered} her, and that was more than could be said
for her. She lifted her lip at him in the old snarl of menace, and his
memory became clear. His forgotten cubhood, all that was associated
with that familiar snarl, rushed back to him. Before he had known the
gods, she had been to him the centre-pin of the universe. The old
familiar feelings of that time came back upon him, surged up within
him. He bounded towards her joyously, and she met him with shrewd fangs
that laid his cheek open to the bone. He did not understand. He backed
away, bewildered and puzzled.

But it was not Kiche’s fault. A wolf-mother was not made to remember
her cubs of a year or so before. So she did not remember White Fang. He
was a strange animal, an intruder; and her present litter of puppies
gave her the right to resent such intrusion.

One of the puppies sprawled up to White Fang. They were half-brothers,
only they did not know it. White Fang sniffed the puppy curiously,
whereupon Kiche rushed upon him, gashing his face a second time. He
backed farther away. All the old memories and associations died down
again and passed into the grave from which they had been resurrected.
He looked at Kiche licking her puppy and stopping now and then to snarl
at him. She was without value to him. He had learned to get along
without her. Her meaning was forgotten. There was no place for her in
his scheme of things, as there was no place for him in hers.

He was still standing, stupid and bewildered, the memories forgotten,
wondering what it was all about, when Kiche attacked him a third time,
intent on driving him away altogether from the vicinity. And White Fang
allowed himself to be driven away. This was a female of his kind, and
it was a law of his kind that the males must not fight the females. He
did not know anything about this law, for it was no generalisation of
the mind, not a something acquired by experience of the world. He knew
it as a secret prompting, as an urge of instinct—of the same instinct
that made him howl at the moon and stars of nights, and that made him
fear death and the unknown.

The months went by. White Fang grew stronger, heavier, and more
compact, while his character was developing along the lines laid down
by his heredity and his environment. His heredity was a life-stuff that
may be likened to clay. It possessed many possibilities, was capable of
being moulded into many different forms. Environment served to model
the clay, to give it a particular form. Thus, had White Fang never come
in to the fires of man, the Wild would have moulded him into a true
wolf. But the gods had given him a different environment, and he was
moulded into a dog that was rather wolfish, but that was a dog and not
a wolf.

And so, according to the clay of his nature and the pressure of his
surroundings, his character was being moulded into a certain particular
shape. There was no escaping it. He was becoming more morose, more
uncompanionable, more solitary, more ferocious; while the dogs were
learning more and more that it was better to be at peace with him than
at war, and Grey Beaver was coming to prize him more greatly with the
passage of each day.

White Fang, seeming to sum up strength in all his qualities,
nevertheless suffered from one besetting weakness. He could not stand
being laughed at. The laughter of men was a hateful thing. They might
laugh among themselves about anything they pleased except himself, and
he did not mind. But the moment laughter was turned upon him he would
fly into a most terrible rage. Grave, dignified, sombre, a laugh made
him frantic to ridiculousness. It so outraged him and upset him that
for hours he would behave like a demon. And woe to the dog that at such
times ran foul of him. He knew the law too well to take it out on Grey
Beaver; behind Grey Beaver were a club and godhead. But behind the dogs
there was nothing but space, and into this space they flew when White
Fang came on the scene, made mad by laughter.

In the third year of his life there came a great famine to the
Mackenzie Indians. In the summer the fish failed. In the winter the
cariboo forsook their accustomed track. Moose were scarce, the rabbits
almost disappeared, hunting and preying animals perished. Denied their
usual food-supply, weakened by hunger, they fell upon and devoured one
another. Only the strong survived. White Fang’s gods were always
hunting animals. The old and the weak of them died of hunger. There was
wailing in the village, where the women and children went without in
order that what little they had might go into the bellies of the lean
and hollow-eyed hunters who trod the forest in the vain pursuit of
meat.

To such extremity were the gods driven that they ate the soft-tanned
leather of their mocassins and mittens, while the dogs ate the
harnesses off their backs and the very whip-lashes. Also, the dogs ate
one another, and also the gods ate the dogs. The weakest and the more
worthless were eaten first. The dogs that still lived, looked on and
understood. A few of the boldest and wisest forsook the fires of the
gods, which had now become a shambles, and fled into the forest, where,
in the end, they starved to death or were eaten by wolves.

In this time of misery, White Fang, too, stole away into the woods. He
was better fitted for the life than the other dogs, for he had the
training of his cubhood to guide him. Especially adept did he become in
stalking small living things. He would lie concealed for hours,
following every movement of a cautious tree-squirrel, waiting, with a
patience as huge as the hunger he suffered from, until the squirrel
ventured out upon the ground. Even then, White Fang was not premature.
He waited until he was sure of striking before the squirrel could gain
a tree-refuge. Then, and not until then, would he flash from his
hiding-place, a grey projectile, incredibly swift, never failing its
mark—the fleeing squirrel that fled not fast enough.

Successful as he was with squirrels, there was one difficulty that
prevented him from living and growing fat on them. There were not
enough squirrels. So he was driven to hunt still smaller things. So
acute did his hunger become at times that he was not above rooting out
wood-mice from their burrows in the ground. Nor did he scorn to do
battle with a weasel as hungry as himself and many times more
ferocious.

In the worst pinches of the famine he stole back to the fires of the
gods. But he did not go into the fires. He lurked in the forest,
avoiding discovery and robbing the snares at the rare intervals when
game was caught. He even robbed Grey Beaver’s snare of a rabbit at a
time when Grey Beaver staggered and tottered through the forest,
sitting down often to rest, what of weakness and of shortness of
breath.

One day White Fang encountered a young wolf, gaunt and scrawny,
loose-jointed with famine. Had he not been hungry himself, White Fang
might have gone with him and eventually found his way into the pack
amongst his wild brethren. As it was, he ran the young wolf down and
killed and ate him.

Fortune seemed to favour him. Always, when hardest pressed for food, he
found something to kill. Again, when he was weak, it was his luck that
none of the larger preying animals chanced upon him. Thus, he was
strong from the two days’ eating a lynx had afforded him when the
hungry wolf-pack ran full tilt upon him. It was a long, cruel chase,
but he was better nourished than they, and in the end outran them. And
not only did he outrun them, but, circling widely back on his track, he
gathered in one of his exhausted pursuers.

After that he left that part of the country and journeyed over to the
valley wherein he had been born. Here, in the old lair, he encountered
Kiche. Up to her old tricks, she, too, had fled the inhospitable fires
of the gods and gone back to her old refuge to give birth to her young.
Of this litter but one remained alive when White Fang came upon the
scene, and this one was not destined to live long. Young life had
little chance in such a famine.

Kiche’s greeting of her grown son was anything but affectionate. But
White Fang did not mind. He had outgrown his mother. So he turned tail
philosophically and trotted on up the stream. At the forks he took the
turning to the left, where he found the lair of the lynx with whom his
mother and he had fought long before. Here, in the abandoned lair, he
settled down and rested for a day.

During the early summer, in the last days of the famine, he met
Lip-lip, who had likewise taken to the woods, where he had eked out a
miserable existence.

White Fang came upon him unexpectedly. Trotting in opposite directions
along the base of a high bluff, they rounded a corner of rock and found
themselves face to face. They paused with instant alarm, and looked at
each other suspiciously.

White Fang was in splendid condition. His hunting had been good, and
for a week he had eaten his fill. He was even gorged from his latest
kill. But in the moment he looked at Lip-lip his hair rose on end all
along his back. It was an involuntary bristling on his part, the
physical state that in the past had always accompanied the mental state
produced in him by Lip-lip’s bullying and persecution. As in the past
he had bristled and snarled at sight of Lip-lip, so now, and
automatically, he bristled and snarled. He did not waste any time. The
thing was done thoroughly and with despatch. Lip-lip essayed to back
away, but White Fang struck him hard, shoulder to shoulder. Lip-lip was
overthrown and rolled upon his back. White Fang’s teeth drove into the
scrawny throat. There was a death-struggle, during which White Fang
walked around, stiff-legged and observant. Then he resumed his course
and trotted on along the base of the bluff.

One day, not long after, he came to the edge of the forest, where a
narrow stretch of open land sloped down to the Mackenzie. He had been
over this ground before, when it was bare, but now a village occupied
it. Still hidden amongst the trees, he paused to study the situation.
Sights and sounds and scents were familiar to him. It was the old
village changed to a new place. But sights and sounds and smells were
different from those he had last had when he fled away from it. There
was no whimpering nor wailing. Contented sounds saluted his ear, and
when he heard the angry voice of a woman he knew it to be the anger
that proceeds from a full stomach. And there was a smell in the air of
fish. There was food. The famine was gone. He came out boldly from the
forest and trotted into camp straight to Grey Beaver’s tepee. Grey
Beaver was not there; but Kloo-kooch welcomed him with glad cries and
the whole of a fresh-caught fish, and he lay down to wait Grey Beaver’s
coming.

%\part{}
\chapter{The Enemy of His Kind}

Had there been in White Fang’s nature any possibility, no matter how
remote, of his ever coming to fraternise with his kind, such
possibility was irretrievably destroyed when he was made leader of the
sled-team. For now the dogs hated him—hated him for the extra meat
bestowed upon him by Mit-sah; hated him for all the real and fancied
favours he received; hated him for that he fled always at the head of
the team, his waving brush of a tail and his perpetually retreating
hind-quarters for ever maddening their eyes.

And White Fang just as bitterly hated them back. Being sled-leader was
anything but gratifying to him. To be compelled to run away before the
yelling pack, every dog of which, for three years, he had thrashed and
mastered, was almost more than he could endure. But endure it he must,
or perish, and the life that was in him had no desire to perish out.
The moment Mit-sah gave his order for the start, that moment the whole
team, with eager, savage cries, sprang forward at White Fang.

There was no defence for him. If he turned upon them, Mit-sah would
throw the stinging lash of the whip into his face. Only remained to him
to run away. He could not encounter that howling horde with his tail
and hind-quarters. These were scarcely fit weapons with which to meet
the many merciless fangs. So run away he did, violating his own nature
and pride with every leap he made, and leaping all day long.

One cannot violate the promptings of one’s nature without having that
nature recoil upon itself. Such a recoil is like that of a hair, made
to grow out from the body, turning unnaturally upon the direction of
its growth and growing into the body—a rankling, festering thing of
hurt. And so with White Fang. Every urge of his being impelled him to
spring upon the pack that cried at his heels, but it was the will of
the gods that this should not be; and behind the will, to enforce it,
was the whip of cariboo-gut with its biting thirty-foot lash. So White
Fang could only eat his heart in bitterness and develop a hatred and
malice commensurate with the ferocity and indomitability of his nature.

If ever a creature was the enemy of its kind, White Fang was that
creature. He asked no quarter, gave none. He was continually marred and
scarred by the teeth of the pack, and as continually he left his own
marks upon the pack. Unlike most leaders, who, when camp was made and
the dogs were unhitched, huddled near to the gods for protection, White
Fang disdained such protection. He walked boldly about the camp,
inflicting punishment in the night for what he had suffered in the day.
In the time before he was made leader of the team, the pack had learned
to get out of his way. But now it was different. Excited by the
day-long pursuit of him, swayed subconsciously by the insistent
iteration on their brains of the sight of him fleeing away, mastered by
the feeling of mastery enjoyed all day, the dogs could not bring
themselves to give way to him. When he appeared amongst them, there was
always a squabble. His progress was marked by snarl and snap and growl.
The very atmosphere he breathed was surcharged with hatred and malice,
and this but served to increase the hatred and malice within him.

When Mit-sah cried out his command for the team to stop, White Fang
obeyed. At first this caused trouble for the other dogs. All of them
would spring upon the hated leader only to find the tables turned.
Behind him would be Mit-sah, the great whip singing in his hand. So the
dogs came to understand that when the team stopped by order, White Fang
was to be let alone. But when White Fang stopped without orders, then
it was allowed them to spring upon him and destroy him if they could.
After several experiences, White Fang never stopped without orders. He
learned quickly. It was in the nature of things, that he must learn
quickly if he were to survive the unusually severe conditions under
which life was vouchsafed him.

But the dogs could never learn the lesson to leave him alone in camp.
Each day, pursuing him and crying defiance at him, the lesson of the
previous night was erased, and that night would have to be learned over
again, to be as immediately forgotten. Besides, there was a greater
consistence in their dislike of him. They sensed between themselves and
him a difference of kind—cause sufficient in itself for hostility. Like
him, they were domesticated wolves. But they had been domesticated for
generations. Much of the Wild had been lost, so that to them the Wild
was the unknown, the terrible, the ever-menacing and ever warring. But
to him, in appearance and action and impulse, still clung the Wild. He
symbolised it, was its personification: so that when they showed their
teeth to him they were defending themselves against the powers of
destruction that lurked in the shadows of the forest and in the dark
beyond the camp-fire.

But there was one lesson the dogs did learn, and that was to keep
together. White Fang was too terrible for any of them to face
single-handed. They met him with the mass-formation, otherwise he would
have killed them, one by one, in a night. As it was, he never had a
chance to kill them. He might roll a dog off its feet, but the pack
would be upon him before he could follow up and deliver the deadly
throat-stroke. At the first hint of conflict, the whole team drew
together and faced him. The dogs had quarrels among themselves, but
these were forgotten when trouble was brewing with White Fang.

On the other hand, try as they would, they could not kill White Fang.
He was too quick for them, too formidable, too wise. He avoided tight
places and always backed out of it when they bade fair to surround him.
While, as for getting him off his feet, there was no dog among them
capable of doing the trick. His feet clung to the earth with the same
tenacity that he clung to life. For that matter, life and footing were
synonymous in this unending warfare with the pack, and none knew it
better than White Fang.

So he became the enemy of his kind, domesticated wolves that they were,
softened by the fires of man, weakened in the sheltering shadow of
man’s strength. White Fang was bitter and implacable. The clay of him
was so moulded. He declared a vendetta against all dogs. And so
terribly did he live this vendetta that Grey Beaver, fierce savage
himself, could not but marvel at White Fang’s ferocity. Never, he
swore, had there been the like of this animal; and the Indians in
strange villages swore likewise when they considered the tale of his
killings amongst their dogs.

When White Fang was nearly five years old, Grey Beaver took him on
another great journey, and long remembered was the havoc he worked
amongst the dogs of the many villages along the Mackenzie, across the
Rockies, and down the Porcupine to the Yukon. He revelled in the
vengeance he wreaked upon his kind. They were ordinary, unsuspecting
dogs. They were not prepared for his swiftness and directness, for his
attack without warning. They did not know him for what he was, a
lightning-flash of slaughter. They bristled up to him, stiff-legged and
challenging, while he, wasting no time on elaborate preliminaries,
snapping into action like a steel spring, was at their throats and
destroying them before they knew what was happening and while they were
yet in the throes of surprise.

He became an adept at fighting. He economised. He never wasted his
strength, never tussled. He was in too quickly for that, and, if he
missed, was out again too quickly. The dislike of the wolf for close
quarters was his to an unusual degree. He could not endure a prolonged
contact with another body. It smacked of danger. It made him frantic.
He must be away, free, on his own legs, touching no living thing. It
was the Wild still clinging to him, asserting itself through him. This
feeling had been accentuated by the Ishmaelite life he had led from his
puppyhood. Danger lurked in contacts. It was the trap, ever the trap,
the fear of it lurking deep in the life of him, woven into the fibre of
him.

In consequence, the strange dogs he encountered had no chance against
him. He eluded their fangs. He got them, or got away, himself untouched
in either event. In the natural course of things there were exceptions
to this. There were times when several dogs, pitching on to him,
punished him before he could get away; and there were times when a
single dog scored deeply on him. But these were accidents. In the main,
so efficient a fighter had he become, he went his way unscathed.

Another advantage he possessed was that of correctly judging time and
distance. Not that he did this consciously, however. He did not
calculate such things. It was all automatic. His eyes saw correctly,
and the nerves carried the vision correctly to his brain. The parts of
him were better adjusted than those of the average dog. They worked
together more smoothly and steadily. His was a better, far better,
nervous, mental, and muscular co-ordination. When his eyes conveyed to
his brain the moving image of an action, his brain without conscious
effort, knew the space that limited that action and the time required
for its completion. Thus, he could avoid the leap of another dog, or
the drive of its fangs, and at the same moment could seize the
infinitesimal fraction of time in which to deliver his own attack. Body
and brain, his was a more perfected mechanism. Not that he was to be
praised for it. Nature had been more generous to him than to the
average animal, that was all.

It was in the summer that White Fang arrived at Fort Yukon. Grey Beaver
had crossed the great watershed between Mackenzie and the Yukon in the
late winter, and spent the spring in hunting among the western outlying
spurs of the Rockies. Then, after the break-up of the ice on the
Porcupine, he had built a canoe and paddled down that stream to where
it effected its junction with the Yukon just under the Artic circle.
Here stood the old Hudson’s Bay Company fort; and here were many
Indians, much food, and unprecedented excitement. It was the summer of
1898, and thousands of gold-hunters were going up the Yukon to Dawson
and the Klondike. Still hundreds of miles from their goal, nevertheless
many of them had been on the way for a year, and the least any of them
had travelled to get that far was five thousand miles, while some had
come from the other side of the world.

Here Grey Beaver stopped. A whisper of the gold-rush had reached his
ears, and he had come with several bales of furs, and another of
gut-sewn mittens and moccasins. He would not have ventured so long a
trip had he not expected generous profits. But what he had expected was
nothing to what he realised. His wildest dreams had not exceeded a
hundred per cent. profit; he made a thousand per cent. And like a true
Indian, he settled down to trade carefully and slowly, even if it took
all summer and the rest of the winter to dispose of his goods.

It was at Fort Yukon that White Fang saw his first white men. As
compared with the Indians he had known, they were to him another race
of beings, a race of superior gods. They impressed him as possessing
superior power, and it is on power that godhead rests. White Fang did
not reason it out, did not in his mind make the sharp generalisation
that the white gods were more powerful. It was a feeling, nothing more,
and yet none the less potent. As, in his puppyhood, the looming bulks
of the tepees, man-reared, had affected him as manifestations of power,
so was he affected now by the houses and the huge fort all of massive
logs. Here was power. Those white gods were strong. They possessed
greater mastery over matter than the gods he had known, most powerful
among which was Grey Beaver. And yet Grey Beaver was as a child-god
among these white-skinned ones.

To be sure, White Fang only felt these things. He was not conscious of
them. Yet it is upon feeling, more often than thinking, that animals
act; and every act White Fang now performed was based upon the feeling
that the white men were the superior gods. In the first place he was
very suspicious of them. There was no telling what unknown terrors were
theirs, what unknown hurts they could administer. He was curious to
observe them, fearful of being noticed by them. For the first few hours
he was content with slinking around and watching them from a safe
distance. Then he saw that no harm befell the dogs that were near to
them, and he came in closer.

In turn he was an object of great curiosity to them. His wolfish
appearance caught their eyes at once, and they pointed him out to one
another. This act of pointing put White Fang on his guard, and when
they tried to approach him he showed his teeth and backed away. Not one
succeeded in laying a hand on him, and it was well that they did not.

White Fang soon learned that very few of these gods—not more than a
dozen—lived at this place. Every two or three days a steamer (another
and colossal manifestation of power) came into the bank and stopped for
several hours. The white men came from off these steamers and went away
on them again. There seemed untold numbers of these white men. In the
first day or so, he saw more of them than he had seen Indians in all
his life; and as the days went by they continued to come up the river,
stop, and then go on up the river out of sight.

But if the white gods were all-powerful, their dogs did not amount to
much. This White Fang quickly discovered by mixing with those that came
ashore with their masters. They were irregular shapes and sizes. Some
were short-legged—too short; others were long-legged—too long. They had
hair instead of fur, and a few had very little hair at that. And none
of them knew how to fight.

As an enemy of his kind, it was in White Fang’s province to fight with
them. This he did, and he quickly achieved for them a mighty contempt.
They were soft and helpless, made much noise, and floundered around
clumsily trying to accomplish by main strength what he accomplished by
dexterity and cunning. They rushed bellowing at him. He sprang to the
side. They did not know what had become of him; and in that moment he
struck them on the shoulder, rolling them off their feet and delivering
his stroke at the throat.

Sometimes this stroke was successful, and a stricken dog rolled in the
dirt, to be pounced upon and torn to pieces by the pack of Indian dogs
that waited. White Fang was wise. He had long since learned that the
gods were made angry when their dogs were killed. The white men were no
exception to this. So he was content, when he had overthrown and
slashed wide the throat of one of their dogs, to drop back and let the
pack go in and do the cruel finishing work. It was then that the white
men rushed in, visiting their wrath heavily on the pack, while White
Fang went free. He would stand off at a little distance and look on,
while stones, clubs, axes, and all sorts of weapons fell upon his
fellows. White Fang was very wise.

But his fellows grew wise in their own way; and in this White Fang grew
wise with them. They learned that it was when a steamer first tied to
the bank that they had their fun. After the first two or three strange
dogs had been downed and destroyed, the white men hustled their own
animals back on board and wrecked savage vengeance on the offenders.
One white man, having seen his dog, a setter, torn to pieces before his
eyes, drew a revolver. He fired rapidly, six times, and six of the pack
lay dead or dying—another manifestation of power that sank deep into
White Fang’s consciousness.

White Fang enjoyed it all. He did not love his kind, and he was shrewd
enough to escape hurt himself. At first, the killing of the white men’s
dogs had been a diversion. After a time it became his occupation. There
was no work for him to do. Grey Beaver was busy trading and getting
wealthy. So White Fang hung around the landing with the disreputable
gang of Indian dogs, waiting for steamers. With the arrival of a
steamer the fun began. After a few minutes, by the time the white men
had got over their surprise, the gang scattered. The fun was over until
the next steamer should arrive.

But it can scarcely be said that White Fang was a member of the gang.
He did not mingle with it, but remained aloof, always himself, and was
even feared by it. It is true, he worked with it. He picked the quarrel
with the strange dog while the gang waited. And when he had overthrown
the strange dog the gang went in to finish it. But it is equally true
that he then withdrew, leaving the gang to receive the punishment of
the outraged gods.

It did not require much exertion to pick these quarrels. All he had to
do, when the strange dogs came ashore, was to show himself. When they
saw him they rushed for him. It was their instinct. He was the Wild—the
unknown, the terrible, the ever-menacing, the thing that prowled in the
darkness around the fires of the primeval world when they, cowering
close to the fires, were reshaping their instincts, learning to fear
the Wild out of which they had come, and which they had deserted and
betrayed. Generation by generation, down all the generations, had this
fear of the Wild been stamped into their natures. For centuries the
Wild had stood for terror and destruction. And during all this time
free licence had been theirs, from their masters, to kill the things of
the Wild. In doing this they had protected both themselves and the gods
whose companionship they shared.

And so, fresh from the soft southern world, these dogs, trotting down
the gang-plank and out upon the Yukon shore had but to see White Fang
to experience the irresistible impulse to rush upon him and destroy
him. They might be town-reared dogs, but the instinctive fear of the
Wild was theirs just the same. Not alone with their own eyes did they
see the wolfish creature in the clear light of day, standing before
them. They saw him with the eyes of their ancestors, and by their
inherited memory they knew White Fang for the wolf, and they remembered
the ancient feud.

All of which served to make White Fang’s days enjoyable. If the sight
of him drove these strange dogs upon him, so much the better for him,
so much the worse for them. They looked upon him as legitimate prey,
and as legitimate prey he looked upon them.

Not for nothing had he first seen the light of day in a lonely lair and
fought his first fights with the ptarmigan, the weasel, and the lynx.
And not for nothing had his puppyhood been made bitter by the
persecution of Lip-lip and the whole puppy pack. It might have been
otherwise, and he would then have been otherwise. Had Lip-lip not
existed, he would have passed his puppyhood with the other puppies and
grown up more doglike and with more liking for dogs. Had Grey Beaver
possessed the plummet of affection and love, he might have sounded the
deeps of White Fang’s nature and brought up to the surface all manner
of kindly qualities. But these things had not been so. The clay of
White Fang had been moulded until he became what he was, morose and
lonely, unloving and ferocious, the enemy of all his kind.

\chapter{The Mad God}

A small number of white men lived in Fort Yukon. These men had been
long in the country. They called themselves Sour-doughs, and took great
pride in so classifying themselves. For other men, new in the land,
they felt nothing but disdain. The men who came ashore from the
steamers were newcomers. They were known as \emph{chechaquos}, and they
always wilted at the application of the name. They made their bread
with baking-powder. This was the invidious distinction between them and
the Sour-doughs, who, forsooth, made their bread from sour-dough
because they had no baking-powder.

All of which is neither here nor there. The men in the fort disdained
the newcomers and enjoyed seeing them come to grief. Especially did
they enjoy the havoc worked amongst the newcomers’ dogs by White Fang
and his disreputable gang. When a steamer arrived, the men of the fort
made it a point always to come down to the bank and see the fun. They
looked forward to it with as much anticipation as did the Indian dogs,
while they were not slow to appreciate the savage and crafty part
played by White Fang.

But there was one man amongst them who particularly enjoyed the sport.
He would come running at the first sound of a steamboat’s whistle; and
when the last fight was over and White Fang and the pack had scattered,
he would return slowly to the fort, his face heavy with regret.
Sometimes, when a soft southland dog went down, shrieking its death-cry
under the fangs of the pack, this man would be unable to contain
himself, and would leap into the air and cry out with delight. And
always he had a sharp and covetous eye for White Fang.

This man was called “Beauty” by the other men of the fort. No one knew
his first name, and in general he was known in the country as Beauty
Smith. But he was anything save a beauty. To antithesis was due his
naming. He was pre-eminently unbeautiful. Nature had been niggardly
with him. He was a small man to begin with; and upon his meagre frame
was deposited an even more strikingly meagre head. Its apex might be
likened to a point. In fact, in his boyhood, before he had been named
Beauty by his fellows, he had been called “Pinhead.”

Backward, from the apex, his head slanted down to his neck and forward
it slanted uncompromisingly to meet a low and remarkably wide forehead.
Beginning here, as though regretting her parsimony, Nature had spread
his features with a lavish hand. His eyes were large, and between them
was the distance of two eyes. His face, in relation to the rest of him,
was prodigious. In order to discover the necessary area, Nature had
given him an enormous prognathous jaw. It was wide and heavy, and
protruded outward and down until it seemed to rest on his chest.
Possibly this appearance was due to the weariness of the slender neck,
unable properly to support so great a burden.

This jaw gave the impression of ferocious determination. But something
lacked. Perhaps it was from excess. Perhaps the jaw was too large. At
any rate, it was a lie. Beauty Smith was known far and wide as the
weakest of weak-kneed and snivelling cowards. To complete his
description, his teeth were large and yellow, while the two eye-teeth,
larger than their fellows, showed under his lean lips like fangs. His
eyes were yellow and muddy, as though Nature had run short on pigments
and squeezed together the dregs of all her tubes. It was the same with
his hair, sparse and irregular of growth, muddy-yellow and
dirty-yellow, rising on his head and sprouting out of his face in
unexpected tufts and bunches, in appearance like clumped and wind-blown
grain.

In short, Beauty Smith was a monstrosity, and the blame of it lay
elsewhere. He was not responsible. The clay of him had been so moulded
in the making. He did the cooking for the other men in the fort, the
dish-washing and the drudgery. They did not despise him. Rather did
they tolerate him in a broad human way, as one tolerates any creature
evilly treated in the making. Also, they feared him. His cowardly rages
made them dread a shot in the back or poison in their coffee. But
somebody had to do the cooking, and whatever else his shortcomings,
Beauty Smith could cook.

This was the man that looked at White Fang, delighted in his ferocious
prowess, and desired to possess him. He made overtures to White Fang
from the first. White Fang began by ignoring him. Later on, when the
overtures became more insistent, White Fang bristled and bared his
teeth and backed away. He did not like the man. The feel of him was
bad. He sensed the evil in him, and feared the extended hand and the
attempts at soft-spoken speech. Because of all this, he hated the man.

With the simpler creatures, good and bad are things simply understood.
The good stands for all things that bring easement and satisfaction and
surcease from pain. Therefore, the good is liked. The bad stands for
all things that are fraught with discomfort, menace, and hurt, and is
hated accordingly. White Fang’s feel of Beauty Smith was bad. From the
man’s distorted body and twisted mind, in occult ways, like mists
rising from malarial marshes, came emanations of the unhealth within.
Not by reasoning, not by the five senses alone, but by other and
remoter and uncharted senses, came the feeling to White Fang that the
man was ominous with evil, pregnant with hurtfulness, and therefore a
thing bad, and wisely to be hated.

White Fang was in Grey Beaver’s camp when Beauty Smith first visited
it. At the faint sound of his distant feet, before he came in sight,
White Fang knew who was coming and began to bristle. He had been lying
down in an abandon of comfort, but he arose quickly, and, as the man
arrived, slid away in true wolf-fashion to the edge of the camp. He did
not know what they said, but he could see the man and Grey Beaver
talking together. Once, the man pointed at him, and White Fang snarled
back as though the hand were just descending upon him instead of being,
as it was, fifty feet away. The man laughed at this; and White Fang
slunk away to the sheltering woods, his head turned to observe as he
glided softly over the ground.

Grey Beaver refused to sell the dog. He had grown rich with his trading
and stood in need of nothing. Besides, White Fang was a valuable
animal, the strongest sled-dog he had ever owned, and the best leader.
Furthermore, there was no dog like him on the Mackenzie nor the Yukon.
He could fight. He killed other dogs as easily as men killed
mosquitoes. (Beauty Smith’s eyes lighted up at this, and he licked his
thin lips with an eager tongue). No, White Fang was not for sale at any
price.

But Beauty Smith knew the ways of Indians. He visited Grey Beaver’s
camp often, and hidden under his coat was always a black bottle or so.
One of the potencies of whisky is the breeding of thirst. Grey Beaver
got the thirst. His fevered membranes and burnt stomach began to
clamour for more and more of the scorching fluid; while his brain,
thrust all awry by the unwonted stimulant, permitted him to go any
length to obtain it. The money he had received for his furs and mittens
and moccasins began to go. It went faster and faster, and the shorter
his money-sack grew, the shorter grew his temper.

In the end his money and goods and temper were all gone. Nothing
remained to him but his thirst, a prodigious possession in itself that
grew more prodigious with every sober breath he drew. Then it was that
Beauty Smith had talk with him again about the sale of White Fang; but
this time the price offered was in bottles, not dollars, and Grey
Beaver’s ears were more eager to hear.

“You ketch um dog you take um all right,” was his last word.

The bottles were delivered, but after two days. “You ketch um dog,”
were Beauty Smith’s words to Grey Beaver.

White Fang slunk into camp one evening and dropped down with a sigh of
content. The dreaded white god was not there. For days his
manifestations of desire to lay hands on him had been growing more
insistent, and during that time White Fang had been compelled to avoid
the camp. He did not know what evil was threatened by those insistent
hands. He knew only that they did threaten evil of some sort, and that
it was best for him to keep out of their reach.

But scarcely had he lain down when Grey Beaver staggered over to him
and tied a leather thong around his neck. He sat down beside White
Fang, holding the end of the thong in his hand. In the other hand he
held a bottle, which, from time to time, was inverted above his head to
the accompaniment of gurgling noises.

An hour of this passed, when the vibrations of feet in contact with the
ground foreran the one who approached. White Fang heard it first, and
he was bristling with recognition while Grey Beaver still nodded
stupidly. White Fang tried to draw the thong softly out of his master’s
hand; but the relaxed fingers closed tightly and Grey Beaver roused
himself.

Beauty Smith strode into camp and stood over White Fang. He snarled
softly up at the thing of fear, watching keenly the deportment of the
hands. One hand extended outward and began to descend upon his head.
His soft snarl grew tense and harsh. The hand continued slowly to
descend, while he crouched beneath it, eyeing it malignantly, his snarl
growing shorter and shorter as, with quickening breath, it approached
its culmination. Suddenly he snapped, striking with his fangs like a
snake. The hand was jerked back, and the teeth came together emptily
with a sharp click. Beauty Smith was frightened and angry. Grey Beaver
clouted White Fang alongside the head, so that he cowered down close to
the earth in respectful obedience.

White Fang’s suspicious eyes followed every movement. He saw Beauty
Smith go away and return with a stout club. Then the end of the thong
was given over to him by Grey Beaver. Beauty Smith started to walk
away. The thong grew taut. White Fang resisted it. Grey Beaver clouted
him right and left to make him get up and follow. He obeyed, but with a
rush, hurling himself upon the stranger who was dragging him away.
Beauty Smith did not jump away. He had been waiting for this. He swung
the club smartly, stopping the rush midway and smashing White Fang down
upon the ground. Grey Beaver laughed and nodded approval. Beauty Smith
tightened the thong again, and White Fang crawled limply and dizzily to
his feet.

He did not rush a second time. One smash from the club was sufficient
to convince him that the white god knew how to handle it, and he was
too wise to fight the inevitable. So he followed morosely at Beauty
Smith’s heels, his tail between his legs, yet snarling softly under his
breath. But Beauty Smith kept a wary eye on him, and the club was held
always ready to strike.

At the fort Beauty Smith left him securely tied and went in to bed.
White Fang waited an hour. Then he applied his teeth to the thong, and
in the space of ten seconds was free. He had wasted no time with his
teeth. There had been no useless gnawing. The thong was cut across,
diagonally, almost as clean as though done by a knife. White Fang
looked up at the fort, at the same time bristling and growling. Then he
turned and trotted back to Grey Beaver’s camp. He owed no allegiance to
this strange and terrible god. He had given himself to Grey Beaver, and
to Grey Beaver he considered he still belonged.

But what had occurred before was repeated—with a difference. Grey
Beaver again made him fast with a thong, and in the morning turned him
over to Beauty Smith. And here was where the difference came in. Beauty
Smith gave him a beating. Tied securely, White Fang could only rage
futilely and endure the punishment. Club and whip were both used upon
him, and he experienced the worst beating he had ever received in his
life. Even the big beating given him in his puppyhood by Grey Beaver
was mild compared with this.

Beauty Smith enjoyed the task. He delighted in it. He gloated over his
victim, and his eyes flamed dully, as he swung the whip or club and
listened to White Fang’s cries of pain and to his helpless bellows and
snarls. For Beauty Smith was cruel in the way that cowards are cruel.
Cringing and snivelling himself before the blows or angry speech of a
man, he revenged himself, in turn, upon creatures weaker than he. All
life likes power, and Beauty Smith was no exception. Denied the
expression of power amongst his own kind, he fell back upon the lesser
creatures and there vindicated the life that was in him. But Beauty
Smith had not created himself, and no blame was to be attached to him.
He had come into the world with a twisted body and a brute
intelligence. This had constituted the clay of him, and it had not been
kindly moulded by the world.

White Fang knew why he was being beaten. When Grey Beaver tied the
thong around his neck, and passed the end of the thong into Beauty
Smith’s keeping, White Fang knew that it was his god’s will for him to
go with Beauty Smith. And when Beauty Smith left him tied outside the
fort, he knew that it was Beauty Smith’s will that he should remain
there. Therefore, he had disobeyed the will of both the gods, and
earned the consequent punishment. He had seen dogs change owners in the
past, and he had seen the runaways beaten as he was being beaten. He
was wise, and yet in the nature of him there were forces greater than
wisdom. One of these was fidelity. He did not love Grey Beaver, yet,
even in the face of his will and his anger, he was faithful to him. He
could not help it. This faithfulness was a quality of the clay that
composed him. It was the quality that was peculiarly the possession of
his kind; the quality that set apart his species from all other
species; the quality that has enabled the wolf and the wild dog to come
in from the open and be the companions of man.

After the beating, White Fang was dragged back to the fort. But this
time Beauty Smith left him tied with a stick. One does not give up a
god easily, and so with White Fang. Grey Beaver was his own particular
god, and, in spite of Grey Beaver’s will, White Fang still clung to him
and would not give him up. Grey Beaver had betrayed and forsaken him,
but that had no effect upon him. Not for nothing had he surrendered
himself body and soul to Grey Beaver. There had been no reservation on
White Fang’s part, and the bond was not to be broken easily.

So, in the night, when the men in the fort were asleep, White Fang
applied his teeth to the stick that held him. The wood was seasoned and
dry, and it was tied so closely to his neck that he could scarcely get
his teeth to it. It was only by the severest muscular exertion and
neck-arching that he succeeded in getting the wood between his teeth,
and barely between his teeth at that; and it was only by the exercise
of an immense patience, extending through many hours, that he succeeded
in gnawing through the stick. This was something that dogs were not
supposed to do. It was unprecedented. But White Fang did it, trotting
away from the fort in the early morning, with the end of the stick
hanging to his neck.

He was wise. But had he been merely wise he would not have gone back to
Grey Beaver who had already twice betrayed him. But there was his
faithfulness, and he went back to be betrayed yet a third time. Again
he yielded to the tying of a thong around his neck by Grey Beaver, and
again Beauty Smith came to claim him. And this time he was beaten even
more severely than before.

Grey Beaver looked on stolidly while the white man wielded the whip. He
gave no protection. It was no longer his dog. When the beating was over
White Fang was sick. A soft southland dog would have died under it, but
not he. His school of life had been sterner, and he was himself of
sterner stuff. He had too great vitality. His clutch on life was too
strong. But he was very sick. At first he was unable to drag himself
along, and Beauty Smith had to wait half-an-hour for him. And then,
blind and reeling, he followed at Beauty Smith’s heels back to the
fort.

But now he was tied with a chain that defied his teeth, and he strove
in vain, by lunging, to draw the staple from the timber into which it
was driven. After a few days, sober and bankrupt, Grey Beaver departed
up the Porcupine on his long journey to the Mackenzie. White Fang
remained on the Yukon, the property of a man more than half mad and all
brute. But what is a dog to know in its consciousness of madness? To
White Fang, Beauty Smith was a veritable, if terrible, god. He was a
mad god at best, but White Fang knew nothing of madness; he knew only
that he must submit to the will of this new master, obey his every whim
and fancy.

\chapter{The Reign of Hate}

Under the tutelage of the mad god, White Fang became a fiend. He was
kept chained in a pen at the rear of the fort, and here Beauty Smith
teased and irritated and drove him wild with petty torments. The man
early discovered White Fang’s susceptibility to laughter, and made it a
point after painfully tricking him, to laugh at him. This laughter was
uproarious and scornful, and at the same time the god pointed his
finger derisively at White Fang. At such times reason fled from White
Fang, and in his transports of rage he was even more mad than Beauty
Smith.

Formerly, White Fang had been merely the enemy of his kind, withal a
ferocious enemy. He now became the enemy of all things, and more
ferocious than ever. To such an extent was he tormented, that he hated
blindly and without the faintest spark of reason. He hated the chain
that bound him, the men who peered in at him through the slats of the
pen, the dogs that accompanied the men and that snarled malignantly at
him in his helplessness. He hated the very wood of the pen that
confined him. And, first, last, and most of all, he hated Beauty Smith.

But Beauty Smith had a purpose in all that he did to White Fang. One
day a number of men gathered about the pen. Beauty Smith entered, club
in hand, and took the chain off from White Fang’s neck. When his master
had gone out, White Fang turned loose and tore around the pen, trying
to get at the men outside. He was magnificently terrible. Fully five
feet in length, and standing two and one-half feet at the shoulder, he
far outweighed a wolf of corresponding size. From his mother he had
inherited the heavier proportions of the dog, so that he weighed,
without any fat and without an ounce of superfluous flesh, over ninety
pounds. It was all muscle, bone, and sinew-fighting flesh in the finest
condition.

The door of the pen was being opened again. White Fang paused.
Something unusual was happening. He waited. The door was opened wider.
Then a huge dog was thrust inside, and the door was slammed shut behind
him. White Fang had never seen such a dog (it was a mastiff); but the
size and fierce aspect of the intruder did not deter him. Here was some
thing, not wood nor iron, upon which to wreak his hate. He leaped in
with a flash of fangs that ripped down the side of the mastiff’s neck.
The mastiff shook his head, growled hoarsely, and plunged at White
Fang. But White Fang was here, there, and everywhere, always evading
and eluding, and always leaping in and slashing with his fangs and
leaping out again in time to escape punishment.

The men outside shouted and applauded, while Beauty Smith, in an
ecstasy of delight, gloated over the ripping and mangling performed by
White Fang. There was no hope for the mastiff from the first. He was
too ponderous and slow. In the end, while Beauty Smith beat White Fang
back with a club, the mastiff was dragged out by its owner. Then there
was a payment of bets, and money clinked in Beauty Smith’s hand.

White Fang came to look forward eagerly to the gathering of the men
around his pen. It meant a fight; and this was the only way that was
now vouchsafed him of expressing the life that was in him. Tormented,
incited to hate, he was kept a prisoner so that there was no way of
satisfying that hate except at the times his master saw fit to put
another dog against him. Beauty Smith had estimated his powers well,
for he was invariably the victor. One day, three dogs were turned in
upon him in succession. Another day a full-grown wolf, fresh-caught
from the Wild, was shoved in through the door of the pen. And on still
another day two dogs were set against him at the same time. This was
his severest fight, and though in the end he killed them both he was
himself half killed in doing it.

In the fall of the year, when the first snows were falling and mush-ice
was running in the river, Beauty Smith took passage for himself and
White Fang on a steamboat bound up the Yukon to Dawson. White Fang had
now achieved a reputation in the land. As “the Fighting Wolf” he was
known far and wide, and the cage in which he was kept on the
steam-boat’s deck was usually surrounded by curious men. He raged and
snarled at them, or lay quietly and studied them with cold hatred. Why
should he not hate them? He never asked himself the question. He knew
only hate and lost himself in the passion of it. Life had become a hell
to him. He had not been made for the close confinement wild beasts
endure at the hands of men. And yet it was in precisely this way that
he was treated. Men stared at him, poked sticks between the bars to
make him snarl, and then laughed at him.

They were his environment, these men, and they were moulding the clay
of him into a more ferocious thing than had been intended by Nature.
Nevertheless, Nature had given him plasticity. Where many another
animal would have died or had its spirit broken, he adjusted himself
and lived, and at no expense of the spirit. Possibly Beauty Smith,
arch-fiend and tormentor, was capable of breaking White Fang’s spirit,
but as yet there were no signs of his succeeding.

If Beauty Smith had in him a devil, White Fang had another; and the two
of them raged against each other unceasingly. In the days before, White
Fang had had the wisdom to cower down and submit to a man with a club
in his hand; but this wisdom now left him. The mere sight of Beauty
Smith was sufficient to send him into transports of fury. And when they
came to close quarters, and he had been beaten back by the club, he
went on growling and snarling, and showing his fangs. The last growl
could never be extracted from him. No matter how terribly he was
beaten, he had always another growl; and when Beauty Smith gave up and
withdrew, the defiant growl followed after him, or White Fang sprang at
the bars of the cage bellowing his hatred.

When the steamboat arrived at Dawson, White Fang went ashore. But he
still lived a public life, in a cage, surrounded by curious men. He was
exhibited as “the Fighting Wolf,” and men paid fifty cents in gold dust
to see him. He was given no rest. Did he lie down to sleep, he was
stirred up by a sharp stick—so that the audience might get its money’s
worth. In order to make the exhibition interesting, he was kept in a
rage most of the time. But worse than all this, was the atmosphere in
which he lived. He was regarded as the most fearful of wild beasts, and
this was borne in to him through the bars of the cage. Every word,
every cautious action, on the part of the men, impressed upon him his
own terrible ferocity. It was so much added fuel to the flame of his
fierceness. There could be but one result, and that was that his
ferocity fed upon itself and increased. It was another instance of the
plasticity of his clay, of his capacity for being moulded by the
pressure of environment.

In addition to being exhibited he was a professional fighting animal.
At irregular intervals, whenever a fight could be arranged, he was
taken out of his cage and led off into the woods a few miles from town.
Usually this occurred at night, so as to avoid interference from the
mounted police of the Territory. After a few hours of waiting, when
daylight had come, the audience and the dog with which he was to fight
arrived. In this manner it came about that he fought all sizes and
breeds of dogs. It was a savage land, the men were savage, and the
fights were usually to the death.

Since White Fang continued to fight, it is obvious that it was the
other dogs that died. He never knew defeat. His early training, when he
fought with Lip-lip and the whole puppy-pack, stood him in good stead.
There was the tenacity with which he clung to the earth. No dog could
make him lose his footing. This was the favourite trick of the wolf
breeds—to rush in upon him, either directly or with an unexpected
swerve, in the hope of striking his shoulder and overthrowing him.
Mackenzie hounds, Eskimo and Labrador dogs, huskies and Malemutes—all
tried it on him, and all failed. He was never known to lose his
footing. Men told this to one another, and looked each time to see it
happen; but White Fang always disappointed them.

Then there was his lightning quickness. It gave him a tremendous
advantage over his antagonists. No matter what their fighting
experience, they had never encountered a dog that moved so swiftly as
he. Also to be reckoned with, was the immediateness of his attack. The
average dog was accustomed to the preliminaries of snarling and
bristling and growling, and the average dog was knocked off his feet
and finished before he had begun to fight or recovered from his
surprise. So often did this happen, that it became the custom to hold
White Fang until the other dog went through its preliminaries, was good
and ready, and even made the first attack.

But greatest of all the advantages in White Fang’s favour, was his
experience. He knew more about fighting than did any of the dogs that
faced him. He had fought more fights, knew how to meet more tricks and
methods, and had more tricks himself, while his own method was scarcely
to be improved upon.

As the time went by, he had fewer and fewer fights. Men despaired of
matching him with an equal, and Beauty Smith was compelled to pit
wolves against him. These were trapped by the Indians for the purpose,
and a fight between White Fang and a wolf was always sure to draw a
crowd. Once, a full-grown female lynx was secured, and this time White
Fang fought for his life. Her quickness matched his; her ferocity
equalled his; while he fought with his fangs alone, and she fought with
her sharp-clawed feet as well.

But after the lynx, all fighting ceased for White Fang. There were no
more animals with which to fight—at least, there was none considered
worthy of fighting with him. So he remained on exhibition until spring,
when one Tim Keenan, a faro-dealer, arrived in the land. With him came
the first bull-dog that had ever entered the Klondike. That this dog
and White Fang should come together was inevitable, and for a week the
anticipated fight was the mainspring of conversation in certain
quarters of the town.
\chapter{The Clinging Death}

Beauty Smith slipped the chain from his neck and stepped back.

For once White Fang did not make an immediate attack. He stood still,
ears pricked forward, alert and curious, surveying the strange animal
that faced him. He had never seen such a dog before. Tim Keenan shoved
the bull-dog forward with a muttered “Go to it.” The animal waddled
toward the centre of the circle, short and squat and ungainly. He came
to a stop and blinked across at White Fang.

There were cries from the crowd of, “Go to him, Cherokee! Sick ’m,
Cherokee! Eat ’m up!”

But Cherokee did not seem anxious to fight. He turned his head and
blinked at the men who shouted, at the same time wagging his stump of a
tail good-naturedly. He was not afraid, but merely lazy. Besides, it
did not seem to him that it was intended he should fight with the dog
he saw before him. He was not used to fighting with that kind of dog,
and he was waiting for them to bring on the real dog.

Tim Keenan stepped in and bent over Cherokee, fondling him on both
sides of the shoulders with hands that rubbed against the grain of the
hair and that made slight, pushing-forward movements. These were so
many suggestions. Also, their effect was irritating, for Cherokee began
to growl, very softly, deep down in his throat. There was a
correspondence in rhythm between the growls and the movements of the
man’s hands. The growl rose in the throat with the culmination of each
forward-pushing movement, and ebbed down to start up afresh with the
beginning of the next movement. The end of each movement was the accent
of the rhythm, the movement ending abruptly and the growling rising
with a jerk.

This was not without its effect on White Fang. The hair began to rise
on his neck and across the shoulders. Tim Keenan gave a final shove
forward and stepped back again. As the impetus that carried Cherokee
forward died down, he continued to go forward of his own volition, in a
swift, bow-legged run. Then White Fang struck. A cry of startled
admiration went up. He had covered the distance and gone in more like a
cat than a dog; and with the same cat-like swiftness he had slashed
with his fangs and leaped clear.

The bull-dog was bleeding back of one ear from a rip in his thick neck.
He gave no sign, did not even snarl, but turned and followed after
White Fang. The display on both sides, the quickness of the one and the
steadiness of the other, had excited the partisan spirit of the crowd,
and the men were making new bets and increasing original bets. Again,
and yet again, White Fang sprang in, slashed, and got away untouched,
and still his strange foe followed after him, without too great haste,
not slowly, but deliberately and determinedly, in a businesslike sort
of way. There was purpose in his method—something for him to do that he
was intent upon doing and from which nothing could distract him.

His whole demeanour, every action, was stamped with this purpose. It
puzzled White Fang. Never had he seen such a dog. It had no hair
protection. It was soft, and bled easily. There was no thick mat of fur
to baffle White Fang’s teeth as they were often baffled by dogs of his
own breed. Each time that his teeth struck they sank easily into the
yielding flesh, while the animal did not seem able to defend itself.
Another disconcerting thing was that it made no outcry, such as he had
been accustomed to with the other dogs he had fought. Beyond a growl or
a grunt, the dog took its punishment silently. And never did it flag in
its pursuit of him.

Not that Cherokee was slow. He could turn and whirl swiftly enough, but
White Fang was never there. Cherokee was puzzled, too. He had never
fought before with a dog with which he could not close. The desire to
close had always been mutual. But here was a dog that kept at a
distance, dancing and dodging here and there and all about. And when it
did get its teeth into him, it did not hold on but let go instantly and
darted away again.

But White Fang could not get at the soft underside of the throat. The
bull-dog stood too short, while its massive jaws were an added
protection. White Fang darted in and out unscathed, while Cherokee’s
wounds increased. Both sides of his neck and head were ripped and
slashed. He bled freely, but showed no signs of being disconcerted. He
continued his plodding pursuit, though once, for the moment baffled, he
came to a full stop and blinked at the men who looked on, at the same
time wagging his stump of a tail as an expression of his willingness to
fight.

In that moment White Fang was in upon him and out, in passing ripping
his trimmed remnant of an ear. With a slight manifestation of anger,
Cherokee took up the pursuit again, running on the inside of the circle
White Fang was making, and striving to fasten his deadly grip on White
Fang’s throat. The bull-dog missed by a hair’s-breadth, and cries of
praise went up as White Fang doubled suddenly out of danger in the
opposite direction.

The time went by. White Fang still danced on, dodging and doubling,
leaping in and out, and ever inflicting damage. And still the bull-dog,
with grim certitude, toiled after him. Sooner or later he would
accomplish his purpose, get the grip that would win the battle. In the
meantime, he accepted all the punishment the other could deal him. His
tufts of ears had become tassels, his neck and shoulders were slashed
in a score of places, and his very lips were cut and bleeding—all from
these lightning snaps that were beyond his foreseeing and guarding.

Time and again White Fang had attempted to knock Cherokee off his feet;
but the difference in their height was too great. Cherokee was too
squat, too close to the ground. White Fang tried the trick once too
often. The chance came in one of his quick doublings and
counter-circlings. He caught Cherokee with head turned away as he
whirled more slowly. His shoulder was exposed. White Fang drove in upon
it: but his own shoulder was high above, while he struck with such
force that his momentum carried him on across over the other’s body.
For the first time in his fighting history, men saw White Fang lose his
footing. His body turned a half-somersault in the air, and he would
have landed on his back had he not twisted, catlike, still in the air,
in the effort to bring his feet to the earth. As it was, he struck
heavily on his side. The next instant he was on his feet, but in that
instant Cherokee’s teeth closed on his throat.

It was not a good grip, being too low down toward the chest; but
Cherokee held on. White Fang sprang to his feet and tore wildly around,
trying to shake off the bull-dog’s body. It made him frantic, this
clinging, dragging weight. It bound his movements, restricted his
freedom. It was like the trap, and all his instinct resented it and
revolted against it. It was a mad revolt. For several minutes he was to
all intents insane. The basic life that was in him took charge of him.
The will to exist of his body surged over him. He was dominated by this
mere flesh-love of life. All intelligence was gone. It was as though he
had no brain. His reason was unseated by the blind yearning of the
flesh to exist and move, at all hazards to move, to continue to move,
for movement was the expression of its existence.

Round and round he went, whirling and turning and reversing, trying to
shake off the fifty-pound weight that dragged at his throat. The
bull-dog did little but keep his grip. Sometimes, and rarely, he
managed to get his feet to the earth and for a moment to brace himself
against White Fang. But the next moment his footing would be lost and
he would be dragging around in the whirl of one of White Fang’s mad
gyrations. Cherokee identified himself with his instinct. He knew that
he was doing the right thing by holding on, and there came to him
certain blissful thrills of satisfaction. At such moments he even
closed his eyes and allowed his body to be hurled hither and thither,
willy-nilly, careless of any hurt that might thereby come to it. That
did not count. The grip was the thing, and the grip he kept.

White Fang ceased only when he had tired himself out. He could do
nothing, and he could not understand. Never, in all his fighting, had
this thing happened. The dogs he had fought with did not fight that
way. With them it was snap and slash and get away, snap and slash and
get away. He lay partly on his side, panting for breath. Cherokee still
holding his grip, urged against him, trying to get him over entirely on
his side. White Fang resisted, and he could feel the jaws shifting
their grip, slightly relaxing and coming together again in a chewing
movement. Each shift brought the grip closer to his throat. The
bull-dog’s method was to hold what he had, and when opportunity
favoured to work in for more. Opportunity favoured when White Fang
remained quiet. When White Fang struggled, Cherokee was content merely
to hold on.

The bulging back of Cherokee’s neck was the only portion of his body
that White Fang’s teeth could reach. He got hold toward the base where
the neck comes out from the shoulders; but he did not know the chewing
method of fighting, nor were his jaws adapted to it. He spasmodically
ripped and tore with his fangs for a space. Then a change in their
position diverted him. The bull-dog had managed to roll him over on his
back, and still hanging on to his throat, was on top of him. Like a
cat, White Fang bowed his hind-quarters in, and, with the feet digging
into his enemy’s abdomen above him, he began to claw with long
tearing-strokes. Cherokee might well have been disembowelled had he not
quickly pivoted on his grip and got his body off of White Fang’s and at
right angles to it.

There was no escaping that grip. It was like Fate itself, and as
inexorable. Slowly it shifted up along the jugular. All that saved
White Fang from death was the loose skin of his neck and the thick fur
that covered it. This served to form a large roll in Cherokee’s mouth,
the fur of which well-nigh defied his teeth. But bit by bit, whenever
the chance offered, he was getting more of the loose skin and fur in
his mouth. The result was that he was slowly throttling White Fang. The
latter’s breath was drawn with greater and greater difficulty as the
moments went by.

It began to look as though the battle were over. The backers of
Cherokee waxed jubilant and offered ridiculous odds. White Fang’s
backers were correspondingly depressed, and refused bets of ten to one
and twenty to one, though one man was rash enough to close a wager of
fifty to one. This man was Beauty Smith. He took a step into the ring
and pointed his finger at White Fang. Then he began to laugh derisively
and scornfully. This produced the desired effect. White Fang went wild
with rage. He called up his reserves of strength, and gained his feet.
As he struggled around the ring, the fifty pounds of his foe ever
dragging on his throat, his anger passed on into panic. The basic life
of him dominated him again, and his intelligence fled before the will
of his flesh to live. Round and round and back again, stumbling and
falling and rising, even uprearing at times on his hind-legs and
lifting his foe clear of the earth, he struggled vainly to shake off
the clinging death.

At last he fell, toppling backward, exhausted; and the bull-dog
promptly shifted his grip, getting in closer, mangling more and more of
the fur-folded flesh, throttling White Fang more severely than ever.
Shouts of applause went up for the victor, and there were many cries of
“Cherokee!” “Cherokee!” To this Cherokee responded by vigorous wagging
of the stump of his tail. But the clamour of approval did not distract
him. There was no sympathetic relation between his tail and his massive
jaws. The one might wag, but the others held their terrible grip on
White Fang’s throat.

It was at this time that a diversion came to the spectators. There was
a jingle of bells. Dog-mushers’ cries were heard. Everybody, save
Beauty Smith, looked apprehensively, the fear of the police strong upon
them. But they saw, up the trail, and not down, two men running with
sled and dogs. They were evidently coming down the creek from some
prospecting trip. At sight of the crowd they stopped their dogs and
came over and joined it, curious to see the cause of the excitement.
The dog-musher wore a moustache, but the other, a taller and younger
man, was smooth-shaven, his skin rosy from the pounding of his blood
and the running in the frosty air.

White Fang had practically ceased struggling. Now and again he resisted
spasmodically and to no purpose. He could get little air, and that
little grew less and less under the merciless grip that ever tightened.
In spite of his armour of fur, the great vein of his throat would have
long since been torn open, had not the first grip of the bull-dog been
so low down as to be practically on the chest. It had taken Cherokee a
long time to shift that grip upward, and this had also tended further
to clog his jaws with fur and skin-fold.

In the meantime, the abysmal brute in Beauty Smith had been rising into
his brain and mastering the small bit of sanity that he possessed at
best. When he saw White Fang’s eyes beginning to glaze, he knew beyond
doubt that the fight was lost. Then he broke loose. He sprang upon
White Fang and began savagely to kick him. There were hisses from the
crowd and cries of protest, but that was all. While this went on, and
Beauty Smith continued to kick White Fang, there was a commotion in the
crowd. The tall young newcomer was forcing his way through, shouldering
men right and left without ceremony or gentleness. When he broke
through into the ring, Beauty Smith was just in the act of delivering
another kick. All his weight was on one foot, and he was in a state of
unstable equilibrium. At that moment the newcomer’s fist landed a
smashing blow full in his face. Beauty Smith’s remaining leg left the
ground, and his whole body seemed to lift into the air as he turned
over backward and struck the snow. The newcomer turned upon the crowd.

“You cowards!” he cried. “You beasts!”

He was in a rage himself—a sane rage. His grey eyes seemed metallic and
steel-like as they flashed upon the crowd. Beauty Smith regained his
feet and came toward him, sniffling and cowardly. The new-comer did not
understand. He did not know how abject a coward the other was, and
thought he was coming back intent on fighting. So, with a “You beast!”
he smashed Beauty Smith over backward with a second blow in the face.
Beauty Smith decided that the snow was the safest place for him, and
lay where he had fallen, making no effort to get up.

“Come on, Matt, lend a hand,” the newcomer called the dog-musher, who
had followed him into the ring.

Both men bent over the dogs. Matt took hold of White Fang, ready to
pull when Cherokee’s jaws should be loosened. This the younger man
endeavoured to accomplish by clutching the bulldog’s jaws in his hands
and trying to spread them. It was a vain undertaking. As he pulled and
tugged and wrenched, he kept exclaiming with every expulsion of breath,
“Beasts!”

The crowd began to grow unruly, and some of the men were protesting
against the spoiling of the sport; but they were silenced when the
newcomer lifted his head from his work for a moment and glared at them.

“You damn beasts!” he finally exploded, and went back to his task.

“It’s no use, Mr. Scott, you can’t break ’m apart that way,” Matt said
at last.

The pair paused and surveyed the locked dogs.

“Ain’t bleedin’ much,” Matt announced. “Ain’t got all the way in yet.”

“But he’s liable to any moment,” Scott answered. “There, did you see
that! He shifted his grip in a bit.”

The younger man’s excitement and apprehension for White Fang was
growing. He struck Cherokee about the head savagely again and again.
But that did not loosen the jaws. Cherokee wagged the stump of his tail
in advertisement that he understood the meaning of the blows, but that
he knew he was himself in the right and only doing his duty by keeping
his grip.

“Won’t some of you help?” Scott cried desperately at the crowd.

But no help was offered. Instead, the crowd began sarcastically to
cheer him on and showered him with facetious advice.

“You’ll have to get a pry,” Matt counselled.

The other reached into the holster at his hip, drew his revolver, and
tried to thrust its muzzle between the bull-dog’s jaws. He shoved, and
shoved hard, till the grating of the steel against the locked teeth
could be distinctly heard. Both men were on their knees, bending over
the dogs. Tim Keenan strode into the ring. He paused beside Scott and
touched him on the shoulder, saying ominously:

“Don’t break them teeth, stranger.”

“Then I’ll break his neck,” Scott retorted, continuing his shoving and
wedging with the revolver muzzle.

“I said don’t break them teeth,” the faro-dealer repeated more
ominously than before.

But if it was a bluff he intended, it did not work. Scott never
desisted from his efforts, though he looked up coolly and asked:

“Your dog?”

The faro-dealer grunted.

“Then get in here and break this grip.”

“Well, stranger,” the other drawled irritatingly, “I don’t mind telling
you that’s something I ain’t worked out for myself. I don’t know how to
turn the trick.”

“Then get out of the way,” was the reply, “and don’t bother me. I’m
busy.”

Tim Keenan continued standing over him, but Scott took no further
notice of his presence. He had managed to get the muzzle in between the
jaws on one side, and was trying to get it out between the jaws on the
other side. This accomplished, he pried gently and carefully, loosening
the jaws a bit at a time, while Matt, a bit at a time, extricated White
Fang’s mangled neck.

“Stand by to receive your dog,” was Scott’s peremptory order to
Cherokee’s owner.

The faro-dealer stooped down obediently and got a firm hold on
Cherokee.

“Now!” Scott warned, giving the final pry.

The dogs were drawn apart, the bull-dog struggling vigorously.

“Take him away,” Scott commanded, and Tim Keenan dragged Cherokee back
into the crowd.

White Fang made several ineffectual efforts to get up. Once he gained
his feet, but his legs were too weak to sustain him, and he slowly
wilted and sank back into the snow. His eyes were half closed, and the
surface of them was glassy. His jaws were apart, and through them the
tongue protruded, draggled and limp. To all appearances he looked like
a dog that had been strangled to death. Matt examined him.

“Just about all in,” he announced; “but he’s breathin’ all right.”

Beauty Smith had regained his feet and come over to look at White Fang.

“Matt, how much is a good sled-dog worth?” Scott asked.

The dog-musher, still on his knees and stooped over White Fang,
calculated for a moment.

“Three hundred dollars,” he answered.

“And how much for one that’s all chewed up like this one?” Scott asked,
nudging White Fang with his foot.

“Half of that,” was the dog-musher’s judgment. Scott turned upon Beauty
Smith.

“Did you hear, Mr. Beast? I’m going to take your dog from you, and I’m
going to give you a hundred and fifty for him.”

He opened his pocket-book and counted out the bills.

Beauty Smith put his hands behind his back, refusing to touch the
proffered money.

“I ain’t a-sellin’,” he said.

“Oh, yes you are,” the other assured him. “Because I’m buying. Here’s
your money. The dog’s mine.”

Beauty Smith, his hands still behind him, began to back away.

Scott sprang toward him, drawing his fist back to strike. Beauty Smith
cowered down in anticipation of the blow.

“I’ve got my rights,” he whimpered.

“You’ve forfeited your rights to own that dog,” was the rejoinder. “Are
you going to take the money? or do I have to hit you again?”

“All right,” Beauty Smith spoke up with the alacrity of fear. “But I
take the money under protest,” he added. “The dog’s a mint. I ain’t
a-goin’ to be robbed. A man’s got his rights.”

“Correct,” Scott answered, passing the money over to him. “A man’s got
his rights. But you’re not a man. You’re a beast.”

“Wait till I get back to Dawson,” Beauty Smith threatened. “I’ll have
the law on you.”

“If you open your mouth when you get back to Dawson, I’ll have you run
out of town. Understand?”

Beauty Smith replied with a grunt.

“Understand?” the other thundered with abrupt fierceness.

“Yes,” Beauty Smith grunted, shrinking away.

“Yes what?”

“Yes, sir,” Beauty Smith snarled.

“Look out! He’ll bite!” some one shouted, and a guffaw of laughter went
up.

Scott turned his back on him, and returned to help the dog-musher, who
was working over White Fang.

Some of the men were already departing; others stood in groups, looking
on and talking. Tim Keenan joined one of the groups.

“Who’s that mug?” he asked.

“Weedon Scott,” some one answered.

“And who in hell is Weedon Scott?” the faro-dealer demanded.

“Oh, one of them crackerjack minin’ experts. He’s in with all the big
bugs. If you want to keep out of trouble, you’ll steer clear of him,
that’s my talk. He’s all hunky with the officials. The Gold
Commissioner’s a special pal of his.”

“I thought he must be somebody,” was the faro-dealer’s comment. “That’s
why I kept my hands offen him at the start.”

\chapter{The Indomitable}

“It’s hopeless,” Weedon Scott confessed.

He sat on the step of his cabin and stared at the dog-musher, who
responded with a shrug that was equally hopeless.

Together they looked at White Fang at the end of his stretched chain,
bristling, snarling, ferocious, straining to get at the sled-dogs.
Having received sundry lessons from Matt, said lessons being imparted
by means of a club, the sled-dogs had learned to leave White Fang
alone; and even then they were lying down at a distance, apparently
oblivious of his existence.

“It’s a wolf and there’s no taming it,” Weedon Scott announced.

“Oh, I don’t know about that,” Matt objected. “Might be a lot of dog in
’m, for all you can tell. But there’s one thing I know sure, an’ that
there’s no gettin’ away from.”

The dog-musher paused and nodded his head confidentially at Moosehide
Mountain.

“Well, don’t be a miser with what you know,” Scott said sharply, after
waiting a suitable length of time. “Spit it out. What is it?”

The dog-musher indicated White Fang with a backward thrust of his
thumb.

“Wolf or dog, it’s all the same—he’s ben tamed ’ready.”

“No!”

“I tell you yes, an’ broke to harness. Look close there. D’ye see them
marks across the chest?”

“You’re right, Matt. He was a sled-dog before Beauty Smith got hold of
him.”

“And there’s not much reason against his bein’ a sled-dog again.”

“What d’ye think?” Scott queried eagerly. Then the hope died down as he
added, shaking his head, “We’ve had him two weeks now, and if anything
he’s wilder than ever at the present moment.”

“Give ’m a chance,” Matt counselled. “Turn ’m loose for a spell.”

The other looked at him incredulously.

“Yes,” Matt went on, “I know you’ve tried to, but you didn’t take a
club.”

“You try it then.”

The dog-musher secured a club and went over to the chained animal.
White Fang watched the club after the manner of a caged lion watching
the whip of its trainer.

“See ’m keep his eye on that club,” Matt said. “That’s a good sign.
He’s no fool. Don’t dast tackle me so long as I got that club handy.
He’s not clean crazy, sure.”

As the man’s hand approached his neck, White Fang bristled and snarled
and crouched down. But while he eyed the approaching hand, he at the
same time contrived to keep track of the club in the other hand,
suspended threateningly above him. Matt unsnapped the chain from the
collar and stepped back.

White Fang could scarcely realise that he was free. Many months had
gone by since he passed into the possession of Beauty Smith, and in all
that period he had never known a moment of freedom except at the times
he had been loosed to fight with other dogs. Immediately after such
fights he had always been imprisoned again.

He did not know what to make of it. Perhaps some new devilry of the
gods was about to be perpetrated on him. He walked slowly and
cautiously, prepared to be assailed at any moment. He did not know what
to do, it was all so unprecedented. He took the precaution to sheer off
from the two watching gods, and walked carefully to the corner of the
cabin. Nothing happened. He was plainly perplexed, and he came back
again, pausing a dozen feet away and regarding the two men intently.

“Won’t he run away?” his new owner asked.

Matt shrugged his shoulders. “Got to take a gamble. Only way to find
out is to find out.”

“Poor devil,” Scott murmured pityingly. “What he needs is some show of
human kindness,” he added, turning and going into the cabin.

He came out with a piece of meat, which he tossed to White Fang. He
sprang away from it, and from a distance studied it suspiciously.

“Hi-yu, Major!” Matt shouted warningly, but too late.

Major had made a spring for the meat. At the instant his jaws closed on
it, White Fang struck him. He was overthrown. Matt rushed in, but
quicker than he was White Fang. Major staggered to his feet, but the
blood spouting from his throat reddened the snow in a widening path.

“It’s too bad, but it served him right,” Scott said hastily.

But Matt’s foot had already started on its way to kick White Fang.
There was a leap, a flash of teeth, a sharp exclamation. White Fang,
snarling fiercely, scrambled backward for several yards, while Matt
stooped and investigated his leg.

“He got me all right,” he announced, pointing to the torn trousers and
undercloths, and the growing stain of red.

“I told you it was hopeless, Matt,” Scott said in a discouraged voice.
“I’ve thought about it off and on, while not wanting to think of it.
But we’ve come to it now. It’s the only thing to do.”

As he talked, with reluctant movements he drew his revolver, threw open
the cylinder, and assured himself of its contents.

“Look here, Mr. Scott,” Matt objected; “that dog’s ben through hell.
You can’t expect ’m to come out a white an’ shinin’ angel. Give ’m
time.”

“Look at Major,” the other rejoined.

The dog-musher surveyed the stricken dog. He had sunk down on the snow
in the circle of his blood and was plainly in the last gasp.

“Served ’m right. You said so yourself, Mr. Scott. He tried to take
White Fang’s meat, an’ he’s dead-O. That was to be expected. I wouldn’t
give two whoops in hell for a dog that wouldn’t fight for his own
meat.”

“But look at yourself, Matt. It’s all right about the dogs, but we must
draw the line somewhere.”

“Served me right,” Matt argued stubbornly. “What’d I want to kick ’m
for? You said yourself that he’d done right. Then I had no right to
kick ’m.”

“It would be a mercy to kill him,” Scott insisted. “He’s untamable.”

“Now look here, Mr. Scott, give the poor devil a fightin’ chance. He
ain’t had no chance yet. He’s just come through hell, an’ this is the
first time he’s ben loose. Give ’m a fair chance, an’ if he don’t
deliver the goods, I’ll kill ’m myself. There!”

“God knows I don’t want to kill him or have him killed,” Scott
answered, putting away the revolver. “We’ll let him run loose and see
what kindness can do for him. And here’s a try at it.”

He walked over to White Fang and began talking to him gently and
soothingly.

“Better have a club handy,” Matt warned.

Scott shook his head and went on trying to win White Fang’s confidence.

White Fang was suspicious. Something was impending. He had killed this
god’s dog, bitten his companion god, and what else was to be expected
than some terrible punishment? But in the face of it he was
indomitable. He bristled and showed his teeth, his eyes vigilant, his
whole body wary and prepared for anything. The god had no club, so he
suffered him to approach quite near. The god’s hand had come out and
was descending upon his head. White Fang shrank together and grew tense
as he crouched under it. Here was danger, some treachery or something.
He knew the hands of the gods, their proved mastery, their cunning to
hurt. Besides, there was his old antipathy to being touched. He snarled
more menacingly, crouched still lower, and still the hand descended. He
did not want to bite the hand, and he endured the peril of it until his
instinct surged up in him, mastering him with its insatiable yearning
for life.

Weedon Scott had believed that he was quick enough to avoid any snap or
slash. But he had yet to learn the remarkable quickness of White Fang,
who struck with the certainty and swiftness of a coiled snake.

Scott cried out sharply with surprise, catching his torn hand and
holding it tightly in his other hand. Matt uttered a great oath and
sprang to his side. White Fang crouched down, and backed away,
bristling, showing his fangs, his eyes malignant with menace. Now he
could expect a beating as fearful as any he had received from Beauty
Smith.

“Here! What are you doing?” Scott cried suddenly.

Matt had dashed into the cabin and come out with a rifle.

“Nothin’,” he said slowly, with a careless calmness that was assumed,
“only goin’ to keep that promise I made. I reckon it’s up to me to kill
’m as I said I’d do.”

“No you don’t!”

“Yes I do. Watch me.”

As Matt had pleaded for White Fang when he had been bitten, it was now
Weedon Scott’s turn to plead.

“You said to give him a chance. Well, give it to him. We’ve only just
started, and we can’t quit at the beginning. It served me right, this
time. And—look at him!”

White Fang, near the corner of the cabin and forty feet away, was
snarling with blood-curdling viciousness, not at Scott, but at the
dog-musher.

“Well, I’ll be everlastingly gosh-swoggled!” was the dog-musher’s
expression of astonishment.

“Look at the intelligence of him,” Scott went on hastily. “He knows the
meaning of firearms as well as you do. He’s got intelligence and we’ve
got to give that intelligence a chance. Put up the gun.”

“All right, I’m willin’,” Matt agreed, leaning the rifle against the
woodpile.

“But will you look at that!” he exclaimed the next moment.

White Fang had quieted down and ceased snarling. “This is worth
investigatin’. Watch.”

Matt, reached for the rifle, and at the same moment White Fang snarled.
He stepped away from the rifle, and White Fang’s lifted lips descended,
covering his teeth.

“Now, just for fun.”

Matt took the rifle and began slowly to raise it to his shoulder. White
Fang’s snarling began with the movement, and increased as the movement
approached its culmination. But the moment before the rifle came to a
level on him, he leaped sidewise behind the corner of the cabin. Matt
stood staring along the sights at the empty space of snow which had
been occupied by White Fang.

The dog-musher put the rifle down solemnly, then turned and looked at
his employer.

“I agree with you, Mr. Scott. That dog’s too intelligent to kill.”

\chapter{The Love-Master}

As White Fang watched Weedon Scott approach, he bristled and snarled to
advertise that he would not submit to punishment. Twenty-four hours had
passed since he had slashed open the hand that was now bandaged and
held up by a sling to keep the blood out of it. In the past White Fang
had experienced delayed punishments, and he apprehended that such a one
was about to befall him. How could it be otherwise? He had committed
what was to him sacrilege, sunk his fangs into the holy flesh of a god,
and of a white-skinned superior god at that. In the nature of things,
and of intercourse with gods, something terrible awaited him.

The god sat down several feet away. White Fang could see nothing
dangerous in that. When the gods administered punishment they stood on
their legs. Besides, this god had no club, no whip, no firearm. And
furthermore, he himself was free. No chain nor stick bound him. He
could escape into safety while the god was scrambling to his feet. In
the meantime he would wait and see.

The god remained quiet, made no movement; and White Fang’s snarl slowly
dwindled to a growl that ebbed down in his throat and ceased. Then the
god spoke, and at the first sound of his voice, the hair rose on White
Fang’s neck and the growl rushed up in his throat. But the god made no
hostile movement, and went on calmly talking. For a time White Fang
growled in unison with him, a correspondence of rhythm being
established between growl and voice. But the god talked on
interminably. He talked to White Fang as White Fang had never been
talked to before. He talked softly and soothingly, with a gentleness
that somehow, somewhere, touched White Fang. In spite of himself and
all the pricking warnings of his instinct, White Fang began to have
confidence in this god. He had a feeling of security that was belied by
all his experience with men.

After a long time, the god got up and went into the cabin. White Fang
scanned him apprehensively when he came out. He had neither whip nor
club nor weapon. Nor was his uninjured hand behind his back hiding
something. He sat down as before, in the same spot, several feet away.
He held out a small piece of meat. White Fang pricked his ears and
investigated it suspiciously, managing to look at the same time both at
the meat and the god, alert for any overt act, his body tense and ready
to spring away at the first sign of hostility.

Still the punishment delayed. The god merely held near to his nose a
piece of meat. And about the meat there seemed nothing wrong. Still
White Fang suspected; and though the meat was proffered to him with
short inviting thrusts of the hand, he refused to touch it. The gods
were all-wise, and there was no telling what masterful treachery lurked
behind that apparently harmless piece of meat. In past experience,
especially in dealing with squaws, meat and punishment had often been
disastrously related.

In the end, the god tossed the meat on the snow at White Fang’s feet.
He smelled the meat carefully; but he did not look at it. While he
smelled it he kept his eyes on the god. Nothing happened. He took the
meat into his mouth and swallowed it. Still nothing happened. The god
was actually offering him another piece of meat. Again he refused to
take it from the hand, and again it was tossed to him. This was
repeated a number of times. But there came a time when the god refused
to toss it. He kept it in his hand and steadfastly proffered it.

The meat was good meat, and White Fang was hungry. Bit by bit,
infinitely cautious, he approached the hand. At last the time came that
he decided to eat the meat from the hand. He never took his eyes from
the god, thrusting his head forward with ears flattened back and hair
involuntarily rising and cresting on his neck. Also a low growl rumbled
in his throat as warning that he was not to be trifled with. He ate the
meat, and nothing happened. Piece by piece, he ate all the meat, and
nothing happened. Still the punishment delayed.

He licked his chops and waited. The god went on talking. In his voice
was kindness—something of which White Fang had no experience whatever.
And within him it aroused feelings which he had likewise never
experienced before. He was aware of a certain strange satisfaction, as
though some need were being gratified, as though some void in his being
were being filled. Then again came the prod of his instinct and the
warning of past experience. The gods were ever crafty, and they had
unguessed ways of attaining their ends.

Ah, he had thought so! There it came now, the god’s hand, cunning to
hurt, thrusting out at him, descending upon his head. But the god went
on talking. His voice was soft and soothing. In spite of the menacing
hand, the voice inspired confidence. And in spite of the assuring
voice, the hand inspired distrust. White Fang was torn by conflicting
feelings, impulses. It seemed he would fly to pieces, so terrible was
the control he was exerting, holding together by an unwonted indecision
the counter-forces that struggled within him for mastery.

He compromised. He snarled and bristled and flattened his ears. But he
neither snapped nor sprang away. The hand descended. Nearer and nearer
it came. It touched the ends of his upstanding hair. He shrank down
under it. It followed down after him, pressing more closely against
him. Shrinking, almost shivering, he still managed to hold himself
together. It was a torment, this hand that touched him and violated his
instinct. He could not forget in a day all the evil that had been
wrought him at the hands of men. But it was the will of the god, and he
strove to submit.

The hand lifted and descended again in a patting, caressing movement.
This continued, but every time the hand lifted, the hair lifted under
it. And every time the hand descended, the ears flattened down and a
cavernous growl surged in his throat. White Fang growled and growled
with insistent warning. By this means he announced that he was prepared
to retaliate for any hurt he might receive. There was no telling when
the god’s ulterior motive might be disclosed. At any moment that soft,
confidence-inspiring voice might break forth in a roar of wrath, that
gentle and caressing hand transform itself into a vice-like grip to
hold him helpless and administer punishment.

But the god talked on softly, and ever the hand rose and fell with
non-hostile pats. White Fang experienced dual feelings. It was
distasteful to his instinct. It restrained him, opposed the will of him
toward personal liberty. And yet it was not physically painful. On the
contrary, it was even pleasant, in a physical way. The patting movement
slowly and carefully changed to a rubbing of the ears about their
bases, and the physical pleasure even increased a little. Yet he
continued to fear, and he stood on guard, expectant of unguessed evil,
alternately suffering and enjoying as one feeling or the other came
uppermost and swayed him.

“Well, I’ll be gosh-swoggled!”

So spoke Matt, coming out of the cabin, his sleeves rolled up, a pan of
dirty dish-water in his hands, arrested in the act of emptying the pan
by the sight of Weedon Scott patting White Fang.

At the instant his voice broke the silence, White Fang leaped back,
snarling savagely at him.

Matt regarded his employer with grieved disapproval.

“If you don’t mind my expressin’ my feelin’s, Mr. Scott, I’ll make free
to say you’re seventeen kinds of a damn fool an’ all of ’em different,
an’ then some.”

Weedon Scott smiled with a superior air, gained his feet, and walked
over to White Fang. He talked soothingly to him, but not for long, then
slowly put out his hand, rested it on White Fang’s head, and resumed
the interrupted patting. White Fang endured it, keeping his eyes fixed
suspiciously, not upon the man that patted him, but upon the man that
stood in the doorway.

“You may be a number one, tip-top minin’ expert, all right all right,”
the dog-musher delivered himself oracularly, “but you missed the chance
of your life when you was a boy an’ didn’t run off an’ join a circus.”

White Fang snarled at the sound of his voice, but this time did not
leap away from under the hand that was caressing his head and the back
of his neck with long, soothing strokes.

It was the beginning of the end for White Fang—the ending of the old
life and the reign of hate. A new and incomprehensibly fairer life was
dawning. It required much thinking and endless patience on the part of
Weedon Scott to accomplish this. And on the part of White Fang it
required nothing less than a revolution. He had to ignore the urges and
promptings of instinct and reason, defy experience, give the lie to
life itself.

Life, as he had known it, not only had had no place in it for much that
he now did; but all the currents had gone counter to those to which he
now abandoned himself. In short, when all things were considered, he
had to achieve an orientation far vaster than the one he had achieved
at the time he came voluntarily in from the Wild and accepted Grey
Beaver as his lord. At that time he was a mere puppy, soft from the
making, without form, ready for the thumb of circumstance to begin its
work upon him. But now it was different. The thumb of circumstance had
done its work only too well. By it he had been formed and hardened into
the Fighting Wolf, fierce and implacable, unloving and unlovable. To
accomplish the change was like a reflux of being, and this when the
plasticity of youth was no longer his; when the fibre of him had become
tough and knotty; when the warp and the woof of him had made of him an
adamantine texture, harsh and unyielding; when the face of his spirit
had become iron and all his instincts and axioms had crystallised into
set rules, cautions, dislikes, and desires.

Yet again, in this new orientation, it was the thumb of circumstance
that pressed and prodded him, softening that which had become hard and
remoulding it into fairer form. Weedon Scott was in truth this thumb.
He had gone to the roots of White Fang’s nature, and with kindness
touched to life potencies that had languished and well-nigh perished.
One such potency was \emph{love}. It took the place of \emph{like}, which latter
had been the highest feeling that thrilled him in his intercourse with
the gods.

But this love did not come in a day. It began with \emph{like} and out of it
slowly developed. White Fang did not run away, though he was allowed to
remain loose, because he liked this new god. This was certainly better
than the life he had lived in the cage of Beauty Smith, and it was
necessary that he should have some god. The lordship of man was a need
of his nature. The seal of his dependence on man had been set upon him
in that early day when he turned his back on the Wild and crawled to
Grey Beaver’s feet to receive the expected beating. This seal had been
stamped upon him again, and ineradicably, on his second return from the
Wild, when the long famine was over and there was fish once more in the
village of Grey Beaver.

And so, because he needed a god and because he preferred Weedon Scott
to Beauty Smith, White Fang remained. In acknowledgment of fealty, he
proceeded to take upon himself the guardianship of his master’s
property. He prowled about the cabin while the sled-dogs slept, and the
first night-visitor to the cabin fought him off with a club until
Weedon Scott came to the rescue. But White Fang soon learned to
differentiate between thieves and honest men, to appraise the true
value of step and carriage. The man who travelled, loud-stepping, the
direct line to the cabin door, he let alone—though he watched him
vigilantly until the door opened and he received the endorsement of the
master. But the man who went softly, by circuitous ways, peering with
caution, seeking after secrecy—that was the man who received no
suspension of judgment from White Fang, and who went away abruptly,
hurriedly, and without dignity.

Weedon Scott had set himself the task of redeeming White Fang—or
rather, of redeeming mankind from the wrong it had done White Fang. It
was a matter of principle and conscience. He felt that the ill done
White Fang was a debt incurred by man and that it must be paid. So he
went out of his way to be especially kind to the Fighting Wolf. Each
day he made it a point to caress and pet White Fang, and to do it at
length.

At first suspicious and hostile, White Fang grew to like this petting.
But there was one thing that he never outgrew—his growling. Growl he
would, from the moment the petting began till it ended. But it was a
growl with a new note in it. A stranger could not hear this note, and
to such a stranger the growling of White Fang was an exhibition of
primordial savagery, nerve-racking and blood-curdling. But White Fang’s
throat had become harsh-fibred from the making of ferocious sounds
through the many years since his first little rasp of anger in the lair
of his cubhood, and he could not soften the sounds of that throat now
to express the gentleness he felt. Nevertheless, Weedon Scott’s ear and
sympathy were fine enough to catch the new note all but drowned in the
fierceness—the note that was the faintest hint of a croon of content
and that none but he could hear.

As the days went by, the evolution of \emph{like} into \emph{love} was
accelerated. White Fang himself began to grow aware of it, though in
his consciousness he knew not what love was. It manifested itself to
him as a void in his being—a hungry, aching, yearning void that
clamoured to be filled. It was a pain and an unrest; and it received
easement only by the touch of the new god’s presence. At such times
love was joy to him, a wild, keen-thrilling satisfaction. But when away
from his god, the pain and the unrest returned; the void in him sprang
up and pressed against him with its emptiness, and the hunger gnawed
and gnawed unceasingly.

White Fang was in the process of finding himself. In spite of the
maturity of his years and of the savage rigidity of the mould that had
formed him, his nature was undergoing an expansion. There was a
burgeoning within him of strange feelings and unwonted impulses. His
old code of conduct was changing. In the past he had liked comfort and
surcease from pain, disliked discomfort and pain, and he had adjusted
his actions accordingly. But now it was different. Because of this new
feeling within him, he ofttimes elected discomfort and pain for the
sake of his god. Thus, in the early morning, instead of roaming and
foraging, or lying in a sheltered nook, he would wait for hours on the
cheerless cabin-stoop for a sight of the god’s face. At night, when the
god returned home, White Fang would leave the warm sleeping-place he
had burrowed in the snow in order to receive the friendly snap of
fingers and the word of greeting. Meat, even meat itself, he would
forego to be with his god, to receive a caress from him or to accompany
him down into the town.

\emph{Like} had been replaced by \emph{love}. And love was the plummet dropped
down into the deeps of him where like had never gone. And responsive
out of his deeps had come the new thing—love. That which was given unto
him did he return. This was a god indeed, a love-god, a warm and
radiant god, in whose light White Fang’s nature expanded as a flower
expands under the sun.

But White Fang was not demonstrative. He was too old, too firmly
moulded, to become adept at expressing himself in new ways. He was too
self-possessed, too strongly poised in his own isolation. Too long had
he cultivated reticence, aloofness, and moroseness. He had never barked
in his life, and he could not now learn to bark a welcome when his god
approached. He was never in the way, never extravagant nor foolish in
the expression of his love. He never ran to meet his god. He waited at
a distance; but he always waited, was always there. His love partook of
the nature of worship, dumb, inarticulate, a silent adoration. Only by
the steady regard of his eyes did he express his love, and by the
unceasing following with his eyes of his god’s every movement. Also, at
times, when his god looked at him and spoke to him, he betrayed an
awkward self-consciousness, caused by the struggle of his love to
express itself and his physical inability to express it.

He learned to adjust himself in many ways to his new mode of life. It
was borne in upon him that he must let his master’s dogs alone. Yet his
dominant nature asserted itself, and he had first to thrash them into
an acknowledgment of his superiority and leadership. This accomplished,
he had little trouble with them. They gave trail to him when he came
and went or walked among them, and when he asserted his will they
obeyed.

In the same way, he came to tolerate Matt—as a possession of his
master. His master rarely fed him. Matt did that, it was his business;
yet White Fang divined that it was his master’s food he ate and that it
was his master who thus fed him vicariously. Matt it was who tried to
put him into the harness and make him haul sled with the other dogs.
But Matt failed. It was not until Weedon Scott put the harness on White
Fang and worked him, that he understood. He took it as his master’s
will that Matt should drive him and work him just as he drove and
worked his master’s other dogs.

Different from the Mackenzie toboggans were the Klondike sleds with
runners under them. And different was the method of driving the dogs.
There was no fan-formation of the team. The dogs worked in single file,
one behind another, hauling on double traces. And here, in the
Klondike, the leader was indeed the leader. The wisest as well as
strongest dog was the leader, and the team obeyed him and feared him.
That White Fang should quickly gain this post was inevitable. He could
not be satisfied with less, as Matt learned after much inconvenience
and trouble. White Fang picked out the post for himself, and Matt
backed his judgment with strong language after the experiment had been
tried. But, though he worked in the sled in the day, White Fang did not
forego the guarding of his master’s property in the night. Thus he was
on duty all the time, ever vigilant and faithful, the most valuable of
all the dogs.

“Makin’ free to spit out what’s in me,” Matt said one day, “I beg to
state that you was a wise guy all right when you paid the price you did
for that dog. You clean swindled Beauty Smith on top of pushin’ his
face in with your fist.”

A recrudescence of anger glinted in Weedon Scott’s grey eyes, and he
muttered savagely, “The beast!”

In the late spring a great trouble came to White Fang. Without warning,
the love-master disappeared. There had been warning, but White Fang was
unversed in such things and did not understand the packing of a grip.
He remembered afterwards that his packing had preceded the master’s
disappearance; but at the time he suspected nothing. That night he
waited for the master to return. At midnight the chill wind that blew
drove him to shelter at the rear of the cabin. There he drowsed, only
half asleep, his ears keyed for the first sound of the familiar step.
But, at two in the morning, his anxiety drove him out to the cold front
stoop, where he crouched, and waited.

But no master came. In the morning the door opened and Matt stepped
outside. White Fang gazed at him wistfully. There was no common speech
by which he might learn what he wanted to know. The days came and went,
but never the master. White Fang, who had never known sickness in his
life, became sick. He became very sick, so sick that Matt was finally
compelled to bring him inside the cabin. Also, in writing to his
employer, Matt devoted a postscript to White Fang.

Weedon Scott reading the letter down in Circle City, came upon the
following:

“That dam wolf won’t work. Won’t eat. Aint got no spunk left. All the
dogs is licking him. Wants to know what has become of you, and I don’t
know how to tell him. Mebbe he is going to die.”

It was as Matt had said. White Fang had ceased eating, lost heart, and
allowed every dog of the team to thrash him. In the cabin he lay on the
floor near the stove, without interest in food, in Matt, nor in life.
Matt might talk gently to him or swear at him, it was all the same; he
never did more than turn his dull eyes upon the man, then drop his head
back to its customary position on his fore-paws.

And then, one night, Matt, reading to himself with moving lips and
mumbled sounds, was startled by a low whine from White Fang. He had got
upon his feet, his ears cocked towards the door, and he was listening
intently. A moment later, Matt heard a footstep. The door opened, and
Weedon Scott stepped in. The two men shook hands. Then Scott looked
around the room.

“Where’s the wolf?” he asked.

Then he discovered him, standing where he had been lying, near to the
stove. He had not rushed forward after the manner of other dogs. He
stood, watching and waiting.

“Holy smoke!” Matt exclaimed. “Look at ’m wag his tail!”

Weedon Scott strode half across the room toward him, at the same time
calling him. White Fang came to him, not with a great bound, yet
quickly. He was awakened from self-consciousness, but as he drew near,
his eyes took on a strange expression. Something, an incommunicable
vastness of feeling, rose up into his eyes as a light and shone forth.

“He never looked at me that way all the time you was gone!” Matt
commented.

Weedon Scott did not hear. He was squatting down on his heels, face to
face with White Fang and petting him—rubbing at the roots of the ears,
making long caressing strokes down the neck to the shoulders, tapping
the spine gently with the balls of his fingers. And White Fang was
growling responsively, the crooning note of the growl more pronounced
than ever.

But that was not all. What of his joy, the great love in him, ever
surging and struggling to express itself, succeeded in finding a new
mode of expression. He suddenly thrust his head forward and nudged his
way in between the master’s arm and body. And here, confined, hidden
from view all except his ears, no longer growling, he continued to
nudge and snuggle.

The two men looked at each other. Scott’s eyes were shining.

“Gosh!” said Matt in an awe-stricken voice.

A moment later, when he had recovered himself, he said, “I always
insisted that wolf was a dog. Look at ’m!”

With the return of the love-master, White Fang’s recovery was rapid.
Two nights and a day he spent in the cabin. Then he sallied forth. The
sled-dogs had forgotten his prowess. They remembered only the latest,
which was his weakness and sickness. At the sight of him as he came out
of the cabin, they sprang upon him.

“Talk about your rough-houses,” Matt murmured gleefully, standing in
the doorway and looking on.

“Give ’m hell, you wolf! Give ’m hell!—an’ then some!”

White Fang did not need the encouragement. The return of the
love-master was enough. Life was flowing through him again, splendid
and indomitable. He fought from sheer joy, finding in it an expression
of much that he felt and that otherwise was without speech. There could
be but one ending. The team dispersed in ignominious defeat, and it was
not until after dark that the dogs came sneaking back, one by one, by
meekness and humility signifying their fealty to White Fang.

Having learned to snuggle, White Fang was guilty of it often. It was
the final word. He could not go beyond it. The one thing of which he
had always been particularly jealous was his head. He had always
disliked to have it touched. It was the Wild in him, the fear of hurt
and of the trap, that had given rise to the panicky impulses to avoid
contacts. It was the mandate of his instinct that that head must be
free. And now, with the love-master, his snuggling was the deliberate
act of putting himself into a position of hopeless helplessness. It was
an expression of perfect confidence, of absolute self-surrender, as
though he said: “I put myself into thy hands. Work thou thy will with
me.”

One night, not long after the return, Scott and Matt sat at a game of
cribbage preliminary to going to bed. “Fifteen-two, fifteen-four an’ a
pair makes six,” Mat was pegging up, when there was an outcry and sound
of snarling without. They looked at each other as they started to rise
to their feet.

“The wolf’s nailed somebody,” Matt said.

A wild scream of fear and anguish hastened them.

“Bring a light!” Scott shouted, as he sprang outside.

Matt followed with the lamp, and by its light they saw a man lying on
his back in the snow. His arms were folded, one above the other, across
his face and throat. Thus he was trying to shield himself from White
Fang’s teeth. And there was need for it. White Fang was in a rage,
wickedly making his attack on the most vulnerable spot. From shoulder
to wrist of the crossed arms, the coat-sleeve, blue flannel shirt and
undershirt were ripped in rags, while the arms themselves were terribly
slashed and streaming blood.

All this the two men saw in the first instant. The next instant Weedon
Scott had White Fang by the throat and was dragging him clear. White
Fang struggled and snarled, but made no attempt to bite, while he
quickly quieted down at a sharp word from the master.

Matt helped the man to his feet. As he arose he lowered his crossed
arms, exposing the bestial face of Beauty Smith. The dog-musher let go
of him precipitately, with action similar to that of a man who has
picked up live fire. Beauty Smith blinked in the lamplight and looked
about him. He caught sight of White Fang and terror rushed into his
face.

At the same moment Matt noticed two objects lying in the snow. He held
the lamp close to them, indicating them with his toe for his employer’s
benefit—a steel dog-chain and a stout club.

Weedon Scott saw and nodded. Not a word was spoken. The dog-musher laid
his hand on Beauty Smith’s shoulder and faced him to the right about.
No word needed to be spoken. Beauty Smith started.

In the meantime the love-master was patting White Fang and talking to
him.

“Tried to steal you, eh? And you wouldn’t have it! Well, well, he made
a mistake, didn’t he?”

“Must ‘a’ thought he had hold of seventeen devils,” the dog-musher
sniggered.

White Fang, still wrought up and bristling, growled and growled, the
hair slowly lying down, the crooning note remote and dim, but growing
in his throat.

%\part{}
\chapter{The Long Trail}

It was in the air. White Fang sensed the coming calamity, even before
there was tangible evidence of it. In vague ways it was borne in upon
him that a change was impending. He knew not how nor why, yet he got
his feel of the oncoming event from the gods themselves. In ways
subtler than they knew, they betrayed their intentions to the wolf-dog
that haunted the cabin-stoop, and that, though he never came inside the
cabin, knew what went on inside their brains.

“Listen to that, will you!” the dog-musher exclaimed at supper one
night.

Weedon Scott listened. Through the door came a low, anxious whine, like
a sobbing under the breath that had just grown audible. Then came the
long sniff, as White Fang reassured himself that his god was still
inside and had not yet taken himself off in mysterious and solitary
flight.

“I do believe that wolf’s on to you,” the dog-musher said.

Weedon Scott looked across at his companion with eyes that almost
pleaded, though this was given the lie by his words.

“What the devil can I do with a wolf in California?” he demanded.

“That’s what I say,” Matt answered. “What the devil can you do with a
wolf in California?”

But this did not satisfy Weedon Scott. The other seemed to be judging
him in a non-committal sort of way.

“White man’s dogs would have no show against him,” Scott went on. “He’d
kill them on sight. If he didn’t bankrupt me with damaged suits, the
authorities would take him away from me and electrocute him.”

“He’s a downright murderer, I know,” was the dog-musher’s comment.

Weedon Scott looked at him suspiciously.

“It would never do,” he said decisively.

“It would never do!” Matt concurred. “Why you’d have to hire a man
’specially to take care of ’m.”

The other’s suspicion was allayed. He nodded cheerfully. In the silence
that followed, the low, half-sobbing whine was heard at the door and
then the long, questing sniff.

“There’s no denyin’ he thinks a hell of a lot of you,” Matt said.

The other glared at him in sudden wrath. “Damn it all, man! I know my
own mind and what’s best!”

“I’m agreein’ with you, only . . . ”

“Only what?” Scott snapped out.

“Only . . . ” the dog-musher began softly, then changed his mind and
betrayed a rising anger of his own. “Well, you needn’t get so all-fired
het up about it. Judgin’ by your actions one’d think you didn’t know
your own mind.”

Weedon Scott debated with himself for a while, and then said more
gently: “You are right, Matt. I don’t know my own mind, and that’s
what’s the trouble.”

“Why, it would be rank ridiculousness for me to take that dog along,”
he broke out after another pause.

“I’m agreein’ with you,” was Matt’s answer, and again his employer was
not quite satisfied with him.

“But how in the name of the great Sardanapolis he knows you’re goin’ is
what gets me,” the dog-musher continued innocently.

“It’s beyond me, Matt,” Scott answered, with a mournful shake of the
head.

Then came the day when, through the open cabin door, White Fang saw the
fatal grip on the floor and the love-master packing things into it.
Also, there were comings and goings, and the erstwhile placid
atmosphere of the cabin was vexed with strange perturbations and
unrest. Here was indubitable evidence. White Fang had already scented
it. He now reasoned it. His god was preparing for another flight. And
since he had not taken him with him before, so, now, he could look to
be left behind.

That night he lifted the long wolf-howl. As he had howled, in his puppy
days, when he fled back from the Wild to the village to find it
vanished and naught but a rubbish-heap to mark the site of Grey
Beaver’s tepee, so now he pointed his muzzle to the cold stars and told
to them his woe.

Inside the cabin the two men had just gone to bed.

“He’s gone off his food again,” Matt remarked from his bunk.

There was a grunt from Weedon Scott’s bunk, and a stir of blankets.

“From the way he cut up the other time you went away, I wouldn’t wonder
this time but what he died.”

The blankets in the other bunk stirred irritably.

“Oh, shut up!” Scott cried out through the darkness. “You nag worse
than a woman.”

“I’m agreein’ with you,” the dog-musher answered, and Weedon Scott was
not quite sure whether or not the other had snickered.

The next day White Fang’s anxiety and restlessness were even more
pronounced. He dogged his master’s heels whenever he left the cabin,
and haunted the front stoop when he remained inside. Through the open
door he could catch glimpses of the luggage on the floor. The grip had
been joined by two large canvas bags and a box. Matt was rolling the
master’s blankets and fur robe inside a small tarpaulin. White Fang
whined as he watched the operation.

Later on two Indians arrived. He watched them closely as they
shouldered the luggage and were led off down the hill by Matt, who
carried the bedding and the grip. But White Fang did not follow them.
The master was still in the cabin. After a time, Matt returned. The
master came to the door and called White Fang inside.

“You poor devil,” he said gently, rubbing White Fang’s ears and tapping
his spine. “I’m hitting the long trail, old man, where you cannot
follow. Now give me a growl—the last, good, good-bye growl.”

But White Fang refused to growl. Instead, and after a wistful,
searching look, he snuggled in, burrowing his head out of sight between
the master’s arm and body.

“There she blows!” Matt cried. From the Yukon arose the hoarse
bellowing of a river steamboat. “You’ve got to cut it short. Be sure
and lock the front door. I’ll go out the back. Get a move on!”

The two doors slammed at the same moment, and Weedon Scott waited for
Matt to come around to the front. From inside the door came a low
whining and sobbing. Then there were long, deep-drawn sniffs.

“You must take good care of him, Matt,” Scott said, as they started
down the hill. “Write and let me know how he gets along.”

“Sure,” the dog-musher answered. “But listen to that, will you!”

Both men stopped. White Fang was howling as dogs howl when their
masters lie dead. He was voicing an utter woe, his cry bursting upward
in great heart-breaking rushes, dying down into quavering misery, and
bursting upward again with a rush upon rush of grief.

The \emph{Aurora} was the first steamboat of the year for the Outside, and
her decks were jammed with prosperous adventurers and broken gold
seekers, all equally as mad to get to the Outside as they had been
originally to get to the Inside. Near the gang-plank, Scott was shaking
hands with Matt, who was preparing to go ashore. But Matt’s hand went
limp in the other’s grasp as his gaze shot past and remained fixed on
something behind him. Scott turned to see. Sitting on the deck several
feet away and watching wistfully was White Fang.

The dog-musher swore softly, in awe-stricken accents. Scott could only
look in wonder.

“Did you lock the front door?” Matt demanded. The other nodded, and
asked, “How about the back?”

“You just bet I did,” was the fervent reply.

White Fang flattened his ears ingratiatingly, but remained where he
was, making no attempt to approach.

“I’ll have to take ’m ashore with me.”

Matt made a couple of steps toward White Fang, but the latter slid away
from him. The dog-musher made a rush of it, and White Fang dodged
between the legs of a group of men. Ducking, turning, doubling, he slid
about the deck, eluding the other’s efforts to capture him.

But when the love-master spoke, White Fang came to him with prompt
obedience.

“Won’t come to the hand that’s fed ’m all these months,” the dog-musher
muttered resentfully. “And you—you ain’t never fed ’m after them first
days of gettin’ acquainted. I’m blamed if I can see how he works it out
that you’re the boss.”

Scott, who had been patting White Fang, suddenly bent closer and
pointed out fresh-made cuts on his muzzle, and a gash between the eyes.

Matt bent over and passed his hand along White Fang’s belly.

“We plump forgot the window. He’s all cut an’ gouged underneath. Must
‘a’ butted clean through it, b’gosh!”

But Weedon Scott was not listening. He was thinking rapidly. The
\emph{Aurora’s} whistle hooted a final announcement of departure. Men were
scurrying down the gang-plank to the shore. Matt loosened the bandana
from his own neck and started to put it around White Fang’s. Scott
grasped the dog-musher’s hand.

“Good-bye, Matt, old man. About the wolf—you needn’t write. You see,
I’ve . . . !”

“What!” the dog-musher exploded. “You don’t mean to say . . .?”

“The very thing I mean. Here’s your bandana. I’ll write to you about
him.”

Matt paused halfway down the gang-plank.

“He’ll never stand the climate!” he shouted back. “Unless you clip ’m
in warm weather!”

The gang-plank was hauled in, and the \emph{Aurora} swung out from the bank.
Weedon Scott waved a last good-bye. Then he turned and bent over White
Fang, standing by his side.

“Now growl, damn you, growl,” he said, as he patted the responsive head
and rubbed the flattening ears.

\chapter{The Southland}

White Fang landed from the steamer in San Francisco. He was appalled.
Deep in him, below any reasoning process or act of consciousness, he
had associated power with godhead. And never had the white men seemed
such marvellous gods as now, when he trod the slimy pavement of San
Francisco. The log cabins he had known were replaced by towering
buildings. The streets were crowded with perils—waggons, carts,
automobiles; great, straining horses pulling huge trucks; and monstrous
cable and electric cars hooting and clanging through the midst,
screeching their insistent menace after the manner of the lynxes he had
known in the northern woods.

All this was the manifestation of power. Through it all, behind it all,
was man, governing and controlling, expressing himself, as of old, by
his mastery over matter. It was colossal, stunning. White Fang was
awed. Fear sat upon him. As in his cubhood he had been made to feel his
smallness and puniness on the day he first came in from the Wild to the
village of Grey Beaver, so now, in his full-grown stature and pride of
strength, he was made to feel small and puny. And there were so many
gods! He was made dizzy by the swarming of them. The thunder of the
streets smote upon his ears. He was bewildered by the tremendous and
endless rush and movement of things. As never before, he felt his
dependence on the love-master, close at whose heels he followed, no
matter what happened never losing sight of him.

But White Fang was to have no more than a nightmare vision of the
city—an experience that was like a bad dream, unreal and terrible, that
haunted him for long after in his dreams. He was put into a baggage-car
by the master, chained in a corner in the midst of heaped trunks and
valises. Here a squat and brawny god held sway, with much noise,
hurling trunks and boxes about, dragging them in through the door and
tossing them into the piles, or flinging them out of the door, smashing
and crashing, to other gods who awaited them.

And here, in this inferno of luggage, was White Fang deserted by the
master. Or at least White Fang thought he was deserted, until he
smelled out the master’s canvas clothes-bags alongside of him, and
proceeded to mount guard over them.

“’Bout time you come,” growled the god of the car, an hour later, when
Weedon Scott appeared at the door. “That dog of yourn won’t let me lay
a finger on your stuff.”

White Fang emerged from the car. He was astonished. The nightmare city
was gone. The car had been to him no more than a room in a house, and
when he had entered it the city had been all around him. In the
interval the city had disappeared. The roar of it no longer dinned upon
his ears. Before him was smiling country, streaming with sunshine, lazy
with quietude. But he had little time to marvel at the transformation.
He accepted it as he accepted all the unaccountable doings and
manifestations of the gods. It was their way.

There was a carriage waiting. A man and a woman approached the master.
The woman’s arms went out and clutched the master around the neck—a
hostile act! The next moment Weedon Scott had torn loose from the
embrace and closed with White Fang, who had become a snarling, raging
demon.

“It’s all right, mother,” Scott was saying as he kept tight hold of
White Fang and placated him. “He thought you were going to injure me,
and he wouldn’t stand for it. It’s all right. It’s all right. He’ll
learn soon enough.”

“And in the meantime I may be permitted to love my son when his dog is
not around,” she laughed, though she was pale and weak from the fright.

She looked at White Fang, who snarled and bristled and glared
malevolently.

“He’ll have to learn, and he shall, without postponement,” Scott said.

He spoke softly to White Fang until he had quieted him, then his voice
became firm.

“Down, sir! Down with you!”

This had been one of the things taught him by the master, and White
Fang obeyed, though he lay down reluctantly and sullenly.

“Now, mother.”

Scott opened his arms to her, but kept his eyes on White Fang.

“Down!” he warned. “Down!”

White Fang, bristling silently, half-crouching as he rose, sank back
and watched the hostile act repeated. But no harm came of it, nor of
the embrace from the strange man-god that followed. Then the
clothes-bags were taken into the carriage, the strange gods and the
love-master followed, and White Fang pursued, now running vigilantly
behind, now bristling up to the running horses and warning them that he
was there to see that no harm befell the god they dragged so swiftly
across the earth.

At the end of fifteen minutes, the carriage swung in through a stone
gateway and on between a double row of arched and interlacing walnut
trees. On either side stretched lawns, their broad sweep broken here
and there by great sturdy-limbed oaks. In the near distance, in
contrast with the young-green of the tended grass, sunburnt hay-fields
showed tan and gold; while beyond were the tawny hills and upland
pastures. From the head of the lawn, on the first soft swell from the
valley-level, looked down the deep-porched, many-windowed house.

Little opportunity was given White Fang to see all this. Hardly had the
carriage entered the grounds, when he was set upon by a sheep-dog,
bright-eyed, sharp-muzzled, righteously indignant and angry. It was
between him and the master, cutting him off. White Fang snarled no
warning, but his hair bristled as he made his silent and deadly rush.
This rush was never completed. He halted with awkward abruptness, with
stiff fore-legs bracing himself against his momentum, almost sitting
down on his haunches, so desirous was he of avoiding contact with the
dog he was in the act of attacking. It was a female, and the law of his
kind thrust a barrier between. For him to attack her would require
nothing less than a violation of his instinct.

But with the sheep-dog it was otherwise. Being a female, she possessed
no such instinct. On the other hand, being a sheep-dog, her instinctive
fear of the Wild, and especially of the wolf, was unusually keen. White
Fang was to her a wolf, the hereditary marauder who had preyed upon her
flocks from the time sheep were first herded and guarded by some dim
ancestor of hers. And so, as he abandoned his rush at her and braced
himself to avoid the contact, she sprang upon him. He snarled
involuntarily as he felt her teeth in his shoulder, but beyond this
made no offer to hurt her. He backed away, stiff-legged with
self-consciousness, and tried to go around her. He dodged this way and
that, and curved and turned, but to no purpose. She remained always
between him and the way he wanted to go.

“Here, Collie!” called the strange man in the carriage.

Weedon Scott laughed.

“Never mind, father. It is good discipline. White Fang will have to
learn many things, and it’s just as well that he begins now. He’ll
adjust himself all right.”

The carriage drove on, and still Collie blocked White Fang’s way. He
tried to outrun her by leaving the drive and circling across the lawn
but she ran on the inner and smaller circle, and was always there,
facing him with her two rows of gleaming teeth. Back he circled, across
the drive to the other lawn, and again she headed him off.

The carriage was bearing the master away. White Fang caught glimpses of
it disappearing amongst the trees. The situation was desperate. He
essayed another circle. She followed, running swiftly. And then,
suddenly, he turned upon her. It was his old fighting trick. Shoulder
to shoulder, he struck her squarely. Not only was she overthrown. So
fast had she been running that she rolled along, now on her back, now
on her side, as she struggled to stop, clawing gravel with her feet and
crying shrilly her hurt pride and indignation.

White Fang did not wait. The way was clear, and that was all he had
wanted. She took after him, never ceasing her outcry. It was the
straightaway now, and when it came to real running, White Fang could
teach her things. She ran frantically, hysterically, straining to the
utmost, advertising the effort she was making with every leap: and all
the time White Fang slid smoothly away from her silently, without
effort, gliding like a ghost over the ground.

As he rounded the house to the \emph{porte-cochère}, he came upon the
carriage. It had stopped, and the master was alighting. At this moment,
still running at top speed, White Fang became suddenly aware of an
attack from the side. It was a deer-hound rushing upon him. White Fang
tried to face it. But he was going too fast, and the hound was too
close. It struck him on the side; and such was his forward momentum and
the unexpectedness of it, White Fang was hurled to the ground and
rolled clear over. He came out of the tangle a spectacle of malignancy,
ears flattened back, lips writhing, nose wrinkling, his teeth clipping
together as the fangs barely missed the hound’s soft throat.

The master was running up, but was too far away; and it was Collie that
saved the hound’s life. Before White Fang could spring in and deliver
the fatal stroke, and just as he was in the act of springing in, Collie
arrived. She had been out-manoeuvred and out-run, to say nothing of her
having been unceremoniously tumbled in the gravel, and her arrival was
like that of a tornado—made up of offended dignity, justifiable wrath,
and instinctive hatred for this marauder from the Wild. She struck
White Fang at right angles in the midst of his spring, and again he was
knocked off his feet and rolled over.

The next moment the master arrived, and with one hand held White Fang,
while the father called off the dogs.

“I say, this is a pretty warm reception for a poor lone wolf from the
Arctic,” the master said, while White Fang calmed down under his
caressing hand. “In all his life he’s only been known once to go off
his feet, and here he’s been rolled twice in thirty seconds.”

The carriage had driven away, and other strange gods had appeared from
out the house. Some of these stood respectfully at a distance; but two
of them, women, perpetrated the hostile act of clutching the master
around the neck. White Fang, however, was beginning to tolerate this
act. No harm seemed to come of it, while the noises the gods made were
certainly not threatening. These gods also made overtures to White
Fang, but he warned them off with a snarl, and the master did likewise
with word of mouth. At such times White Fang leaned in close against
the master’s legs and received reassuring pats on the head.

The hound, under the command, “Dick! Lie down, sir!” had gone up the
steps and lain down to one side of the porch, still growling and
keeping a sullen watch on the intruder. Collie had been taken in charge
by one of the woman-gods, who held arms around her neck and petted and
caressed her; but Collie was very much perplexed and worried, whining
and restless, outraged by the permitted presence of this wolf and
confident that the gods were making a mistake.

All the gods started up the steps to enter the house. White Fang
followed closely at the master’s heels. Dick, on the porch, growled,
and White Fang, on the steps, bristled and growled back.

“Take Collie inside and leave the two of them to fight it out,”
suggested Scott’s father. “After that they’ll be friends.”

“Then White Fang, to show his friendship, will have to be chief mourner
at the funeral,” laughed the master.

The elder Scott looked incredulously, first at White Fang, then at
Dick, and finally at his son.

“You mean . . .?”

Weedon nodded his head. “I mean just that. You’d have a dead Dick
inside one minute—two minutes at the farthest.”

He turned to White Fang. “Come on, you wolf. It’s you that’ll have to
come inside.”

White Fang walked stiff-legged up the steps and across the porch, with
tail rigidly erect, keeping his eyes on Dick to guard against a flank
attack, and at the same time prepared for whatever fierce manifestation
of the unknown that might pounce out upon him from the interior of the
house. But no thing of fear pounced out, and when he had gained the
inside he scouted carefully around, looking at it and finding it not.
Then he lay down with a contented grunt at the master’s feet, observing
all that went on, ever ready to spring to his feet and fight for life
with the terrors he felt must lurk under the trap-roof of the dwelling.

\chapter{The God's Domain}

Not only was White Fang adaptable by nature, but he had travelled much,
and knew the meaning and necessity of adjustment. Here, in Sierra
Vista, which was the name of Judge Scott’s place, White Fang quickly
began to make himself at home. He had no further serious trouble with
the dogs. They knew more about the ways of the Southland gods than did
he, and in their eyes he had qualified when he accompanied the gods
inside the house. Wolf that he was, and unprecedented as it was, the
gods had sanctioned his presence, and they, the dogs of the gods, could
only recognise this sanction.

Dick, perforce, had to go through a few stiff formalities at first,
after which he calmly accepted White Fang as an addition to the
premises. Had Dick had his way, they would have been good friends;
but White Fang was averse to friendship. All he asked of other dogs was
to be let alone. His whole life he had kept aloof from his kind, and he
still desired to keep aloof. Dick’s overtures bothered him, so he
snarled Dick away. In the north he had learned the lesson that he must
let the master’s dogs alone, and he did not forget that lesson now. But
he insisted on his own privacy and self-seclusion, and so thoroughly
ignored Dick that that good-natured creature finally gave him up and
scarcely took as much interest in him as in the hitching-post near the
stable.

Not so with Collie. While she accepted him because it was the mandate
of the gods, that was no reason that she should leave him in peace.
Woven into her being was the memory of countless crimes he and his had
perpetrated against her ancestry. Not in a day nor a generation were
the ravaged sheepfolds to be forgotten. All this was a spur to her,
pricking her to retaliation. She could not fly in the face of the gods
who permitted him, but that did not prevent her from making life
miserable for him in petty ways. A feud, ages old, was between them,
and she, for one, would see to it that he was reminded.

So Collie took advantage of her sex to pick upon White Fang and
maltreat him. His instinct would not permit him to attack her, while
her persistence would not permit him to ignore her. When she rushed at
him he turned his fur-protected shoulder to her sharp teeth and walked
away stiff-legged and stately. When she forced him too hard, he was
compelled to go about in a circle, his shoulder presented to her, his
head turned from her, and on his face and in his eyes a patient and
bored expression. Sometimes, however, a nip on his hind-quarters
hastened his retreat and made it anything but stately. But as a rule he
managed to maintain a dignity that was almost solemnity. He ignored her
existence whenever it was possible, and made it a point to keep out of
her way. When he saw or heard her coming, he got up and walked off.

There was much in other matters for White Fang to learn. Life in the
Northland was simplicity itself when compared with the complicated
affairs of Sierra Vista. First of all, he had to learn the family of
the master. In a way he was prepared to do this. As Mit-sah and
Kloo-kooch had belonged to Grey Beaver, sharing his food, his fire, and
his blankets, so now, at Sierra Vista, belonged to the love-master all
the denizens of the house.

But in this matter there was a difference, and many differences. Sierra
Vista was a far vaster affair than the tepee of Grey Beaver. There were
many persons to be considered. There was Judge Scott, and there was his
wife. There were the master’s two sisters, Beth and Mary. There was his
wife, Alice, and then there were his children, Weedon and Maud,
toddlers of four and six. There was no way for anybody to tell him
about all these people, and of blood-ties and relationship he knew
nothing whatever and never would be capable of knowing. Yet he quickly
worked it out that all of them belonged to the master. Then, by
observation, whenever opportunity offered, by study of action, speech,
and the very intonations of the voice, he slowly learned the intimacy
and the degree of favour they enjoyed with the master. And by this
ascertained standard, White Fang treated them accordingly. What was of
value to the master he valued; what was dear to the master was to be
cherished by White Fang and guarded carefully.

Thus it was with the two children. All his life he had disliked
children. He hated and feared their hands. The lessons were not tender
that he had learned of their tyranny and cruelty in the days of the
Indian villages. When Weedon and Maud had first approached him, he
growled warningly and looked malignant. A cuff from the master and a
sharp word had then compelled him to permit their caresses, though he
growled and growled under their tiny hands, and in the growl there was
no crooning note. Later, he observed that the boy and girl were of
great value in the master’s eyes. Then it was that no cuff nor sharp
word was necessary before they could pat him.

Yet White Fang was never effusively affectionate. He yielded to the
master’s children with an ill but honest grace, and endured their
fooling as one would endure a painful operation. When he could no
longer endure, he would get up and stalk determinedly away from them.
But after a time, he grew even to like the children. Still he was not
demonstrative. He would not go up to them. On the other hand, instead
of walking away at sight of them, he waited for them to come to him.
And still later, it was noticed that a pleased light came into his eyes
when he saw them approaching, and that he looked after them with an
appearance of curious regret when they left him for other amusements.

All this was a matter of development, and took time. Next in his
regard, after the children, was Judge Scott. There were two reasons,
possibly, for this. First, he was evidently a valuable possession of
the master’s, and next, he was undemonstrative. White Fang liked to lie
at his feet on the wide porch when he read the newspaper, from time to
time favouring White Fang with a look or a word—untroublesome tokens
that he recognised White Fang’s presence and existence. But this was
only when the master was not around. When the master appeared, all
other beings ceased to exist so far as White Fang was concerned.

White Fang allowed all the members of the family to pet him and make
much of him; but he never gave to them what he gave to the master. No
caress of theirs could put the love-croon into his throat, and, try as
they would, they could never persuade him into snuggling against them.
This expression of abandon and surrender, of absolute trust, he
reserved for the master alone. In fact, he never regarded the members
of the family in any other light than possessions of the love-master.

Also White Fang had early come to differentiate between the family and
the servants of the household. The latter were afraid of him, while he
merely refrained from attacking them. This because he considered that
they were likewise possessions of the master. Between White Fang and
them existed a neutrality and no more. They cooked for the master and
washed the dishes and did other things just as Matt had done up in the
Klondike. They were, in short, appurtenances of the household.

Outside the household there was even more for White Fang to learn. The
master’s domain was wide and complex, yet it had its metes and bounds.
The land itself ceased at the county road. Outside was the common
domain of all gods—the roads and streets. Then inside other fences were
the particular domains of other gods. A myriad laws governed all these
things and determined conduct; yet he did not know the speech of the
gods, nor was there any way for him to learn save by experience. He
obeyed his natural impulses until they ran him counter to some law.
When this had been done a few times, he learned the law and after that
observed it.

But most potent in his education was the cuff of the master’s hand, the
censure of the master’s voice. Because of White Fang’s very great love,
a cuff from the master hurt him far more than any beating Grey Beaver
or Beauty Smith had ever given him. They had hurt only the flesh of
him; beneath the flesh the spirit had still raged, splendid and
invincible. But with the master the cuff was always too light to hurt
the flesh. Yet it went deeper. It was an expression of the master’s
disapproval, and White Fang’s spirit wilted under it.

In point of fact, the cuff was rarely administered. The master’s voice
was sufficient. By it White Fang knew whether he did right or not. By
it he trimmed his conduct and adjusted his actions. It was the compass
by which he steered and learned to chart the manners of a new land and
life.

In the Northland, the only domesticated animal was the dog. All other
animals lived in the Wild, and were, when not too formidable, lawful
spoil for any dog. All his days White Fang had foraged among the live
things for food. It did not enter his head that in the Southland it was
otherwise. But this he was to learn early in his residence in Santa
Clara Valley. Sauntering around the corner of the house in the early
morning, he came upon a chicken that had escaped from the chicken-yard.
White Fang’s natural impulse was to eat it. A couple of bounds, a flash
of teeth and a frightened squawk, and he had scooped in the adventurous
fowl. It was farm-bred and fat and tender; and White Fang licked his
chops and decided that such fare was good.

Later in the day, he chanced upon another stray chicken near the
stables. One of the grooms ran to the rescue. He did not know White
Fang’s breed, so for weapon he took a light buggy-whip. At the first
cut of the whip, White Fang left the chicken for the man. A club might
have stopped White Fang, but not a whip. Silently, without flinching,
he took a second cut in his forward rush, and as he leaped for the
throat the groom cried out, “My God!” and staggered backward. He
dropped the whip and shielded his throat with his arms. In consequence,
his forearm was ripped open to the bone.

The man was badly frightened. It was not so much White Fang’s ferocity
as it was his silence that unnerved the groom. Still protecting his
throat and face with his torn and bleeding arm, he tried to retreat to
the barn. And it would have gone hard with him had not Collie appeared
on the scene. As she had saved Dick’s life, she now saved the groom’s.
She rushed upon White Fang in frenzied wrath. She had been right. She
had known better than the blundering gods. All her suspicions were
justified. Here was the ancient marauder up to his old tricks again.

The groom escaped into the stables, and White Fang backed away before
Collie’s wicked teeth, or presented his shoulder to them and circled
round and round. But Collie did not give over, as was her wont, after a
decent interval of chastisement. On the contrary, she grew more excited
and angry every moment, until, in the end, White Fang flung dignity to
the winds and frankly fled away from her across the fields.

“He’ll learn to leave chickens alone,” the master said. “But I can’t
give him the lesson until I catch him in the act.”

Two nights later came the act, but on a more generous scale than the
master had anticipated. White Fang had observed closely the
chicken-yards and the habits of the chickens. In the night-time, after
they had gone to roost, he climbed to the top of a pile of newly hauled
lumber. From there he gained the roof of a chicken-house, passed over
the ridgepole and dropped to the ground inside. A moment later he was
inside the house, and the slaughter began.

In the morning, when the master came out on to the porch, fifty white
Leghorn hens, laid out in a row by the groom, greeted his eyes. He
whistled to himself, softly, first with surprise, and then, at the end,
with admiration. His eyes were likewise greeted by White Fang, but
about the latter there were no signs of shame nor guilt. He carried
himself with pride, as though, forsooth, he had achieved a deed
praiseworthy and meritorious. There was about him no consciousness of
sin. The master’s lips tightened as he faced the disagreeable task.
Then he talked harshly to the unwitting culprit, and in his voice there
was nothing but godlike wrath. Also, he held White Fang’s nose down to
the slain hens, and at the same time cuffed him soundly.

White Fang never raided a chicken-roost again. It was against the law,
and he had learned it. Then the master took him into the chicken-yards.
White Fang’s natural impulse, when he saw the live food fluttering
about him and under his very nose, was to spring upon it. He obeyed the
impulse, but was checked by the master’s voice. They continued in the
yards for half an hour. Time and again the impulse surged over White
Fang, and each time, as he yielded to it, he was checked by the
master’s voice. Thus it was he learned the law, and ere he left the
domain of the chickens, he had learned to ignore their existence.

“You can never cure a chicken-killer.” Judge Scott shook his head sadly
at luncheon table, when his son narrated the lesson he had given White
Fang. “Once they’ve got the habit and the taste of blood . . .” Again
he shook his head sadly.

But Weedon Scott did not agree with his father. “I’ll tell you what
I’ll do,” he challenged finally. “I’ll lock White Fang in with the
chickens all afternoon.”

“But think of the chickens,” objected the judge.

“And furthermore,” the son went on, “for every chicken he kills, I’ll
pay you one dollar gold coin of the realm.”

“But you should penalise father, too,” interposed Beth.

Her sister seconded her, and a chorus of approval arose from around the
table. Judge Scott nodded his head in agreement.

“All right.” Weedon Scott pondered for a moment. “And if, at the end of
the afternoon White Fang hasn’t harmed a chicken, for every ten minutes
of the time he has spent in the yard, you will have to say to him,
gravely and with deliberation, just as if you were sitting on the bench
and solemnly passing judgment, ‘White Fang, you are smarter than I
thought.’”

From hidden points of vantage the family watched the performance. But
it was a fizzle. Locked in the yard and there deserted by the master,
White Fang lay down and went to sleep. Once he got up and walked over
to the trough for a drink of water. The chickens he calmly ignored. So
far as he was concerned they did not exist. At four o’clock he executed
a running jump, gained the roof of the chicken-house and leaped to the
ground outside, whence he sauntered gravely to the house. He had
learned the law. And on the porch, before the delighted family, Judge
Scott, face to face with White Fang, said slowly and solemnly, sixteen
times, “White Fang, you are smarter than I thought.”

But it was the multiplicity of laws that befuddled White Fang and often
brought him into disgrace. He had to learn that he must not touch the
chickens that belonged to other gods. Then there were cats, and
rabbits, and turkeys; all these he must let alone. In fact, when he had
but partly learned the law, his impression was that he must leave all
live things alone. Out in the back-pasture, a quail could flutter up
under his nose unharmed. All tense and trembling with eagerness and
desire, he mastered his instinct and stood still. He was obeying the
will of the gods.

And then, one day, again out in the back-pasture, he saw Dick start a
jackrabbit and run it. The master himself was looking on and did not
interfere. Nay, he encouraged White Fang to join in the chase. And thus
he learned that there was no taboo on jackrabbits. In the end he worked
out the complete law. Between him and all domestic animals there must
be no hostilities. If not amity, at least neutrality must obtain. But
the other animals—the squirrels, and quail, and cottontails, were
creatures of the Wild who had never yielded allegiance to man. They
were the lawful prey of any dog. It was only the tame that the gods
protected, and between the tame deadly strife was not permitted. The
gods held the power of life and death over their subjects, and the gods
were jealous of their power.

Life was complex in the Santa Clara Valley after the simplicities of
the Northland. And the chief thing demanded by these intricacies of
civilisation was control, restraint—a poise of self that was as
delicate as the fluttering of gossamer wings and at the same time as
rigid as steel. Life had a thousand faces, and White Fang found he must
meet them all—thus, when he went to town, in to San Jose, running
behind the carriage or loafing about the streets when the carriage
stopped. Life flowed past him, deep and wide and varied, continually
impinging upon his senses, demanding of him instant and endless
adjustments and correspondences, and compelling him, almost always, to
suppress his natural impulses.

There were butcher-shops where meat hung within reach. This meat he
must not touch. There were cats at the houses the master visited that
must be let alone. And there were dogs everywhere that snarled at him
and that he must not attack. And then, on the crowded sidewalks there
were persons innumerable whose attention he attracted. They would stop
and look at him, point him out to one another, examine him, talk of
him, and, worst of all, pat him. And these perilous contacts from all
these strange hands he must endure. Yet this endurance he achieved.
Furthermore, he got over being awkward and self-conscious. In a lofty
way he received the attentions of the multitudes of strange gods. With
condescension he accepted their condescension. On the other hand, there
was something about him that prevented great familiarity. They patted
him on the head and passed on, contented and pleased with their own
daring.

But it was not all easy for White Fang. Running behind the carriage in
the outskirts of San Jose, he encountered certain small boys who made a
practice of flinging stones at him. Yet he knew that it was not
permitted him to pursue and drag them down. Here he was compelled to
violate his instinct of self-preservation, and violate it he did, for
he was becoming tame and qualifying himself for civilisation.

Nevertheless, White Fang was not quite satisfied with the arrangement.
He had no abstract ideas about justice and fair play. But there is a
certain sense of equity that resides in life, and it was this sense in
him that resented the unfairness of his being permitted no defence
against the stone-throwers. He forgot that in the covenant entered into
between him and the gods they were pledged to care for him and defend
him. But one day the master sprang from the carriage, whip in hand, and
gave the stone-throwers a thrashing. After that they threw stones no
more, and White Fang understood and was satisfied.

One other experience of similar nature was his. On the way to town,
hanging around the saloon at the cross-roads, were three dogs that made
a practice of rushing out upon him when he went by. Knowing his deadly
method of fighting, the master had never ceased impressing upon White
Fang the law that he must not fight. As a result, having learned the
lesson well, White Fang was hard put whenever he passed the cross-roads
saloon. After the first rush, each time, his snarl kept the three dogs
at a distance but they trailed along behind, yelping and bickering and
insulting him. This endured for some time. The men at the saloon even
urged the dogs on to attack White Fang. One day they openly sicked the
dogs on him. The master stopped the carriage.

“Go to it,” he said to White Fang.

But White Fang could not believe. He looked at the master, and he
looked at the dogs. Then he looked back eagerly and questioningly at
the master.

The master nodded his head. “Go to them, old fellow. Eat them up.”

White Fang no longer hesitated. He turned and leaped silently among his
enemies. All three faced him. There was a great snarling and growling,
a clashing of teeth and a flurry of bodies. The dust of the road arose
in a cloud and screened the battle. But at the end of several minutes
two dogs were struggling in the dirt and the third was in full flight.
He leaped a ditch, went through a rail fence, and fled across a field.
White Fang followed, sliding over the ground in wolf fashion and with
wolf speed, swiftly and without noise, and in the centre of the field
he dragged down and slew the dog.

With this triple killing his main troubles with dogs ceased. The word
went up and down the valley, and men saw to it that their dogs did not
molest the Fighting Wolf.

\chapter{The Call of Kind}

The months came and went. There was plenty of food and no work in the
Southland, and White Fang lived fat and prosperous and happy. Not alone
was he in the geographical Southland, for he was in the Southland of
life. Human kindness was like a sun shining upon him, and he flourished
like a flower planted in good soil.

And yet he remained somehow different from other dogs. He knew the law
even better than did the dogs that had known no other life, and he
observed the law more punctiliously; but still there was about him a
suggestion of lurking ferocity, as though the Wild still lingered in
him and the wolf in him merely slept.

He never chummed with other dogs. Lonely he had lived, so far as his
kind was concerned, and lonely he would continue to live. In his
puppyhood, under the persecution of Lip-lip and the puppy-pack, and in
his fighting days with Beauty Smith, he had acquired a fixed aversion
for dogs. The natural course of his life had been diverted, and,
recoiling from his kind, he had clung to the human.

Besides, all Southland dogs looked upon him with suspicion. He aroused
in them their instinctive fear of the Wild, and they greeted him always
with snarl and growl and belligerent hatred. He, on the other hand,
learned that it was not necessary to use his teeth upon them. His naked
fangs and writhing lips were uniformly efficacious, rarely failing to
send a bellowing on-rushing dog back on its haunches.

But there was one trial in White Fang’s life—Collie. She never gave him
a moment’s peace. She was not so amenable to the law as he. She defied
all efforts of the master to make her become friends with White Fang.
Ever in his ears was sounding her sharp and nervous snarl. She had
never forgiven him the chicken-killing episode, and persistently held
to the belief that his intentions were bad. She found him guilty before
the act, and treated him accordingly. She became a pest to him, like a
policeman following him around the stable and the hounds, and, if he
even so much as glanced curiously at a pigeon or chicken, bursting into
an outcry of indignation and wrath. His favourite way of ignoring her
was to lie down, with his head on his fore-paws, and pretend sleep.
This always dumfounded and silenced her.

With the exception of Collie, all things went well with White Fang. He
had learned control and poise, and he knew the law. He achieved a
staidness, and calmness, and philosophic tolerance. He no longer lived
in a hostile environment. Danger and hurt and death did not lurk
everywhere about him. In time, the unknown, as a thing of terror and
menace ever impending, faded away. Life was soft and easy. It flowed
along smoothly, and neither fear nor foe lurked by the way.

He missed the snow without being aware of it. “An unduly long summer,”
would have been his thought had he thought about it; as it was, he
merely missed the snow in a vague, subconscious way. In the same
fashion, especially in the heat of summer when he suffered from the
sun, he experienced faint longings for the Northland. Their only effect
upon him, however, was to make him uneasy and restless without his
knowing what was the matter.

White Fang had never been very demonstrative. Beyond his snuggling and
the throwing of a crooning note into his love-growl, he had no way of
expressing his love. Yet it was given him to discover a third way. He
had always been susceptible to the laughter of the gods. Laughter had
affected him with madness, made him frantic with rage. But he did not
have it in him to be angry with the love-master, and when that god
elected to laugh at him in a good-natured, bantering way, he was
nonplussed. He could feel the pricking and stinging of the old anger as
it strove to rise up in him, but it strove against love. He could not
be angry; yet he had to do something. At first he was dignified, and
the master laughed the harder. Then he tried to be more dignified, and
the master laughed harder than before. In the end, the master laughed
him out of his dignity. His jaws slightly parted, his lips lifted a
little, and a quizzical expression that was more love than humour came
into his eyes. He had learned to laugh.

Likewise he learned to romp with the master, to be tumbled down and
rolled over, and be the victim of innumerable rough tricks. In return
he feigned anger, bristling and growling ferociously, and clipping his
teeth together in snaps that had all the seeming of deadly intention.
But he never forgot himself. Those snaps were always delivered on the
empty air. At the end of such a romp, when blow and cuff and snap and
snarl were fast and furious, they would break off suddenly and stand
several feet apart, glaring at each other. And then, just as suddenly,
like the sun rising on a stormy sea, they would begin to laugh. This
would always culminate with the master’s arms going around White Fang’s
neck and shoulders while the latter crooned and growled his love-song.

But nobody else ever romped with White Fang. He did not permit it. He
stood on his dignity, and when they attempted it, his warning snarl and
bristling mane were anything but playful. That he allowed the master
these liberties was no reason that he should be a common dog, loving
here and loving there, everybody’s property for a romp and good time.
He loved with single heart and refused to cheapen himself or his love.

The master went out on horseback a great deal, and to accompany him was
one of White Fang’s chief duties in life. In the Northland he had
evidenced his fealty by toiling in the harness; but there were no sleds
in the Southland, nor did dogs pack burdens on their backs. So he
rendered fealty in the new way, by running with the master’s horse. The
longest day never played White Fang out. His was the gait of the wolf,
smooth, tireless and effortless, and at the end of fifty miles he would
come in jauntily ahead of the horse.

It was in connection with the riding, that White Fang achieved one
other mode of expression—remarkable in that he did it but twice in all
his life. The first time occurred when the master was trying to teach a
spirited thoroughbred the method of opening and closing gates without
the rider’s dismounting. Time and again and many times he ranged the
horse up to the gate in the effort to close it and each time the horse
became frightened and backed and plunged away. It grew more nervous and
excited every moment. When it reared, the master put the spurs to it
and made it drop its fore-legs back to earth, whereupon it would begin
kicking with its hind-legs. White Fang watched the performance with
increasing anxiety until he could contain himself no longer, when he
sprang in front of the horse and barked savagely and warningly.

Though he often tried to bark thereafter, and the master encouraged
him, he succeeded only once, and then it was not in the master’s
presence. A scamper across the pasture, a jackrabbit rising suddenly
under the horse’s feet, a violent sheer, a stumble, a fall to earth,
and a broken leg for the master, was the cause of it. White Fang sprang
in a rage at the throat of the offending horse, but was checked by the
master’s voice.

“Home! Go home!” the master commanded when he had ascertained his
injury.

White Fang was disinclined to desert him. The master thought of writing
a note, but searched his pockets vainly for pencil and paper. Again he
commanded White Fang to go home.

The latter regarded him wistfully, started away, then returned and
whined softly. The master talked to him gently but seriously, and he
cocked his ears, and listened with painful intentness.

“That’s all right, old fellow, you just run along home,” ran the talk.
“Go on home and tell them what’s happened to me. Home with you, you
wolf. Get along home!”

White Fang knew the meaning of “home,” and though he did not understand
the remainder of the master’s language, he knew it was his will that he
should go home. He turned and trotted reluctantly away. Then he
stopped, undecided, and looked back over his shoulder.

“Go home!” came the sharp command, and this time he obeyed.

The family was on the porch, taking the cool of the afternoon, when
White Fang arrived. He came in among them, panting, covered with dust.

“Weedon’s back,” Weedon’s mother announced.

The children welcomed White Fang with glad cries and ran to meet him.
He avoided them and passed down the porch, but they cornered him
against a rocking-chair and the railing. He growled and tried to push
by them. Their mother looked apprehensively in their direction.

“I confess, he makes me nervous around the children,” she said. “I have
a dread that he will turn upon them unexpectedly some day.”

Growling savagely, White Fang sprang out of the corner, overturning the
boy and the girl. The mother called them to her and comforted them,
telling them not to bother White Fang.

“A wolf is a wolf!” commented Judge Scott. “There is no trusting one.”

“But he is not all wolf,” interposed Beth, standing for her brother in
his absence.

“You have only Weedon’s opinion for that,” rejoined the judge. “He
merely surmises that there is some strain of dog in White Fang; but as
he will tell you himself, he knows nothing about it. As for his
appearance—”

He did not finish his sentence. White Fang stood before him, growling
fiercely.

“Go away! Lie down, sir!” Judge Scott commanded.

White Fang turned to the love-master’s wife. She screamed with fright
as he seized her dress in his teeth and dragged on it till the frail
fabric tore away. By this time he had become the centre of interest.

He had ceased from his growling and stood, head up, looking into their
faces. His throat worked spasmodically, but made no sound, while he
struggled with all his body, convulsed with the effort to rid himself
of the incommunicable something that strained for utterance.

“I hope he is not going mad,” said Weedon’s mother. “I told Weedon that
I was afraid the warm climate would not agree with an Arctic animal.”

“He’s trying to speak, I do believe,” Beth announced.

At this moment speech came to White Fang, rushing up in a great burst
of barking.

“Something has happened to Weedon,” his wife said decisively.

They were all on their feet now, and White Fang ran down the steps,
looking back for them to follow. For the second and last time in his
life he had barked and made himself understood.

After this event he found a warmer place in the hearts of the Sierra
Vista people, and even the groom whose arm he had slashed admitted that
he was a wise dog even if he was a wolf. Judge Scott still held to the
same opinion, and proved it to everybody’s dissatisfaction by
measurements and descriptions taken from the encyclopaedia and various
works on natural history.

The days came and went, streaming their unbroken sunshine over the
Santa Clara Valley. But as they grew shorter and White Fang’s second
winter in the Southland came on, he made a strange discovery. Collie’s
teeth were no longer sharp. There was a playfulness about her nips and
a gentleness that prevented them from really hurting him. He forgot
that she had made life a burden to him, and when she disported herself
around him he responded solemnly, striving to be playful and becoming
no more than ridiculous.

One day she led him off on a long chase through the back-pasture land
into the woods. It was the afternoon that the master was to ride, and
White Fang knew it. The horse stood saddled and waiting at the door.
White Fang hesitated. But there was that in him deeper than all the law
he had learned, than the customs that had moulded him, than his love
for the master, than the very will to live of himself; and when, in the
moment of his indecision, Collie nipped him and scampered off, he
turned and followed after. The master rode alone that day; and in the
woods, side by side, White Fang ran with Collie, as his mother, Kiche,
and old One Eye had run long years before in the silent Northland
forest.

\chapter{The Sleeping Wolf}

It was about this time that the newspapers were full of the daring
escape of a convict from San Quentin prison. He was a ferocious man. He
had been ill-made in the making. He had not been born right, and he had
not been helped any by the moulding he had received at the hands of
society. The hands of society are harsh, and this man was a striking
sample of its handiwork. He was a beast—a human beast, it is true, but
nevertheless so terrible a beast that he can best be characterised as
carnivorous.

In San Quentin prison he had proved incorrigible. Punishment failed to
break his spirit. He could die dumb-mad and fighting to the last, but
he could not live and be beaten. The more fiercely he fought, the more
harshly society handled him, and the only effect of harshness was to
make him fiercer. Strait-jackets, starvation, and beatings and
clubbings were the wrong treatment for Jim Hall; but it was the
treatment he received. It was the treatment he had received from the
time he was a little pulpy boy in a San Francisco slum—soft clay in the
hands of society and ready to be formed into something.

It was during Jim Hall’s third term in prison that he encountered a
guard that was almost as great a beast as he. The guard treated him
unfairly, lied about him to the warden, lost his credits, persecuted
him. The difference between them was that the guard carried a bunch of
keys and a revolver. Jim Hall had only his naked hands and his teeth.
But he sprang upon the guard one day and used his teeth on the other’s
throat just like any jungle animal.

After this, Jim Hall went to live in the incorrigible cell. He lived
there three years. The cell was of iron, the floor, the walls, the
roof. He never left this cell. He never saw the sky nor the sunshine.
Day was a twilight and night was a black silence. He was in an iron
tomb, buried alive. He saw no human face, spoke to no human thing. When
his food was shoved in to him, he growled like a wild animal. He hated
all things. For days and nights he bellowed his rage at the universe.
For weeks and months he never made a sound, in the black silence eating
his very soul. He was a man and a monstrosity, as fearful a thing of
fear as ever gibbered in the visions of a maddened brain.

And then, one night, he escaped. The warders said it was impossible,
but nevertheless the cell was empty, and half in half out of it lay the
body of a dead guard. Two other dead guards marked his trail through
the prison to the outer walls, and he had killed with his hands to
avoid noise.

He was armed with the weapons of the slain guards—a live arsenal that
fled through the hills pursued by the organised might of society. A
heavy price of gold was upon his head. Avaricious farmers hunted him
with shot-guns. His blood might pay off a mortgage or send a son to
college. Public-spirited citizens took down their rifles and went out
after him. A pack of bloodhounds followed the way of his bleeding feet.
And the sleuth-hounds of the law, the paid fighting animals of society,
with telephone, and telegraph, and special train, clung to his trail
night and day.

Sometimes they came upon him, and men faced him like heroes, or
stampeded through barbed-wire fences to the delight of the commonwealth
reading the account at the breakfast table. It was after such
encounters that the dead and wounded were carted back to the towns, and
their places filled by men eager for the man-hunt.

And then Jim Hall disappeared. The bloodhounds vainly quested on the
lost trail. Inoffensive ranchers in remote valleys were held up by
armed men and compelled to identify themselves; while the remains of
Jim Hall were discovered on a dozen mountain-sides by greedy claimants
for blood-money.

In the meantime the newspapers were read at Sierra Vista, not so much
with interest as with anxiety. The women were afraid. Judge Scott
pooh-poohed and laughed, but not with reason, for it was in his last
days on the bench that Jim Hall had stood before him and received
sentence. And in open court-room, before all men, Jim Hall had
proclaimed that the day would come when he would wreak vengeance on the
Judge that sentenced him.

For once, Jim Hall was right. He was innocent of the crime for which he
was sentenced. It was a case, in the parlance of thieves and police, of
“rail-roading.” Jim Hall was being “rail-roaded” to prison for a crime
he had not committed. Because of the two prior convictions against him,
Judge Scott imposed upon him a sentence of fifty years.

Judge Scott did not know all things, and he did not know that he was
party to a police conspiracy, that the evidence was hatched and
perjured, that Jim Hall was guiltless of the crime charged. And Jim
Hall, on the other hand, did not know that Judge Scott was merely
ignorant. Jim Hall believed that the judge knew all about it and was
hand in glove with the police in the perpetration of the monstrous
injustice. So it was, when the doom of fifty years of living death was
uttered by Judge Scott, that Jim Hall, hating all things in the society
that misused him, rose up and raged in the court-room until dragged
down by half a dozen of his blue-coated enemies. To him, Judge Scott
was the keystone in the arch of injustice, and upon Judge Scott he
emptied the vials of his wrath and hurled the threats of his revenge
yet to come. Then Jim Hall went to his living death . . . and escaped.

Of all this White Fang knew nothing. But between him and Alice, the
master’s wife, there existed a secret. Each night, after Sierra Vista
had gone to bed, she rose and let in White Fang to sleep in the big
hall. Now White Fang was not a house-dog, nor was he permitted to sleep
in the house; so each morning, early, she slipped down and let him out
before the family was awake.

On one such night, while all the house slept, White Fang awoke and lay
very quietly. And very quietly he smelled the air and read the message
it bore of a strange god’s presence. And to his ears came sounds of the
strange god’s movements. White Fang burst into no furious outcry. It
was not his way. The strange god walked softly, but more softly walked
White Fang, for he had no clothes to rub against the flesh of his body.
He followed silently. In the Wild he had hunted live meat that was
infinitely timid, and he knew the advantage of surprise.

The strange god paused at the foot of the great staircase and listened,
and White Fang was as dead, so without movement was he as he watched
and waited. Up that staircase the way led to the love-master and to the
love-master’s dearest possessions. White Fang bristled, but waited. The
strange god’s foot lifted. He was beginning the ascent.

Then it was that White Fang struck. He gave no warning, with no snarl
anticipated his own action. Into the air he lifted his body in the
spring that landed him on the strange god’s back. White Fang clung with
his fore-paws to the man’s shoulders, at the same time burying his
fangs into the back of the man’s neck. He clung on for a moment, long
enough to drag the god over backward. Together they crashed to the
floor. White Fang leaped clear, and, as the man struggled to rise, was
in again with the slashing fangs.

Sierra Vista awoke in alarm. The noise from downstairs was as that of a
score of battling fiends. There were revolver shots. A man’s voice
screamed once in horror and anguish. There was a great snarling and
growling, and over all arose a smashing and crashing of furniture and
glass.

But almost as quickly as it had arisen, the commotion died away. The
struggle had not lasted more than three minutes. The frightened
household clustered at the top of the stairway. From below, as from out
an abyss of blackness, came up a gurgling sound, as of air bubbling
through water. Sometimes this gurgle became sibilant, almost a whistle.
But this, too, quickly died down and ceased. Then naught came up out of
the blackness save a heavy panting of some creature struggling sorely
for air.

Weedon Scott pressed a button, and the staircase and downstairs hall
were flooded with light. Then he and Judge Scott, revolvers in hand,
cautiously descended. There was no need for this caution. White Fang
had done his work. In the midst of the wreckage of overthrown and
smashed furniture, partly on his side, his face hidden by an arm, lay a
man. Weedon Scott bent over, removed the arm and turned the man’s face
upward. A gaping throat explained the manner of his death.

“Jim Hall,” said Judge Scott, and father and son looked significantly
at each other.

Then they turned to White Fang. He, too, was lying on his side. His
eyes were closed, but the lids slightly lifted in an effort to look at
them as they bent over him, and the tail was perceptibly agitated in a
vain effort to wag. Weedon Scott patted him, and his throat rumbled an
acknowledging growl. But it was a weak growl at best, and it quickly
ceased. His eyelids drooped and went shut, and his whole body seemed to
relax and flatten out upon the floor.

“He’s all in, poor devil,” muttered the master.

“We’ll see about that,” asserted the Judge, as he started for the
telephone.

“Frankly, he has one chance in a thousand,” announced the surgeon,
after he had worked an hour and a half on White Fang.

Dawn was breaking through the windows and dimming the electric lights.
With the exception of the children, the whole family was gathered about
the surgeon to hear his verdict.

“One broken hind-leg,” he went on. “Three broken ribs, one at least of
which has pierced the lungs. He has lost nearly all the blood in his
body. There is a large likelihood of internal injuries. He must have
been jumped upon. To say nothing of three bullet holes clear through
him. One chance in a thousand is really optimistic. He hasn’t a chance
in ten thousand.”

“But he mustn’t lose any chance that might be of help to him,” Judge
Scott exclaimed. “Never mind expense. Put him under the X-ray—anything.
Weedon, telegraph at once to San Francisco for Doctor Nichols. No
reflection on you, doctor, you understand; but he must have the
advantage of every chance.”

The surgeon smiled indulgently. “Of course I understand. He deserves
all that can be done for him. He must be nursed as you would nurse a
human being, a sick child. And don’t forget what I told you about
temperature. I’ll be back at ten o’clock again.”

White Fang received the nursing. Judge Scott’s suggestion of a trained
nurse was indignantly clamoured down by the girls, who themselves
undertook the task. And White Fang won out on the one chance in ten
thousand denied him by the surgeon.

The latter was not to be censured for his misjudgment. All his life he
had tended and operated on the soft humans of civilisation, who lived
sheltered lives and had descended out of many sheltered generations.
Compared with White Fang, they were frail and flabby, and clutched life
without any strength in their grip. White Fang had come straight from
the Wild, where the weak perish early and shelter is vouchsafed to
none. In neither his father nor his mother was there any weakness, nor
in the generations before them. A constitution of iron and the vitality
of the Wild were White Fang’s inheritance, and he clung to life, the
whole of him and every part of him, in spirit and in flesh, with the
tenacity that of old belonged to all creatures.

Bound down a prisoner, denied even movement by the plaster casts and
bandages, White Fang lingered out the weeks. He slept long hours and
dreamed much, and through his mind passed an unending pageant of
Northland visions. All the ghosts of the past arose and were with him.
Once again he lived in the lair with Kiche, crept trembling to the
knees of Grey Beaver to tender his allegiance, ran for his life before
Lip-lip and all the howling bedlam of the puppy-pack.

He ran again through the silence, hunting his living food through the
months of famine; and again he ran at the head of the team, the
gut-whips of Mit-sah and Grey Beaver snapping behind, their voices
crying “Ra! Raa!” when they came to a narrow passage and the team
closed together like a fan to go through. He lived again all his days
with Beauty Smith and the fights he had fought. At such times he
whimpered and snarled in his sleep, and they that looked on said that
his dreams were bad.

But there was one particular nightmare from which he suffered—the
clanking, clanging monsters of electric cars that were to him colossal
screaming lynxes. He would lie in a screen of bushes, watching for a
squirrel to venture far enough out on the ground from its tree-refuge.
Then, when he sprang out upon it, it would transform itself into an
electric car, menacing and terrible, towering over him like a mountain,
screaming and clanging and spitting fire at him. It was the same when
he challenged the hawk down out of the sky. Down out of the blue it
would rush, as it dropped upon him changing itself into the ubiquitous
electric car. Or again, he would be in the pen of Beauty Smith. Outside
the pen, men would be gathering, and he knew that a fight was on. He
watched the door for his antagonist to enter. The door would open, and
thrust in upon him would come the awful electric car. A thousand times
this occurred, and each time the terror it inspired was as vivid and
great as ever.

Then came the day when the last bandage and the last plaster cast were
taken off. It was a gala day. All Sierra Vista was gathered around. The
master rubbed his ears, and he crooned his love-growl. The master’s
wife called him the “Blessed Wolf,” which name was taken up with
acclaim and all the women called him the Blessed Wolf.

He tried to rise to his feet, and after several attempts fell down from
weakness. He had lain so long that his muscles had lost their cunning,
and all the strength had gone out of them. He felt a little shame
because of his weakness, as though, forsooth, he were failing the gods
in the service he owed them. Because of this he made heroic efforts to
arise and at last he stood on his four legs, tottering and swaying back
and forth.

“The Blessed Wolf!” chorused the women.

Judge Scott surveyed them triumphantly.

“Out of your own mouths be it,” he said. “Just as I contended right
along. No mere dog could have done what he did. He’s a wolf.”

“A Blessed Wolf,” amended the Judge’s wife.

“Yes, Blessed Wolf,” agreed the Judge. “And henceforth that shall be my
name for him.”

“He’ll have to learn to walk again,” said the surgeon; “so he might as
well start in right now. It won’t hurt him. Take him outside.”

And outside he went, like a king, with all Sierra Vista about him and
tending on him. He was very weak, and when he reached the lawn he lay
down and rested for a while.

Then the procession started on, little spurts of strength coming into
White Fang’s muscles as he used them and the blood began to surge
through them. The stables were reached, and there in the doorway, lay
Collie, a half-dozen pudgy puppies playing about her in the sun.

White Fang looked on with a wondering eye. Collie snarled warningly at
him, and he was careful to keep his distance. The master with his toe
helped one sprawling puppy toward him. He bristled suspiciously, but
the master warned him that all was well. Collie, clasped in the arms of
one of the women, watched him jealously and with a snarl warned him
that all was not well.

The puppy sprawled in front of him. He cocked his ears and watched it
curiously. Then their noses touched, and he felt the warm little tongue
of the puppy on his jowl. White Fang’s tongue went out, he knew not
why, and he licked the puppy’s face.

Hand-clapping and pleased cries from the gods greeted the performance.
He was surprised, and looked at them in a puzzled way. Then his
weakness asserted itself, and he lay down, his ears cocked, his head on
one side, as he watched the puppy. The other puppies came sprawling
toward him, to Collie’s great disgust; and he gravely permitted them to
clamber and tumble over him. At first, amid the applause of the gods,
he betrayed a trifle of his old self-consciousness and awkwardness.
This passed away as the puppies’ antics and mauling continued, and he
lay with half-shut patient eyes, drowsing in the sun.
\end{document}
